\documentclass{article}

\usepackage{amsmath,amssymb,amsthm,amsfonts,amscd,mathrsfs,mathtools}
\usepackage{biblatex}
\usepackage{anysize}
\usepackage{amsthm}
\usepackage{cleveref}

\usepackage[math-style=ISO]{unicode-math}

\defaultfontfeatures{Scale = MatchLowercase}
\setmainfont{TeX Gyre Pagella}
\setmathfont{Asana Math}

\usepackage{microtype}

\usepackage{verbatim}
\newenvironment{code}{\verbatim}{\endverbatim}

\makeatletter
\def\@seccntformat#1{%
  \expandafter\ifx\csname c@#1\endcsname\c@section\else
  \csname the#1\endcsname\quad
  \fi}
\makeatother

% \marginsize{2cm}{2cm}{0cm}{1cm}
\usepackage[a4paper,margin=2cm]{geometry}

\usepackage{hott}
\usepackage{macros}

\addbibresource{cites.bib}

\title{Something about commutativity and well-orders}
\author{Anonymous}

\date{\today}

\begin{document}
\maketitle

In this talk, we study free monoids, free commutative monoids, and their connections with sorting and well-orders, using
univalent type theory, implemented in Cubical Agda.

\subsection*{Background}

First, we review the basics of universal algebra, free algebras and their universal property.
%
We write $\Set$ for the category of $\mathsf{hSet}$s and functions.
%
A signature $\sigma$ is given by a type of operations with an arity function:
$\dsum{\op:\Set}{\ar \colon \op \to \Set}$.
%
This gives a signature endofunctor $\Sig(X) \defeq \dsum*{f\colon\op}{X^{\ar(f)}}$ on $\Set$.
%
A $\sigma$-structure~$\str{X}$ is an $\Sig$-algebra: $\dsum{X:\Set}{\alpha_X\colon\Sig(X) \to X}$, with carrier set~$X$,
and a morphism of $\sigma$-structures is a $\Sig$-algebra morphism,
giving the category of $\sigma$-algebras $\sigAlg$.

The forgetful functor $U_\sigma\colon\sigAlg$ to $\Set$ admits a left adjoint,
giving the free $\sigma$-algebra construction on a carrier set.
%
As is standard, this construction is given by an inductive type of trees $\tree{V}$,
generated by two constructors,
$\term{leaf}\colon V \to \tree{V}$ and $\term{node}\colon \Sigma_{\sigma}(\tree{V}) \to \tree{V}$.
%
$\tree{V}$ is canonically a $\sigma$-algebra $\str{T}(V) = (\tree{V}, \term{node})$,
with the universal map $\eta_{V} : V \to \tree{V}$ given by $\term{leaf}$.
%
The universal property states that, given any $\sigma$-structure $\str{X}$,
composition with $\eta_{V}$ is an equivalence:
$(\blank) \comp \eta_V \colon \sigAlg(\str{T}(V),\str{X}) \xto{\sim} (V \to X)$.
%
The inverse to this map is the extension operation $\ext{(\blank)}$,
which extends a map $f\colon V \to X$ to a homomorphism $\ext{f}\colon \str{T}(V) \to \str{X}$.

An equational signature $\varepsilon$ is given by a type of equations with an arity of free variables:
$\dsum{\eq:\Set}{\fv \colon \eq \to \Set}$.
%
A system of equations (or a theory $T$) over $(\sigma,\varepsilon)$ is given by
a pair of trees on the set of free variables, for each equation:
$\lhs,\rhs\colon\dfun{e:\eq}{\tree{\fv(e)}}$.
%
A $\sigma$-structure $\str{X}$ satisfies $T$, written $\str{X} \entails T$, if,
for each equation $e:\eq$ and $\rho \colon \fv(e) \to X$,
$\ext{\rho}(\lhs(e)) \id \ext{\rho}(\rhs(e))$.
%
The full subcategory of $\sigAlg$ given by $\sigma$-structures satisfying $T$ is the variety of $T$-algebras.

Using the framework we present different constructions of free monoids,
such as list $List$, free monoid as a HIT $FreeMon$, and as an index function $\Sigma (n : \mathbb{N}). Fin \: n \rightarrow X$ which we call $Array$.
We then extend our work by constructing free commutative monoids,
such as swapped-list $SList$ and free monoid quotiented by a permutation relation $QFreeMon$,
for example $List$ quotiented by permutation ($PList$)
and $Array$ quotiented by isomorphism on index $\Sigma (\sigma : Fin \: n \simeq \: Fin \: m). \: v = w \circ \sigma$ ($Bag$).
We prove these constructions are indeed free algebras by proving their universal property directly,
and we can derive canonical maps between different constructions directly using the universal property.

Using the same framework, we can axiomatize the notion of a sort function.

\newtheorem{myconj}{Conjecture}
\newtheorem{mydef}{Definition}
\newtheorem{mylemma}{Lemma}
\newtheorem{mythm}{Proposition}

\begin{mydef}
    Given a section $s : SList \: X \rightarrow List \: X$ to the canonical map $List \: X \rightarrow SList \: X$,
    $xs$ is said to be sorted if $xs$ is in the image of $s$.
\end{mydef}

We define the proposition $\textit{is-sorted} : List \: X \rightarrow \mathcal{U}$ to be
$\lambda xs. \: \exists (ys : SList \: X). \: s(ys) = xs$.
We also use the universal property of free monoid to define membership proofs for $List \: X$ and $SList \: X$.
We do so by noting propositions form a commutative monoid under $\vee$,
which allow us to define membership proof for an element $x$ using the extension operation $\ext{(\blank)}$
by lifting the function $\lambda y. \: x = y$ from $X \rightarrow Prop$ to $List \: X \rightarrow Prop$ and $SList \: X \rightarrow Prop$ respectively.

\begin{mythm}
    A section $s : SList \: X \rightarrow List \: X$ to the canonical map from the free monoid to the free commutative monoid on set $X$
    implies a total order on set $X$ iff
    $\: \forall x \: y \: xs. \: \textit{is-sorted}(x :: xs) \rightarrow y \in x :: xs \rightarrow \textit{is-sorted}([x, y])$.
\end{mythm}
It is well known that a total order on set $X$ would imply a sort function on $List \: X$. It should follow that a sort function on set $X$
would imply a total order on $X$. We can formalize the notion of a sort function as a section to the canonical map from $List$ to
$SList$, which can be thought of as a function which picks a canonical representation from an unordered list, thereby sorting
the list in the process. However, we cannot prove transitivity purely from $s$ being a section, one example being
a function $s : SList \: \mathbb{N} \rightarrow List \: \mathbb{N}$ which sorts ascendingly given an odd-lengthed
$SList$ and descendingly given an even-lengthed $SList$. Hence, a stronger assumption is needed to fully construct
a total order.

\begin{mydef}
    Given a section $s : SList \: X \rightarrow List \: X$ to the canonical map $List \: X \rightarrow SList \: X$,
    we define a relation $\leq$: if $x$ is the head of $s(\{x, y\})$, $x \leq y$.
\end{mydef}

We note that $x \leq y$ iff $\textit{is-sorted}([x, y])$.

\begin{mylemma}
    \label{sort-either}
    Given a section $s : SList \: X \rightarrow List \: X$ to the canonical map $List \: X \rightarrow SList \: X$,
    $s(\{x, y\})$ must either be $[x, y]$ or $[y, x]$.
\end{mylemma}

We note that the canonical map $q : List \: X \rightarrow SList \: X$ preserves length and preserves membership, $x \in xs$ iff $x \in q(xs)$.
Since $q(s(\{x, y\})) = \{x, y\}$ by definition, $s(\{x,y\})$ must therefore have length 2, and $x, y \in s(\{x, y\})$.
Let $q(\{x, y\})$ be $[u, v]$, we perform a proof by cases. For case $x = u$, $y = v$ and $x = v$, $y = u$ the proof is trivial.
For case $x = u$, $y = u$ and $x = v$, $y = v$, we obtain a proof $x = y$. Since $s(\{x, x\}) = [u , v]$ by assumption,
and $q([u, v]) = \{x, x\}$ by definition of $s$, $u, v \in \{x, x\}$, therefore $u, v$ must equal to $x$.
Since $u = v = x = y$, and $s(\{x, y\}) = [u, v]$, $s(\{x, y\}) = [x, y]$.

With this lemma we can then prove $\leq$ does indeed satisfy all axioms of total order.

\begin{mythm}
    $\leq$ is reflexive.
\end{mythm}
By Lemma \ref{sort-either}, $s(\{x, x\})$ must either be $[x, x]$ or $[x ,x]$. Either case we obtain a proof that
$s(\{x, x\}) = [x, x]$. Since $x$ is the head of $[x, x]$, $x \leq x$.

\begin{mythm}
    $\leq$ is antisymmetric.
\end{mythm}
Given $x \leq y$ and $y \leq x$, we want to show $x = y$.
By Lemma \ref{sort-either}, $s(\{x, y\})$ must either be $[x, y]$ or $[y, x]$. In the case $s(\{x, y\}) = [x, y]$,
since $y \leq x$, $y$ is the head of $[x, y]$, and we obtain a proof $y = x$. In the case $s(\{x, y\}) = [y, x]$,
we invert $x$ and $y$ in the previous proof and obtain $x = y$. Either case we obtain $x = y$.

\begin{mythm}
    $\leq$ is total.
\end{mythm}
We want to show for any $x$ and $y$, either $x \leq y$ or $y \leq x$.
By Lemma \ref{sort-either}, $s(\{x, y\})$ must either be $[x, y]$ or $[y, x]$, therefore either $x$ or $y$
is the head of $s(\{x, y\})$. We obtain either $x \leq y$ or $y \leq x$.

\begin{mythm}
    $\leq$ is transitive.
\end{mythm}
Given $x \leq y$ and $y \leq z$, we want to show $x \leq z$. We first note that the head of $s(\{x,y,z\})$
must either be $x$, $y$, or $z$.
For case $x$, by assumption $[x, z]$ is sorted, therefore $x \leq z$.
For case $y$, by assumption $[y, x]$ is sorted, therefore $y \leq x$, and since $x \leq y$ by assumption,
by antisymmetry $x = y$, and by assumption $y \leq z$, therefore $x \leq z$.
For case $z$, by assumption $[z, y]$ is sorted, therefore $z \leq y$, and since $y \leq z$ by assumption,
by antisymmetry $y = z$, and by assumption $x \leq y$, therefore $x \leq z$.

If we assume $X$ to have decidable equality, we can construct a linear order from the total order.
With more constraint it should be possible to prove that the constructed linear order would be a well order, and therefore imply $X$ is a choice
set. We can then show that by assuming every set $X$ is decidable and has a section to the canonical map from the free monoid to the free commutative monoid, we
would be able to derive the axiom of choice.

\begin{myconj}
    SList and Bag are free symmetric monoidal groupoid
\end{myconj}
Previous works already established $SList$ and $Bag$ are free commutative
monoids when 0-truncated, and it is folklore that when 1-truncated they should be free symmetric monoidal groupoid, although
this has not yet been shown internally in HoTT.

In the case of $SList$, higher path constructors are needed. To establish the coherence of $swap$ we need a higher $hexagon$
path constructor. However to define the $hexagon$ constructor it would involve compositions of $swap$ which leads to a regularity
problem. To avoid this we need to split the $hexagon$ equations as below:

\begin{code}
    hexagon- : (a b c : A) (cs : SList A)
    -> a :: b :: c :: cs = c :: b :: a :: cs
    hexagon↑ : (a b c : A) (cs : SList A)
    -> Square (\i -> b :: swap a c cs i) (hexagon- a b c cs)
    (swap b a (c :: cs)) (swap b c (a :: cs))
    hexagon↓ : (a b c : A) (cs : SList A)
    -> Square (hexagon- a b c cs) (swap a c (b :: cs))
    (\i -> a :: swap b c cs i) (\i -> c :: swap b a cs i)
\end{code}


In the case of $Bag$, we first need to show that $Array$ itself is a free monoidal groupoid. We then need to show a free
monoidal groupoid quotiented by a permutation relation would be a free symmetric monoidal groupoid, and show that by quotienting
$Array$ with an isomorphism on the index, we would indeed get a free symmetric monoidal groupoid.

\printbibliography

\end{document}
