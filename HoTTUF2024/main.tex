\documentclass{article}

\usepackage{amsmath,amssymb,amsthm,amsfonts,amscd,mathrsfs,mathtools}
\usepackage{microtype}
\usepackage{biblatex}  
\usepackage{anysize}

\makeatletter
\def\@seccntformat#1{%
  \expandafter\ifx\csname c@#1\endcsname\c@section\else
  \csname the#1\endcsname\quad
  \fi}
\makeatother

\marginsize{2cm}{2cm}{2cm}{2cm}

\addbibresource{cites.bib}

\title{Something about commutativity}
\author{Anonymous}

\date{\today}

\begin{document}
\maketitle

In our work we created a general framework for doing universal algebra and the construction of
free algebra in Cubical Agda. Using the framework we formalized different constructions of free monoids,
such as List, free monoid as a HIT, and index function. We then extend our work by constructing free commutative monoids,
such as swapped-list and free monoid quotiented by a permutation relation, for example list quotiented by permutation
and index function quotiented by isomorphism on index. We directly prove the universal property of the above constructions
within the framework.

Using the same framework, we hope to prove the following conjectures:

\section{Conjecture 1}
\textit{A section to the canonical map from the free monoid to the free commutative monoid on set $X$ implies a linear order on set $X$}.

It is well known that a linear order on set $X$ would imply a sort function on $List$ $X$. It should follow that a sort function on set $X$
would imply a linear order on $X$. We can formalize the notion of a sort function as a section to the canonical map from the free monoid to the
free commutative monoid, which can be thought of as a function which picks a canonical representation from an unordered list, thereby sorting
the list in the process.

With more constraint it should be possible to prove that the constructed linear order would be a well order, and therefore imply $X$ is a choice
set. We can then show that by assuming every set $X$ has a section to the canonical map from the free monoid to the free commutative monoid, we
would be able to derive the axiom of choice, and therefore such assumption would be a constructive taboo.

\section{Conjecture 2}
\textit{Swapped-list and index function quotiented by isomorphism on index are free symmetric monoidal groupoid}.

Previous works already established swapped-list and index function quotiented by isomorphism on index are free commutative
monoids when 0-truncated, and it is folklore that when 1-truncated they should be free symmetric monoidal groupoid, although
this has not yet been shown internally in HoTT. 

In the case of swapped-list, higher path constructors are needed to establish the coherence of swap, whereas in the case of
index function, we first need to show that the index function itself is a free monoidal groupoid. We then need to show a free
monoidal groupoid quotiented by a permutation relation would be a free symmetric monoidal groupoid, and show that by quotienting
the index function with an isomorphism on the index, we would indeed get a free symmetric monoidal groupoid.


\printbibliography

\end{document}
