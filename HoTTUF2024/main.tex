\documentclass{article}

\usepackage{amsmath,amssymb,amsthm,amsfonts,amscd,mathrsfs,mathtools}
\usepackage{quiver} 
\usepackage{microtype}
\usepackage{biblatex}  
\addbibresource{cites.bib}

\title{Something about commutativity}
\author{Anonymous}

\date{\today}

\begin{document}
\maketitle
In this talk, we will discuss free monoids and free commutative monoids, in Homotopy Type Theory, and formalized in
Cubical Agda.

First, we will discuss a general framework for doing universal algebra, and the construction of free algebras\dots

Then, we will discuss the construction of free monoids, and free commutative monoids, giving several constructions of
free monoids, and adding the appropriate notion of commutativity to each representation, to obtain free commutative
monoids. 

From HITs for free monoids to HITs for free commutative monoids.
From Free Monoids to Free commutative monoids by quotienting with permutation relations.
From Array to Bag, and List to PList.
From List to SList.

We prove properties of these constructions.
List(A+B) ~= List(A) + List(B)
SList(A+B) ~= SList(A) x SList(B)

We study specifications for normal forms for these constructions, by showing that cons is injective in each
representation. To do this, we study the path spaces of these types.

Finally, we will relate these ideas to sorting. We will show that sorting can be formalized as a section to the
canonical map from the free monoid to the free commutative monoid, and the correctness of a sorting algorithm can be
stated abstractly in this framework.

Note: It has to be new, previous work: Free Commutative Monoids in Homotopy Type Theory, Final coalgebras of Analytic
functors in Homotopy Type Theory, and A Fresh look at Commutativity.

\printbibliography

\end{document}
