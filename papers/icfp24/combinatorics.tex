\section{Combinatorics}
\label{sec:combinatorics}

The purpose of all the hard work behind establishing the universal properties of our types of (ordered) and unordered
lists is to be able to define operations on them systematically, which are mathematically sound, and to be able to
reason about them -- which is a common theme in the Algebra of Programming~\cite{birdAlgebraProgramming1997} community.
%
Using our tools, we explore definitions of various combinatorial properties and operations for both
free monoids and free commutative monoids.
%
By univalence (and the structure identity principle), everything henceforth holds for any presentation of free monoids
and free commutative monoids, therefore we avoid picking a specific construction.
%
We use $\FF(A)$ to denote the free monoid or free commutative monoid on $A$, $\LL(A)$ to exclusively denote the free
monoid construction, and $\MM(A)$ to exclusively denote the free commutative monoid construction.

With universal properties we can structure and reason about our programs using algebraic ideas.
%
For example $\term{length}$ is a common operation defined inductively for $\List$,
but usually in proof engineering, properties about $\term{length}$, e.g.
$\term{length}(xs \doubleplus ys) = \term{length}(xs) + \term{length}(ys)$,
are proven externally.
%
Within our framework, where \emph{fold} (or the $\ext{(\blank)}$ operation) is a correct-by-construction homomorphism,
we can define operations like $\term{length}$ simply by extension,
which also gives us a proof that they are homomorphisms for free!
%
A further application of the universal property is to prove two different types are equal, by showing they both satisfy
the same universal property (see~\cref{lem:free-algebras-unique}), which is desirable especially when proving a direct
equivalence between the two types turns out to be a difficult exercise in combinatorics.

To illustrate this, we give examples of some common operations that are defined in terms of the universal property.

\subsection{Length}

Any presentation of free monoids or free commutative monoids has a $\term{length} : \FF(A) \to \Nat$ function.
%
$\Nat$ is not just a monoid with $(0,+)$, the $+$ operation is also commutative!
%
We can extend the constant function $\lambda x.\, 1\,:\,A \to \Nat$
to obtain $\ext{(\lambda x.\, 1)} : \FF(A) \to \Nat$, which is the length homomorphism.
%
This allows us to define $\term{length}$ for any construction of free (commutative) monoids,
and also gives us a proof of $\term{length}$ being a homomorphism for free.

\begin{figure}[H]
    \centering
    \begin{tikzcd}[ampersand replacement=\&,cramped]
        {\FF(A)} \&\& {(\mathbb{N}, 0, +)} \\
        \\
        A
        \arrow["\eta_A", from=3-1, to=1-1]
        \arrow["{\ext{(\lambda x. \, 1)}}", from=1-1, to=1-3]
        \arrow["{\lambda x. \, 1}"', from=3-1, to=1-3]
    \end{tikzcd}
    \caption{Definition of $\term{length}$ by universal property}
    \label{fig:enter-label}
\end{figure}

\subsection{Membership}\label{comb:member}

Going further, any presentation of free monoids or free commutative monoids has a membership predicate:
$\_\in\_ : A \to \FF(A) \to \hProp$, for any set $A$.
%
By fixing an element $x: A$, the element we want to define the membership proof for,
we define $\yo_A(x) = \lambda y.\, x \id y : A \to \hProp$.
%
This is formally the Yoneda map under the ``types are groupoids'' correspondence,
where $x:A$ is being sent to the Hom-groupoid (formed by the identity type), of type $\hProp$.
%
Now, the main observation is that $\hProp$ forms a (commutative) monoid under logical or: $\vee$ and false: $\bot$.
%
Note that the proofs of monoid laws use equality, which requires the use of univalence (or at least, propositional
extensionality).
%
Using this (commutative) monoid structure, we extend $\yo_A(x)$ to obtain $x \in \blank : \FF(A) \to \hProp$, which
gives us the membership predicate for $x$, and its homomorphic properties (the colluquial~$\term{here}/\term{there}$
constructors for the de Bruijn encoding).

\begin{figure}[H]
    \centering
    % https://q.uiver.app/#q=WzAsNCxbMCwwLCJcXEZGKEEpIl0sWzIsMF0sWzAsMiwiQSJdLFszLDAsIihcXGhQcm9wLCBcXGJvdCwgXFx2ZWUpIl0sWzIsMCwiXFxldGFfQSJdLFswLDMsIih4IFxcaW4gXFxibGFuaykgLyBcXGV4dHtcXHlvX0EoeCl9Il0sWzIsMywiXFx5b19BKHgpIiwyXV0=
    \begin{tikzcd}[ampersand replacement=\&,cramped]
        {\FF(A)} \&\& {} \& {(\hProp, \bot, \vee)} \\
        \\
        A
        \arrow["{\eta_A}", from=3-1, to=1-1]
        \arrow["{(x \in \blank) / \ext{\yo_A(x)}}", from=1-1, to=1-4]
        \arrow["{\yo_A(x)}"', from=3-1, to=1-4]
    \end{tikzcd}
    \caption{Definition of membership proof for $x$ by universal property}
    \label{fig:enter-label}
\end{figure}

\subsection{Any and All predicates}

More generally, any presentation of free (commutative) monoids $\FF(A)$ also supports the
$\term{Any}$ and $\term{All}$ predicates, which allow us to lift a predicate $A \to \hProp$ (on $A$),
to \emph{any} or \emph{all} elements of $xs : \FF(A)$, respectively.
%
We note that $\hProp$ forms a (commutative) monoid in two different ways: $(\bot,\vee)$ and $(\top,\wedge)$
(disjunction and conjunction), which are the two different ways to get $\term{Any}$ and $\term{All}$, respectively.
That is,
\begin{align*}
    \type{Any}(P) & = \ext{P} : \FF(A) \to (\hProp, \bot, \vee)   \\
    \type{All}(P) & = \ext{P} : \FF(A) \to (\hProp, \top, \wedge)
\end{align*}
\begin{figure}[H]
    \centering
    \begin{minipage}[t]{0.49\textwidth}
        \centering
        \begin{tikzcd}[ampersand replacement=\&,cramped]
            {\FF(A)} \&\& {(\hProp, \top, \wedge)} \\
            \\
            A
            \arrow["\eta_A", from=3-1, to=1-1]
            \arrow["{\ext{P}}", from=1-1, to=1-3]
            \arrow["{P}"', from=3-1, to=1-3]
        \end{tikzcd}
        \caption{Definition of $\term{All}$ by universal property}
        \label{fig:enter-label}
    \end{minipage}
    \begin{minipage}[t]{0.49\textwidth}
        \centering
        \begin{tikzcd}[ampersand replacement=\&,cramped]
            {\FF(A)} \&\& {(\hProp, \bot, \vee)} \\
            \\
            A
            \arrow["\eta_A", from=3-1, to=1-1]
            \arrow["{\ext{P}}", from=1-1, to=1-3]
            \arrow["{P}"', from=3-1, to=1-3]
        \end{tikzcd}
        \caption{Definition of $\term{Any}$ by universal property}
        \label{fig:enter-label}
    \end{minipage}
\end{figure}
These constructions are used in the proofs of our main results in~\cref{sec:sorting}.

\subsection{Other combinatorial properties}

Free commutative monoid and free monoids also share common combinatorial properties, for example both satisfy the
conical monoid property.
%
However, more often than not, they both do not exhibit the same properties, for example:
\begin{itemize}
    \item The coproduct of two (free) commutative monoids is given by their underlying cartesian product
          (dual of Fox's theorem~\cite{foxCoalgebrasCartesianCategories1976}),
          that is, $\MM(A + B) \eqv \MM(A) \times \MM(B)$.
          %
          This is important in linear logic, and used in the Seely isomorphism.
          %
          However, the coproduct of two (free) monoids is given by their free product,
          which does not have such a characterisation.
    \item Free commutative monoids satisfy the Riesz refinement property,
          which is important in the semantics of differential linear logic,
          however free monoids do not.
\end{itemize}
We mention these facts for completeness -- these show that the addition of commutativity leads to \emph{unordering},
which allows shuffling around the elements of free monoids (lists), leading to interesting combinatorial structure.
%
These are explored in more detail and proved in~\cite{choudhuryFreeCommutativeMonoids2023}.

\subsection{Heads and Tails}\label{sec:head}

More generally, commutativity is a way of enforcing unordering, or forgetting ordering.
%
The universal property of free commutative monoids only allows eliminating to another commutative monoid --
can we define functions from $\MM(A)$ to a non-commutative structure?
%
In baby steps, we will consider the very popular $\term{head}$ function on lists.
% from $\MM(A)$ to a structure which is not commutative, we must impose some kind of ordering in order to define such a
% function.

The $\term{head} : \FF(A) \to A$ function takes the \emph{first} element out of $\FF(A)$, and the rest of the list is
its tail.
%
We analyse this by cases on the length of $\FF(A)$
(which is definable for both free monoids and free commutative monoids).

\begin{itemize}
    \item
          Case 0: $\term{head}$ cannot be defined, for example if $A = \emptyt$, then $\FF[\emptyt] \eqv \unitt$,
          so a $\term{head}$ function would produce an element of $\emptyt$, which is impossible.

    \item
          Case 1: For a singleton, $\term{head}$ can be defined for both $\MM(A)$ and $\LL(A)$.
          %
          In fact, it is a equivalence between $xs : \FF(A)$ where $\term{length}(xs) = 1$, and the type $A$,
          where the inverse is given by $\eta_A$.

    \item
          Case $n \geq 2$: $\term{head}$ can be defined for any $\LL(A)$, of course.
          %
          Using $\List$ as an example, one can simply take the first element of the list:
          $\term{head}(x :: y :: xs) = x$.
          %
          However, for $\MM(A)$, things go wrong!
          %
          Using $\SList$ as an example, by $\term{swap}$, $\term{head}$ must satisfy:
          $\term{head}(x :: y :: xs) = \term{head}(y :: x :: xs)$ for any $x, y : A$ and $xs : \SList(A)$.
          %
          Which one do we pick -- $x$ or $y$?
          %
          Informally, this means, we somehow need an ordering on $A$, or a sorted list $xs$ of $A$,
          so we can pick a ``least'' (or ``biggest''?) element $x \in xs$ such that for any permutation $ys$ of $xs$,
          also $\term{head}(ys) = x$.
\end{itemize}

This final observation leads to our main result in~\cref{sec:sorting}.
