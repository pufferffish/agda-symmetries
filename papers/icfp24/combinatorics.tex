\section{Combinatorics}
\label{sec:combinatorics}

We now explore various combinatorial properties and operations we can define for both
free monoids and free commutative monoids. By univalence and structure identity principle,
everything henceforth holds for any presentation of free monoids and free commutative monoids, 
therefore we avoid naming any specific constructions.
We use $\FF(A)$ to denote the free monoid or free commutative monoid on $A$,
$\LL(A)$ to exclusively denote free monoid and $\MM(A)$ to exclusively denote free commutative monoid.

With universal properties we can structure our programs using ideas from algebra.
For example $\term{length}$ is a common operation defined inductively for $\List$,
but to state properties about $\term{length}$, e.g.
$\term{length}(xs \doubleplus ys) = \term{length}(xs) + \term{length}(ys)$,
we would need to prove them externally. With our framework we can define operations like $\term{length}$
using the $\ext{(\blank)}$ operation, which would also give us a proof that they are homomorphisms for free.
We can also use universal property to prove two different types are equal by showing they both
satisfy the same universal property, which is desirable especially if proving a direct equivalence between
the two types is very difficult.

To illustrate this, we give some examples of how some common operations can be defined
in terms of universal property.

\subsection{Length}

Any presentation of free monoids or free commutative monoids has a $\term{length} : \FF(A) \to \Nat$ function.
$\Nat$ is a (commutative) monoid with $(0,+)$, so we can extend $\lambda x.\, 1\,:\,A \to \Nat$.
This allows us to define $\term{length}$ for any construction of free (commutative) monoid, and also
gives us a proof of $\term{length}$ being a homomorphism for free.

\begin{figure}[H]
    \centering
    \begin{tikzcd}[ampersand replacement=\&,cramped]
    	{\FF(A)} \&\& {(\mathbb{N}, 0, +)} \\
    	\\
    	A
    	\arrow["\eta_A", from=3-1, to=1-1]
    	\arrow["{\ext{(\lambda x. \, 1)}}", from=1-1, to=1-3]
    	\arrow["{\lambda x. \, 1}"', from=3-1, to=1-3]
    \end{tikzcd}
    \caption{Definition of $\term{length}$ by universal property}
    \label{fig:enter-label}
\end{figure}


\subsection{Membership}
Any presentation of free monoids or free commutative monoids has a membership predicate:
$\_\in\_ : A \to \FF(A) \to \hProp$. By fixing an element $x: A$, the element we want to define
the membership proof for, and 
assuming $A$ is a set, we can define $\yo_A(x) = \lambda y.\, x \id y : A \to \hProp$.
Since $\hProp$ forms a (commutative) monoid under $\vee$,
we can extend $\yo_A(x)$ to obtain $x \in \blank : \FF(A) \to \hProp$, giving us the membership predicate for $x$.

\begin{figure}[H]
    \centering
    % https://q.uiver.app/#q=WzAsNCxbMCwwLCJcXEZGKEEpIl0sWzIsMF0sWzAsMiwiQSJdLFszLDAsIihcXGhQcm9wLCBcXGJvdCwgXFx2ZWUpIl0sWzIsMCwiXFxldGFfQSJdLFswLDMsIih4IFxcaW4gXFxibGFuaykgLyBcXGV4dHtcXHlvX0EoeCl9Il0sWzIsMywiXFx5b19BKHgpIiwyXV0=
    \begin{tikzcd}[ampersand replacement=\&,cramped]
    	{\FF(A)} \&\& {} \& {(\hProp, \bot, \vee)} \\
    	\\
    	A
    	\arrow["{\eta_A}", from=3-1, to=1-1]
    	\arrow["{(x \in \blank) / \ext{\yo_A(x)}}", from=1-1, to=1-4]
    	\arrow["{\yo_A(x)}"', from=3-1, to=1-4]
    \end{tikzcd}
    \caption{Definition of membership proof for $x$ by universal property}
    \label{fig:enter-label}
\end{figure}


\subsection{Any and All predicate}

Any presentation of free monoids or free commutative monoids $\FF(A)$ has the
$\term{Any}$ and $\term{All}$ predicates, which allow us to lift a predicate $A \to \hProp$
to any or all elements of $xs : \FF(A)$. We note that
$\hProp$ forms a (commutative) monoid in two different ways: $(\bot,\vee)$ and $(\top,\wedge)$,
therefore given a predicate $P : A \to \hProp$, we get:
\begin{align*}
    Any(P) &= \ext{P} : \FF(A) \to (\hProp, \bot, \vee) \\
    All(P) &= \ext{P} : \FF(A) \to (\hProp, \top, \wedge)
\end{align*}

\begin{figure}[H]
    \centering
    \begin{minipage}[t]{0.49\textwidth}
        \centering
        \begin{tikzcd}[ampersand replacement=\&,cramped]
        	{\FF(A)} \&\& {(\hProp, \top, \wedge)} \\
        	\\
        	A
        	\arrow["\eta_A", from=3-1, to=1-1]
        	\arrow["{\ext{P}}", from=1-1, to=1-3]
        	\arrow["{P}"', from=3-1, to=1-3]
        \end{tikzcd}
        \caption{Definition of $\term{All}$ by universal property}
        \label{fig:enter-label}
    \end{minipage}
    \begin{minipage}[t]{0.49\textwidth}
        \centering
        \begin{tikzcd}[ampersand replacement=\&,cramped]
        	{\FF(A)} \&\& {(\hProp, \bot, \vee)} \\
        	\\
        	A
        	\arrow["\eta_A", from=3-1, to=1-1]
        	\arrow["{\ext{P}}", from=1-1, to=1-3]
        	\arrow["{P}"', from=3-1, to=1-3]
        \end{tikzcd}
        \caption{Definition of $\term{Any}$ by universal property}
        \label{fig:enter-label}
    \end{minipage}
\end{figure}

\subsection{Conical-monoid relation}
Free commutative monoids and free monoids also share some common properties, one for example being
they are conical~\cite{wehrungTensorProductsStructures1996},
in other words, $\forall as,\, bs : \FF(A),\,as \otimes bs = e \to as = bs = e$.

\subsection{Seely isomorphism}
These common properties break down however when we study combinatorics exclusive to free commutative monoids.
One such property is that the products of free commutative monoids
is equal to the free commutative monoids of coproducts by univalence.
\begin{equation*}
    \MM(A) \times \MM(B) = \MM(A + B)\quad,\quad\unitt = \MM(\emptyt)
\end{equation*}

\cite{choudhuryFreeCommutativeMonoids2023} have given a proof on this property. We give a diagram here
showing how this equivlance can be constructed.

\begin{center}
% https://q.uiver.app/#q=WzAsMyxbMCwwLCJcXE1NKEEpXFx0aW1lc1xcTU0oQikiXSxbMSwxLCJcXE1NKEErQikgXFx0aW1lcyBcXE1NKEErQikiXSxbMiwwLCJcXE1NKEEgKyBCKSJdLFswLDEsIlxcTU0oaV8xKSBcXHRpbWVzIFxcTU0oaV8yKSIsMix7ImxhYmVsX3Bvc2l0aW9uIjowfV0sWzEsMiwiXFxvdGltZXNfeyhBK0IpfSIsMix7ImxhYmVsX3Bvc2l0aW9uIjoxMDB9XSxbMCwyLCJcXHNpbWVxIl1d
\begin{tikzcd}[ampersand replacement=\&,cramped]
	{\MM(A)\times\MM(B)} \&\& {\MM(A + B)} \\
	\& {\MM(A+B) \times \MM(A+B)}
	\arrow["{\MM(i_1) \times \MM(i_2)}"'{pos=0}, from=1-1, to=2-2]
	\arrow["{\otimes_{(A+B)}}"'{pos=1}, from=2-2, to=1-3]
	\arrow["\simeq", from=1-1, to=1-3]
\end{tikzcd}
\end{center}

This property does not hold for free monoids, because $\LL(A + B)$ is larger than $\LL(A) \times \LL(B)$.
Given $a: A$ and $b: B$, we can only construct $([a], [b]) : \LL(A) \times \LL(B)$, however we can
construct both $[\term{inl}(a), \term{inr}(b)]$ and $[\term{inr}(b), \term{inl}(a)]$ for $\LL(A + B)$.
By forgetting the order with commutativity,
$[\term{inl}(a), \term{inr}(b)]$ and $[\term{inr}(b), \term{inl}(a)]$ would count as the same elements
of $\MM(A + B)$, allowing us to construct an isomorphism from $\MM(A + B)$ to $\MM(A) \times \MM(B)$.

\subsection{Riesz refinement property}
Another property exclusive to free commutative monoids is the Riesz refinement
property~\cite{dobbertinRefinementMonoidsVaught1983}.
\begin{equation*}
\begin{gathered}
    as \otimes bs = cs \otimes ds \\
    \iff \\
    \exists (xs_{1}, xs_{2}, ys_{1}, ys_{2} : \MM(A)).
    (as = xs_{1} \otimes xs_{2}) \land (bs = ys_{1} \otimes ys_{2}) \land
    (xs_{1} \otimes ys_{1} = cs) \land (xs_{2} \otimes ys_{2} = ds).
\end{gathered}
\end{equation*}

Concretely, this means we can break down $as$ into $xs_1 \otimes xs_2$,
$bs$ into $ys_2 \otimes ys_2$, and show that $as \otimes bs = (xs_1 \otimes ys_1) \otimes (xs_2 \otimes ys_2)$.
The swapping of $xs_{1,2}$ and $ys_{1,2}$ is inherently commutative in nature,
therefore $\LL(A)$ does not satisfy this property.

\subsection{Path space}
\redtext{Say something about Path space. Cons is injective.}

\subsection{Head}

