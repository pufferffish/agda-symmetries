\section{Combinatorics}
\label{sec:combinatorics}

With universal properties we can structure our programs using ideas from algebra.
For example $\term{length}$ is a common operation defined inductively for $\List$,
but to state properties about $\term{length}$, e.g.
$\term{length}(xs \doubleplus ys) = \term{length}(xs) + \term{length}(ys)$,
we would need to prove them externally. With our framework we can define operations like $\term{length}$
using the $\ext{(\blank)}$ operation, which would also give us a proof that they are homomorphisms for free.
We can also use universal property to prove two different types are equal by showing they both
satisfy the same universal property, which is desirable especially if proving a direct equivalence between
the two types is very difficult.

To illustrate this, we give some examples of how some common operations can be defined
in terms of universal property.
We use $\FF(A)$ to denote the free monoid or free commutative monoid on $A$.

\subsection{Length}

Any presentation of free monoids or free commutative monoids has a $\term{length} : \FF(A) \to \Nat$ function.
$\Nat$ is a (commutative) monoid with $(0,+)$, so we can extend $\lambda x.\, 1\,:\,A \to \Nat$.
This allows us to define $\term{length}$ for any construction of free (commutative) monoid, and also
gives us a proof of $\term{length}$ being a homomorphism for free.

\begin{figure}[H]
    \centering
    \begin{tikzcd}[ampersand replacement=\&,cramped]
    	{\FF(A)} \&\& {(\mathbb{N}, 0, +)} \\
    	\\
    	A
    	\arrow["\eta_A", from=3-1, to=1-1]
    	\arrow["{\ext{(\lambda x. \, 1)}}", from=1-1, to=1-3]
    	\arrow["{\lambda x. \, 1}"', from=3-1, to=1-3]
    \end{tikzcd}
    \caption{Definition of $\term{length}$ by universal property}
    \label{fig:enter-label}
\end{figure}


\subsection{Membership}
Any presentation of free monoids or free commutative monoids has a membership predicate:
$\_\in\_ : A \to \FF(A) \to \hProp$. By fixing an element $x: A$, the element we want to define
the membership proof for, and 
assuming $A$ is a set, we can define $\yo_A(x) = \lambda y.\, x \id y : A \to \hProp$.
Since $\hProp$ forms a (commutative) monoid under $\vee$,
we can extend $\yo_A(x)$ to obtain $x \in \blank : \FF(A) \to \hProp$, giving us the membership predicate for $x$.

\begin{figure}[H]
    \centering
    % https://q.uiver.app/#q=WzAsNCxbMCwwLCJcXEZGKEEpIl0sWzIsMF0sWzAsMiwiQSJdLFszLDAsIihcXGhQcm9wLCBcXGJvdCwgXFx2ZWUpIl0sWzIsMCwiXFxldGFfQSJdLFswLDMsIih4IFxcaW4gXFxibGFuaykgLyBcXGV4dHtcXHlvX0EoeCl9Il0sWzIsMywiXFx5b19BKHgpIiwyXV0=
    \begin{tikzcd}[ampersand replacement=\&,cramped]
    	{\FF(A)} \&\& {} \& {(\hProp, \bot, \vee)} \\
    	\\
    	A
    	\arrow["{\eta_A}", from=3-1, to=1-1]
    	\arrow["{(x \in \blank) / \ext{\yo_A(x)}}", from=1-1, to=1-4]
    	\arrow["{\yo_A(x)}"', from=3-1, to=1-4]
    \end{tikzcd}
    \caption{Definition of membership proof for $x$ by universal property}
    \label{fig:enter-label}
\end{figure}


\subsection{Any and All predicate}

Any presentation of free monoids or free commutative monoids $\FF(A)$ has the
$\term{Any}$ and $\term{All}$ predicates, which allow us to lift a predicate $A \to \hProp$
to any or all elements of $xs : \FF(A)$. We note that
$\hProp$ forms a (commutative) monoid in two different ways: $(\bot,\vee)$ and $(\top,\wedge)$,
therefore given a predicate $P : A \to \hProp$, we get:
\begin{align*}
    Any(P) &= \ext{P} : \FF(A) \to (\hProp, \bot, \vee) \\
    All(P) &= \ext{P} : \FF(A) \to (\hProp, \top, \wedge)
\end{align*}

\begin{figure}[H]
    \centering
    \begin{minipage}[t]{0.49\textwidth}
        \centering
        \begin{tikzcd}[ampersand replacement=\&,cramped]
        	{\FF(A)} \&\& {(\hProp, \top, \wedge)} \\
        	\\
        	A
        	\arrow["\eta_A", from=3-1, to=1-1]
        	\arrow["{\ext{P}}", from=1-1, to=1-3]
        	\arrow["{P}"', from=3-1, to=1-3]
        \end{tikzcd}
        \caption{Definition of $\term{All}$ by universal property}
        \label{fig:enter-label}
    \end{minipage}
    \begin{minipage}[t]{0.49\textwidth}
        \centering
        \begin{tikzcd}[ampersand replacement=\&,cramped]
        	{\FF(A)} \&\& {(\hProp, \bot, \vee)} \\
        	\\
        	A
        	\arrow["\eta_A", from=3-1, to=1-1]
        	\arrow["{\ext{P}}", from=1-1, to=1-3]
        	\arrow["{P}"', from=3-1, to=1-3]
        \end{tikzcd}
        \caption{Definition of $\term{Any}$ by universal property}
        \label{fig:enter-label}
    \end{minipage}
\end{figure}

% We denote free monoid on $A$ with $\LL(A)$ and free commutative monoid on $A$ with $\MM(A)$.
% $\MM(A + B) \eqv \MM(A) \times \MM(B)$.


