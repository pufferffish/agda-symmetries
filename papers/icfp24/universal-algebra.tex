\section{Universal Algebra}
\label{sec:universal-algebra}

We first develop some basic notions from universal algebra and equational
logic~\cite{birkhoffStructureAbstractAlgebras1935}.
%
Universal algebra is the abstract study of algebraic structures, which have (algebraic) operations and (universal)
equations.
%
This gives us some vocabulary and a framework to express our results in.
%
The point of view we take is the standard category-theoretic approach to universal algebra, which predates the Lawvere
theory or abstract clone point of view.
%
We keep a running example of monoids in mind, while explaining and defining the abstract concepts.
% In the language of universal algebra, such a structure is formalized by giving a signature of operations, and a
% structure being a carrier set with functions that interpret these operations. We describe such a framework for universal
% algebra in HoTT, as follows.

\begin{definition}[Signature]
    A signature, denoted $\sigma$, is a (dependent) pair consisting of:
    \begin{itemize}
        \item a set of operations, $\term{op} : \Set$,
        \item an arity function for each symbol, $\term{ar} : \term{op} \to \Set$.
    \end{itemize}
\end{definition}

\begin{example}
    A monoid is a set with an identity element (or a nullary operation), and a binary multiplication operation.
    %
    The signature for monoids $\sigma_{\Mon}$ is encoded as:
    $(\Fin[2],\lambda \{0 \mapsto \Fin[0] ; 1 \mapsto \Fin[2] \})$.
    %
    Informally, the set of operations is the two-element set $\{e,\mult\}$, which is written as $\Fin[2]$,
    and the arity function picks out a (finite) set denoting the (finite) arity of each operation.
    %
    Of course, this is an example of a finitary signature,
    but in general the arity function can be any (not necessarily finite) set.
\end{example}

Every signature $\sigma$ induces a signature functor on $\Set$, written $F_{\sigma}$.

\begin{definition}[Signature functor]
    \begin{align*}
        F_{\sigma} \colon \Set & \to \Set                                     \\
        X                      & \mapsto \dsum{o:\term{op}}{X^{\term{ar}(o)}} \\
        X \xto{f} Y            & \mapsto
        \dsum{o:\term{op}}{X^{\term{ar}(o)}}
        \xto{(o, \blank \comp f)}
        \dsum{o:\term{op}}{Y^{\term{ar}(o)}}
    \end{align*}
\end{definition}

\begin{example}
    The signature functor for monoids, $F_{\sigma_{\Mon}}$, assigns to a carrier set $X$,
    the sets of inputs for each operation.
    %
    Expanding the dependent product on $\Fin[2]$, we obtain a coproduct or disjoint union of sets:
    $F_{\sigma_{\Mon}}(X) \eqv X^{\Fin[0]} + X^{\Fin[2]}$.
\end{example}

A $\sigma$-structure is given by a carrier set, with functions interpreting each operation symbol.
%
The signature functor applied to a carrier set gives the inputs to each operation, and the output is simply a map back
to the carrier set.
%
Formally, these two pieces of data are an algebra for the $F_{\sigma}$ functor.
%
We write $\str{X}$ for a $\sigma$-structure with carrier set $X$, following the model-theoretic notational convention.

\begin{definition}[Structure]
    A $\sigma$-structure $\str{X}$ is an $F_{\sigma}$ algebra, that is:
    \begin{itemize}
        \item a carrier set $X$, and
        \item an algebra map $\alpha_{X}\colon F_{\sigma}(X) \to X$.
    \end{itemize}
\end{definition}

\begin{example}
    Concretely, an $F_{\sigma_{\Mon}}$-algebra has the type
    \[
        \begin{array}{rcl}
            \alpha_{X} & :    & \dsum{o:\Fin[2]}{X^{\term{ar}(o)}} \to X       \\
                       & \eqv & (X^{\Fin[0] + \Fin[2]}) \to X                  \\
                       & \eqv & (X^{\Fin[0]} \to X) \times (X^{\Fin[2]} \to X) \\
                       & \eqv & (\unitt \to X) \times (X \times X \to X)
        \end{array}
    \]
    The natural numbers with the carrier set $\Nat$,
    and the constant $0$ and the addition operation $+$ give an example of such a structure.
    %
    Similarly, with the constant $1$ and the multiplication operation $\times$, we have another monoid structure.
\end{example}

\begin{definition}[Homomorphism]
    A homomorphism is
\end{definition}

For any object \( \mathfrak{Y} \) in $\sigma$-Alg, $(\blank) \circ \eta_X$ is an equivalence:

\begin{figure}[H]
    \centering
    % https://q.uiver.app/#q=WzAsMyxbMCwwLCJYIl0sWzIsMCwiVShBKSJdLFsyLDIsIlUoQikiXSxbMCwxLCJpIiwwLHsiY29sb3VyIjpbMSwxMDAsNjBdfSxbMSwxMDAsNjAsMV1dLFswLDIsImciLDJdLFsxLDIsIlUoZikiLDAseyJzdHlsZSI6eyJib2R5Ijp7Im5hbWUiOiJkb3R0ZWQifX19XV0=
    \[\begin{tikzcd}[ampersand replacement=\&]
            \mathfrak{F}(X) \\
            \\
            \mathfrak{Y}
            \arrow["f", dotted, from=1-1, to=3-1]
        \end{tikzcd}
        \mapsto
        \begin{tikzcd}[ampersand replacement=\&]
            X \&\& {F(X)} \\
            \\
            \&\& {Y}
            \arrow["\eta_X", color={rgb,255:red,255;green,54;blue,51}, from=1-1, to=1-3]
            \arrow["f \circ \eta_X"', from=1-1, to=3-3]
            \arrow["{f}", dotted, from=1-3, to=3-3]
        \end{tikzcd}\]
    \caption{Universal property of free algebras}
    \label{fig:universal-property}
\end{figure}



In this paper, we are interested in the constructions of free monoids and free commutative monoids.