\section{Construction of Free Commutative Monoids}
\label{sec:commutative-monoids}

We can use HITs where adjacent swaps generate all symmetries,
for example swap-list taken from \cite{Choudhury_2023}.

\begin{lstlisting}[language=Haskell]
data SList (A : Type) : Type where
  [] : SList A
  _::_ : A -> SList A -> SList A
  swap : $\forall$ x y xs -> x :: y :: xs $\equiv$ y :: x :: xs 
  trunc : $\forall$ x y -> (p q : x $\equiv$ y) -> p $\equiv$ q
\end{lstlisting}

\noindent
Alternatively, we can also quotient Array with symmetries to get commutativity.
This construction is taken from \cite{Choudhury_2023}. \cite{joram_et_al:LIPIcs.ITP.2023.20}
also gave a similar construction, where only the index function is quotiented, instead of
the entire array.

\begin{lstlisting}
_$\approx$_ : Array A $\to$ Array A $\to$ Type
(n , f) $\approx$ (m , g) = $\Sigma$[ $\sigma$ $\in$ Iso (Fin n) (Fin m) ] v $\equiv$ w $\circ$ $\sigma$ .fun

Bag : Type -> Type
Bag A = Array A / _$\approx$_
\end{lstlisting}

\noindent
We can also do the same to List, quotienting it with permutation to get commutativity.
This construction is taken from \cite{joram_et_al:LIPIcs.ITP.2023.20}.

\begin{lstlisting}
PList : Type -> Type
PList A = Array A / Perm
\end{lstlisting}

More generally, if $F : \mathcal{U} \to \mathcal{U}$ satisfies the universal property of free monoid,
then $F(X)$ quotiented by a permutation relation would be a free commutative monoid on $X$.

A relation $\approx$ is a correct permutation relation iff it:
\begin{itemize}
    \item is reflexive, symmetric, transitive (equivalence),
    \item is a congruence wrt $\otimes$: $a \otimes b \to c \otimes d \to a \otimes c \approx b \otimes d$,
    \item is commutative: $a \otimes b \approx b \otimes a$, and
    \item respects $\ext{(\blank)}$: $\forall f. \, a \approx b \to \ext{f}(a) = \ext{f}(b)$.
\end{itemize}

\subsection{Bag}
How to show $\approx$ respects commutativity: $a \otimes b \approx b \otimes a$?

Let $a = (n , f)$ and $b = (m , g)$, we need to compute an isomorphism $\phi\colon \Fin[n+m]\xto{\sim}\Fin[m+n]$,
such that: $(f \oplus g) = (g \oplus f) \circ \phi$.

We can use combinators from formal operations in symmetric rig groupoids \cite{10.1145/3498667} to define $\phi$:
\begin{align*}
    \phi := \Fin[n+m] \xto{\sim} \Fin[n] + \Fin[m] \xto{\term{swap}_{+}} \Fin[m] + \Fin[n] \xto{\sim} \Fin[m+n] \\
\end{align*}

\begin{figure}[H]
    \centering
    \vspace{-1em}
      % https://q.uiver.app/#q=WzAsMixbMCwwLCJcXHswLDEsXFxkb3RzLG4tMSwgbixuKzEsXFxkb3RzLG4rbS0xXFx9Il0sWzAsMSwiIFxce24sbisxXFxkb3RzLG4rbS0xLDAsMSxcXGRvdHMsbi0xXFx9Il0sWzAsMSwiIiwwLHsic3R5bGUiOnsidGFpbCI6eyJuYW1lIjoibWFwcyB0byJ9fX1dXQ==
\begin{tikzcd}[ampersand replacement=\&,cramped]
	{\{\color{red}0,1,\dots,n-1, \color{blue} n,n+1,\dots,n+m-1 \color{black}\}} \\
	{ \{\color{blue}n,n+1\dots,n+m-1, \color{red}0,1,\dots,n-1 \color{black}\}}
    \arrow["\phi", maps to, from=1-1, to=2-1]
\end{tikzcd}
    \caption{Operation of $\phi$}
    \label{fig:enter-label}
\end{figure}

To show $\approx$ respects $\ext{(\blank)}$: $\forall f. \, a \approx b \to \ext{f}(a) = \ext{f}(b)$,
we need to prove $\ext{f}$ is invariant under permutation: for all $\phi\colon \Fin[n]\xto{\sim}\Fin[n]$,
$\ext{f}(n, i) \id \ext{f}(n, i \circ \phi)$.

\begin{lemma}
Given $\phi\colon \Fin[S(n)]\xto{\sim}\Fin[S(n)]$, there is a $\tau\colon \Fin[S(n)]\xto{\sim}\Fin[S(n)]$
such that $\tau(0) = 0$, and $\ext{f}(S(n), i \circ \phi) = \ext{f}(S(n), i \circ \tau)$.
\end{lemma}

Let $k$ be $\phi^{-1}(0)$, and $k + j = S(n)$:
\begin{align*}
    \tau := \Fin[S(n)] \xto{\phi} \Fin[S(n)] \xto{\sim} \Fin[k+j] \xto{\sim} \Fin[k] + \Fin[j]
    \xto{\term{swap}_{+}} \Fin[j] + \Fin[k] \xto{\sim} \Fin[j+k] \xto{\sim} \Fin[S(n)]
\end{align*}

\begin{figure}[H]
    \centering
    \begin{tikzcd}[ampersand replacement=\&,cramped]
    	{\{\color{blue}0, 1, 2, \dots, \color{red}k, k+1, k+2, \dots \color{black}\}} \\
    	{ \{\color{blue}x, y, z, \dots, \color{red}0, u, v, \dots \color{black}\}}
        \arrow["\phi", maps to, from=1-1, to=2-1]
    \end{tikzcd}
    \hspace{1em}
    \begin{tikzcd}[ampersand replacement=\&,cramped]
    	{\{\color{blue}0, 1, 2, \dots, \color{red}k, k+1, k+2, \dots \color{black}\}} \\
    	{ \{\color{red}0, u, v, \dots, \color{blue}x, y, z, \dots \color{black}\}}
        \arrow["\tau", maps to, from=1-1, to=2-1]
    \end{tikzcd}
    \caption{Operation of $\tau$ constructed from $\phi$}
    \label{fig:enter-label}
\end{figure}

\begin{lemma}
Given $\tau\colon \Fin[S(n)]\xto{\sim}\Fin[S(n)]$ where $\tau(0) = 0$,
there is a $\psi : \Fin[n] \xto{\sim} \Fin[n]$ such that $\tau \circ S = S \circ \psi$.
\end{lemma}

\begin{figure}[H]
    \centering
    \begin{tikzcd}[ampersand replacement=\&,cramped]
    	{\{\color{blue}0, \color{red}1, 2, 3, \dots \color{black}\}} \\
    	{ \{\color{blue}0, \color{red} x, y, z \dots \color{black}\}}
        \arrow["\tau", maps to, from=1-1, to=2-1]
    \end{tikzcd}
    \hspace{1em}
    \begin{tikzcd}[ampersand replacement=\&,cramped]
    	{\{\color{red}0, 1, 2, \dots \color{black}\}} \\
    	{ \{\color{red} x-1, y-1, z-1 \dots \color{black}\}}
        \arrow["\psi", maps to, from=1-1, to=2-1]
    \end{tikzcd}
    \caption{Operation of $\psi$ constructed from $\tau$}
    \label{fig:enter-label}
\end{figure}

\begin{theorem}
For all $\phi\colon \Fin[n]\xto{\sim}\Fin[n]$, $\ext{f}(n, i) \id \ext{f}(n, i \circ \phi)$.
\end{theorem}

\begin{itemize}
    \item On $n = 0$, $\ext{f}(n, i) \id \ext{f}(n, i \circ \phi) = e$.
    \item On $n = S(m)$,
        \begin{align*}
        & \ext{f}(S(m), i \circ \phi) \\
        & = \ext{f}(S(m), i \circ \tau) \\
        & = f(i(\tau(0))) \otimes \ext{f}(m, i \circ \tau \circ S) & \text{(by definition)} \\
        & = f(i(0)) \otimes \ext{f}(m, i \circ \tau \circ S) & \text{(by property of $\tau$)} \\
        & = f(i(0)) \otimes \ext{f}(m, i \circ S \circ \psi) \\
        & = f(i(0)) \otimes \ext{f}(m, i \circ S) & \text{(induction)} \\
        & = \ext{f}(S(m), i)
        \end{align*}
\end{itemize}

