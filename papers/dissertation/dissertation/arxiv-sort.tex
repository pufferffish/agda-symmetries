\RequirePackage{currfile}
\ifcurrfilename{arxiv-sort.tex}%
{\documentclass[acmsmall,nonacm,natbib=false]{acmart}}
{\documentclass[acmsmall,anonymous,screen,review,natbib=false]{acmart}}
\settopmatter{printfolios=true,printccs=false,printacmref=false}

%% Rights management information.  This information is sent to you
%% when you complete the rights form.  These commands have SAMPLE
%% values in them; it is your responsibility as an author to replace
%% the commands and values with those provided to you when you
%% complete the rights form.
\setcopyright{none}
\copyrightyear{2024}
\acmYear{2024}
% \acmDOI{XXXXXXX.XXXXXXX}

%% These commands are for a PROCEEDINGS abstract or papier.
\acmConference[ICFP'24]{International Conference on Functional Programming}{Sep 2--7, 2024}{Milan, Italy}
%%
%%  Uncomment \acmBooktitle if the title of the proceedings is different
%%  from ``Proceedings of ...''!
%%
%%\acmBooktitle{Woodstock '18: ACM Symposium on Neural Gaze Detection,
%%  June 03--05, 2018, Woodstock, NY}
% \acmPrice{15.00}
% \acmISBN{978-1-4503-XXXX-X/18/06}

%%
%% Submission ID.
%% Use this when submitting an article to a sponsored event. You'll
%% receive a unique submission ID from the organizers
%% of the event, and this ID should be used as the parameter to this command.
%%\acmSubmissionID{123-A56-BU3}

%%
%% For managing citations, it is recommended to use bibliography
%% files in BibTeX format.
%%
%% You can then either use BibTeX with the ACM-Reference-Format style,
%% or BibLaTeX with the acmnumeric or acmauthoryear sytles, that include
%% support for advanced citation of software artefact from the
%% biblatex-software package, also separately available on CTAN.
%%
%% Look at the sample-*-biblatex.tex files for templates showcasing
%% the biblatex styles.
%%
% \bibliographystyle{ACM-Reference-Format}
\RequirePackage[
  % datamodel=acmdatamodel,
  style=acmauthoryear,
  sortcites=true,
  sorting=ynt,
  backend=biber,
  ]{biblatex}

\addbibresource{../symmetries.bib}

%%
%% The majority of ACM publications use numbered citations and
%% references.  The command \citestyle{authoryear} switches to the
%% "author year" style.
%%
%% If you are preparing content for an event
%% sponsored by ACM SIGGRAPH, you must use the "author year" style of
%% citations and references.
%% Uncommenting
%% the next command will enable that style.
% \citestyle{acmauthoryear}

\usepackage{currfile}
\ifcurrfilename{icfp24-sort-strip.tex}%
{\usepackage[appendix=strip,bibliography=common]{apxproof}}%
{\usepackage[appendix=append,bibliography=common]{apxproof}}
\usepackage[useregional]{datetime2}

%% figures
\usepackage{subcaption}
\usepackage{float}
\usepackage{afterpage}
%\usepackage[section,above]{placeins}

%% tikz
\usepackage{tikz}
\usepackage{xifthen}

%% math
\let\Bbbk\relax
\usepackage{amsmath,amsfonts,amsthm,amssymb}
\usepackage{newtxmath}
\usepackage{bbold}
\usepackage{mathpartir}
\usepackage{stmaryrd}
\usepackage{quiver}
\usepackage{ebproof}
\usepackage{listings}

\lstset{%
  basicstyle=\ttfamily,
  columns=fullflexible,
  keepspaces=true,
  mathescape=true,      % Allow escaping to LaTeX math mode within $$
  escapechar=\%,        % Set the escape character (e.g., % for LaTeX commands)
}


%% typesetting
%% fancy quotation
\usepackage[
  style=english,
  english=american,
  french=guillemets,
  autopunct=true,
  csdisplay=true
]{csquotes}
%% customize quote margins
\newenvironment*{innerquote}
  {\setlength{\leftmargini}{0.2cm}%
   \quote}
  {\endquote}
\SetBlockEnvironment{innerquote}

%% special hyphenation
\hyphenation{
  sorting
}
%% tweak hyphenation
\tolerance=9999
\emergencystretch=10pt
\hyphenpenalty=10000
\exhyphenpenalty=100
%% microtypography
\usepackage{microtype}

%% space saving magic macro
% \usepackage[
%   all=normal,
%   paragraphs,
%   floats,
%   wordspacing,
%   charwidths,
%   mathdisplays,
%   indent,
% ]{savetrees}

%% custom macros
\usepackage{hott}
\usepackage{math}
\usepackage{code}
\usepackage{macros}

%%
%% end of the preamble, start of the body of the document source.
\begin{document}

%%
%% The "title" command has an optional parameter,
%% allowing the author to define a "short title" to be used in page headers.
\title{Symmetries and Sorting}
\subtitle{symmetry kills order, sorting resurrects order}
%\subtitle{The order is dead. Long live the order!}
%%
%% The "author" command and its associated commands are used to define
%% the authors and their affiliations.
%% Of note is the shared affiliation of the first two authors, and the
%% "authornote" and "authornotemark" commands
%% used to denote shared contribution to the research.
\author{Wind Wong}
% \orcid{0000-0000-0000-0000}
\affiliation{
  \department{Department of Computer Science}
  \institution{University of Glasgow}
  \city{Glasgow}
  \postcode{G12 8QQ}
  \country{UK}
}
\email{me@windtfw.com}

\author{Vikraman Choudhury}
\orcid{0000-0003-2030-8056}
\email{vikraman.choudhury@unibo.it}
\affiliation{
  \department{Dipartimento di Informatica -- Scienza e Ingegneria}
  \institution{Universit\`{a} di Bologna}
  \city{Bologna}
  \postcode{40126}
  \country{Italy}
}
\affiliation{
  \department{OLAS Team}
  \institution{INRIA}
  \city{Sophia-Antipolis}
  \country{France}
}
\authornote{Supported by EU Marie Sk\l{}odowska-Curie fellowship 101106046 ReGraDe-CS.}

\author{Simon J. Gay}
\orcid{0000-0003-3033-9091}
\email{Simon.Gay@glasgow.ac.uk}
\affiliation{
  \department{Department of Computer Science}
  \institution{University of Glasgow}
  \city{Glasgow}
  \postcode{G12 8QQ}
  \country{UK}
}
\authornote{In a supervisory role.}

%%
%% By default, the full list of authors will be used in the page
%% headers. Often, this list is too long, and will overlap
%% other information printed in the page headers. This command allows
%% the author to define a more concise list
%% of authors' names for this purpose.
% \renewcommand{\shortauthors}{Last updated on \DTMnow}

%%
%% The abstract is a short summary of the work to be presented in the
%% article.
\begin{abstract}
  Sorting algorithms are among the most fundamental in computing science, and are frequently studied in relation to functional programming. There have been many investigations of the correctness of sorting algorithms in terms of total orders and category theory. We provide a new perspective on this topic by approaching it via universal algebra, viewing a sorting algorithm as a section (right inverse) to a surjective function from a free monoid to a free commutative monoid. Our main contribution is a new axiomatization of correct sorting algorithms, using two axioms that are not expressed in terms of a pre-existing total order. Instead, the total order can be recovered from the definition of an algorithm that satisfies the axioms. Our theory is formalized in Cubical Agda.  





%Sorting algorithms are one of the most common algorithms in functional programming.
%Previous works have investigated the correctness of sorting algorithms in terms of
%total orders and category theory. 
%
%We provide another perspective on the correctness of sorting algorithms with
%universal algebra by understanding it in terms of surjective functions from
%free monoids to free commutative monoids, leveraging the power of cubical
%type theory to implement theses ideas.
\end{abstract}

%%
%% The code below is generated by the tool at http://dl.acm.org/ccs.cfm.
%% Please copy and paste the code instead of the example below.
%%
\begin{CCSXML}
  <ccs2012>
  <concept>
  <concept_id>10003752.10003790.10011740</concept_id>
  <concept_desc>Theory of computation~Type theory</concept_desc>
  <concept_significance>500</concept_significance>
  </concept>
  <concept>
  <concept_id>10003752.10010124.10010131.10010137</concept_id>
  <concept_desc>Theory of computation~Categorical semantics</concept_desc>
  <concept_significance>500</concept_significance>
  </concept>
  <concept>
  <concept_id>10003752.10010124.10010131.10010133</concept_id>
  <concept_desc>Theory of computation~Denotational semantics</concept_desc>
  <concept_significance>500</concept_significance>
  </concept>
  <concept>
  <concept_id>10011007.10011006.10011008.10011009.10011012</concept_id>
  <concept_desc>Software and its engineering~Functional languages</concept_desc>
  <concept_significafnce>500</concept_significance>
  </concept>
  <concept>
  <concept_id>10011007.10011006.10011039.10011040</concept_id>
  <concept_desc>Software and its engineering~Syntax</concept_desc>
  <concept_significance>500</concept_significance>
  </concept>
  <concept>
  <concept_id>10011007.10011006.10011039.10011311</concept_id>
  <concept_desc>Software and its engineering~Semantics</concept_desc>
  <concept_significance>500</concept_significance>
  </concept>
  </ccs2012>
\end{CCSXML}

\ccsdesc[500]{Theory of computation~Type theory}
\ccsdesc[500]{Theory of computation~Categorical semantics}
\ccsdesc[500]{Theory of computation~Denotational semantics}
\ccsdesc[500]{Software and its engineering~Functional languages}
\ccsdesc[500]{Software and its engineering~Syntax}
\ccsdesc[500]{Software and its engineering~Semantics}

%%
%% Keywords. The author(s) should pick words that accurately describe
%% the work being presented. Separate the keywords with commas.
\keywords{universal algebra, category theory, type theory, homotopy type theory, combinatorics, formalization}
%% A "teaser" image appears between the author and affiliation
%% information and the body of the document, and typically spans the
%% page.

%%
%% This command processes the author and affiliation and title
%% information and builds the first part of the formatted document.
\maketitle

\begin{acks}
\end{acks}

\section{Introduction}
\label{sec:introduction}

We consider a puzzle about sorting,
inspired by Dijkstra's Dutch National Flag problem~\cite[Ch.14]{dijkstraDisciplineProgramming1997}.
Suppose there are balls of three colors,
corresponding to the colors of the Dutch flag: red, white, and blue.
\[
  \{
  \tikz[anchor=base, baseline]{%
    \foreach \x/\color in {1/red,2/white,3/blue} {
        \node[circle,draw,fill=\color,line width=1pt] at (\x,0) {\phantom{\tiny\x}};
        \ifthenelse{\NOT 3 = \x}{\node at ({\x+0.5},0) {,};}{}
      }
  }
  \}
\]
Given a finite sequence of such balls, how many ways can you sort them?
\[
  [
      \tikz[anchor=base, baseline]{%
        \foreach \x/\color in {1/red,2/red,3/blue,4/white,5/blue,6/red,7/white,8/blue} {
            \node[circle,draw,fill=\color,line width=1pt] at (\x,0) {\phantom{\tiny\x}};
            \ifthenelse{\NOT 8 = \x}{\node at ({\x+0.5},0) {,};}{}
          }
      }
    ]
\]
Intuitively, what should a correct sorting algorithm do?
%
All it knows about balls are their colors, so all it can do is compare balls by color, and move them around.
%
A correct sorting algorithm is going to rearrange these balls so the sequence gets partitioned into ``stripes''.
\[
  [
      \tikz[anchor=base, baseline]{%
        \foreach \x/\color in {1/red,2/red,3/red,4/white,5/white,6/blue,7/blue,8/blue} {
            \node[circle,draw,fill=\color,line width=1pt] at (\x,0) {\phantom{\tiny\x}};
            \ifthenelse{\NOT 8 = \x}{\node at ({\x+0.5},0) {,};}{}
          }
      }
    ]
\]
There are 3 partitions corresponding to the 3 colors, hence there are $3! = 6$ ways to arrange the partitions.
%
Thus, the numbered of correct sorting functions is equal to the number of ways of ordering the colors!

\subsection*{Solving the puzzle}

Suppose $A$ is the 3-element set $\{a,b,c\}$. There are many possible orderings of these elements,
eg, $[a,b,c]$, or $[a,c,b]$, $[b,c,a]$, 6 of them.
Consider lists of $A$, such as $[a,b,c,a,c,b,c,c]$, and we want to sort this list, so we want to write a sort function $s\colon List(A) \to List(A)$.
Obviously, there would be countably many functions $List(A) \to List(A)$, but how many of them
would actually be correct sorting functions? We posit that there are only 6 correct sorting
functions, and our justification is because there are only 6 possible ordering on $A$.
But how do we prove this?

Let's try to write some functions $List(A) \to List(A)$, for example:
\begin{lstlisting}[language=haskell]
s1 : List A $\to$ List A
s1 [] = []
s1 (x :: xs) = if x = a then ... else ...
\end{lstlisting}
$s1$ is a sort function only if $A$ is ordered by $a < c < b$.
In fact, this goes both ways!
If we write a correct sort function $s$, we will get a total order on $A$.
We gain our main insight: fixing a permutation on $\{a,b,c\}$ fixes an ordering on the elements which determines a correct sort function for lists of $A$.

%% Now, suppose you have lists of elements of $A$, such as, $[(a,"foo"), (b, "bar"), (c, "baz")]$, and we want to sort this list using the first element of the tuple as the key -- in how many ways can we sort correctly?

%% Sorting algorithm is a common algorithm we introduce to undergraduate students
%% when we teach them algorithmics. For example, a sort algorithm can be a function
%% $\text{List} \, \mathbb{N} \to \text{List} \, \mathbb{N}$, which produces $[1,2,3,4]$
%% given the input $[2,3,1,4]$.
However, how do we formalize the correctness of this sort function?
The common formalization of a correct sort algorithm is to say a sort algorithm produces
an "orederd list", for example, one such formalization would be the definition
from Verified Functional Algorithms \cite{appel2016verified}, translated into Agda.

\begin{lstlisting}[language=Haskell]
data Sorted : List A $\to$ Type where
  sorted-nil : Sorted []
  sorted-one : $\forall$ x $\to$ Sorted [ x ]
  sorted-cons : $\forall$ x y zs $\to$ x $\leq$ y $\to$ Sorted (y :: zs) $\to$ Sorted (x :: y :: zs)
\end{lstlisting}

This is a perfectly valid formalization of an ordered list! Empty lists and singleton lists
are ordered, and we can inductively construct an "orderness witness" by prepending a new element
to an ordered list, provided that the new element is less than the head of the old ordered list of
course. This definition, however, assumes an existing total order on $A$. This is
unsatisfactory, in a way, because a sorting algorithm is fundamentally just a function. Can we
axiomatize the correctness of a sorting algorithm purely by the properties of functions?

In a sense, this problem has already been solved, first by Hinze et al. \cite{10.1145/2364394.2364405}
and later extended by Alexandru in their thesis \cite{alexandru_intrinsically_2023}.
Their formalization is defined in terms of bialgebras, which not
only captures the correctness of sorting algorithms purely in a categorical settings, but
also isolate the computational essence of sorting algorithms in terms of distributive laws,
allowing us to construct more sorting algorithms "for free". Their work are thus necessarily
below the level of extensional equality, i.e. input-output behavior, and allow us to reason
with the structures of the sorting algorithms themselves. Our work only concerns the correctness
of sorting algorithms, with the goal to axiomatize sorting functions as functions satisfying
some abstract properties, independent of a given ordering, which allows us to gain
insight into how sorting relates to order and vice versa.
% and its implications on axiom of choice.
% maybe write more on its relationship to AC, in a sepreate paragraph?

\subsection*{Sorting and commutativity}
What a sorting algorithm does ultimately is to turn an unordered list into an ordered list,
and commutativity gives us insight into how an unordered list can be constructed.
Finite multiset for example would be a commutative data structure, where $xs \cup ys = ys \cup xs$.
Consider the canonical map $q$ from list to finite multiset, which would map
$[a, a, b, c]$ to $\{a, a, b, c\}$. A section $s$ to $q$ must perform sorting of some kind.
To illustrate this, we note $s(\{a, a, b, c\}) = s(\{b, a, c, a\})$, because commutativity
of finite multiset gives us $\{a, a, b, c\} = \{b, a, c, a\}$. This of course applies to
any permutation of $\{a, a, b, c\}$. Commutativity is a way of enforcing unordering,
or forgetting ordering, and by constructing a list from an unordered data structure, $s$ itself
must impose an ordering of some kind.

We can now study sorting from an algebraic perspective, viewing sorting as mappings from free commutative
monoids to free monoids. There are many ways to construct free commutative monoids, such as using adjacent transpositions to generate all symmetries, or quotienting free monoids by symmetries, which we would
elaborate on in ~\cref{sec:commutative-monoids}, all of which can be very naturally represented
using higher inductive types in univalent type theory. We also
create a framework for universal algebra, allowing us to formalize the notion of free algebras and
their universal property, and finally within the framework we can formalize the axioms of sorting
algorithms and their relationship to total orders.
% Univalent type theory and Cubical Agda gives us a
% rigorous framework for implementing these ideas, and higher inductive types allow us to express
% the constructions of free commutative monoids very naturally.

% \begin{itemize}
%     \item What exactly is special about commutativity? We study this from the ordinateur science point of view, using monoids and commutative monoids, which are important data structures.
%     \item In some sense, commutativity or unordering is formally understood as non-determinism, such as in concurrency using finite powersets or free commutative monoids. When can a commutative structure be linearised?
%     \item Commutativity is a way of enforcing unordering, or forgetting ordering, there are many ways of doing it: by using adjacent transpositions to generate all symmetries, or quotienting by symmetries
%     \item if you forget the ordering and get an unordered list (or data structure), can you recover the ordering? Of course, it's a sorting algorithm, that is very basic to ordinateur science. In fact, this gives us a specification for sorting, which is our application.
%     \item Why do we want to write intrinsically verified sorting algorithms?
%         \begin{itemize}
%         \item Using HITs, the sort function is correct by construction. This is unlike verifying
%             sort externally in Coq.
%         \item Can we structure the program in a better way? This is already solved by the bialgebraic sorting point of view, are we doing anything new?
%         \item We give a more conceptual understanding of the correctness of sorting algorithms, using an abstract framework.
%         \item Conventional way: a sort function turns a list into an ordered list.
%         \item Our way: a sort function is a function satisfying some abstract properties, independent of a given ordering.
%         \end{itemize}
% \end{itemize}

% overall plan: with the universal algebra



\subsubsection*{Outline and Contributions}

\begin{myitemize}
  \item In~\cref{sec:type-theory}, we discuss the basics of HoTT and Cubical Agda, and describe the notation we use in the papier.
  \item In~\cref{sec:universal-algebra}, we give some background on a categorical framework for universal algebra including equations, and its formalization in HoTT. We give the definition of free algebras and their universal property. \vc{We do not give the construction of free algbras for arbitrary signatures, because we show that their existence implies non-constructive principles.}
  \item In~\cref{sec:monoids}, we give various constructions of free monoids, and their proofs of universal property.
  \item In~\cref{sec:commutative-monoids}, we add commutativity to free monoids, and show how to extend the proofs of universal property appropriately, to get the universal property of free commutative monoids.
  \item \vc{In~\cref{sec:combinatorics}, we show how free monoids and free commutative monoids are different, by discussing various combinatorial properties and operations, which can be defined for both, and ones which cannot be defined for both}.
  \item In \cref{sec:sorting}, we build on the constructions of the previous sections and study intrinsically verified sorting function. The main result in this section is to connect total orders and sorting and commutativity, by proving an equivalence between decidable total orders on a carrier set $A$, and correct sorting algorithms on lists of elements of $A$.
  \item In~\cref{sec:discussion}, we discuss connections to related work, and future work.
\end{myitemize}
 %% 2.5 pages
\section{Notation}\label{sec:notation}
As we have explained, our work is formalized in Cubical Agda and Cubical Type Theory,
which is a variant of Homotopy Type Theory that is designed to preserve
computational properties of type theory.
We refer the readers to other works such as~\cite{vezzosiCubicalAgdaDependently2019}
and~\cite{cohenCubicalTypeTheory2018} for a more in-depth explanation on Cubical Type Theory
and how we can program in Cubical Agda. We give a quick overview of the type
theory notation in use in this paper.

We denote the type of types with $\mathcal{U}$. 
In practice, $\mathcal{U}$ would be indexed by a type level 
à la Russell for consistency, however we opted to omit the type level
for simplicity and clarity.
We use $\Fin[n]$ to denote finite sets of cardinality $n$ in HoTT~\cite{yorgeyCombinatorialSpeciesLabelled2014a}.
This is defined as follow:
\vspace{-1em}
\begin{code}
Fin : $\mathbb{N}$ -> UU
Fin n = $\Sigma$[ m $\in$ $\mathbb{N}$ ] (m < n)
\end{code}

\noindent
There are multiple ways to define $\Fin$, one example being an indexed inductive type,
for example:
\vspace{-1em}
\begin{code}
data Fin : $\mathbb{N}$ -> UU where
    fzero : $\forall$ {n : $\mathbb{N}$} -> Fin (fsuc n)
    fsuc  : $\forall$ {n : $\mathbb{N}$} -> Fin n -> Fin (fsuc n)
\end{code}
In our construction we opted to use this definition because cubical Agda does not behave well
when pattern matching on indexed inductive types.

We also use $\times$ to denote product types and $+$ to denote coproduct types.
For mere propositions, we use $\land$ to denote logical and, and $\vee$ to denote logical or.
In Cubical Agda these are defined as propositionally-truncated products and coproducts. %% 1 page
\section{Universal Algebra}
\label{sec:universal-algebra}

We first develop some basic notions from universal algebra and equational
logic~\cite{birkhoffStructureAbstractAlgebras1935}.
%
Universal algebra is the abstract study of algebraic structures, which have (algebraic) operations and (universal)
equations.
%
This gives us some vocabulary and a framework to express our results in.
%
The point of view we take is the standard category-theoretic approach to universal algebra, which predates the Lawvere
theory or abstract clone point of view.
%
We keep a running example of monoids in mind, while explaining and defining the abstract concepts.
% In the language of universal algebra, such a structure is formalized by giving a signature of operations, and a
% structure being a carrier set with functions that interpret these operations. We describe such a framework for universal
% algebra in HoTT, as follows.

\begin{definition}[Signature]
    A signature, denoted $\sigma$, is a (dependent) pair consisting of:
    \begin{itemize}
        \item a set of operations, $\term{op} : \Set$,
        \item an arity function for each symbol, $\term{ar} : \term{op} \to \Set$.
    \end{itemize}
\end{definition}

\begin{example}
    A monoid is a set with an identity element (or a nullary operation), and a binary multiplication operation.
    %
    The signature for monoids $\sigma_{\Mon}$ is encoded as:
    $(\Fin[2],\lambda \{0 \mapsto \Fin[0] ; 1 \mapsto \Fin[2] \})$.
    %
    Informally, the set of operations is the two-element set $\{e,\mult\}$, which is written as $\Fin[2]$,
    and the arity function picks out a (finite) set denoting the (finite) arity of each operation.
    %
    Of course, this is an example of a finitary signature,
    but in general the arity function can be any (not necessarily finite) set.
\end{example}

Every signature $\sigma$ induces a signature functor on $\Set$, written $F_{\sigma}$.

\begin{definition}[Signature functor]
    \begin{align*}
        F_{\sigma} \colon \Set & \to \Set                                     \\
        X                      & \mapsto \dsum{o:\term{op}}{X^{\term{ar}(o)}} \\
        X \xto{f} Y            & \mapsto
        \dsum{o:\term{op}}{X^{\term{ar}(o)}}
        \xto{(o, \blank \comp f)}
        \dsum{o:\term{op}}{Y^{\term{ar}(o)}}
    \end{align*}
\end{definition}

\begin{example}
    The signature functor for monoids, $F_{\sigma_{\Mon}}$, assigns to a carrier set $X$,
    the sets of inputs for each operation.
    %
    Expanding the dependent product on $\Fin[2]$, we obtain a coproduct or disjoint union of sets:
    $F_{\sigma_{\Mon}}(X) \eqv X^{\Fin[0]} + X^{\Fin[2]}$.
\end{example}

A $\sigma$-structure is given by a carrier set, with functions interpreting each operation symbol.
%
The signature functor applied to a carrier set gives the inputs to each operation, and the output is simply a map back
to the carrier set.
%
Formally, these two pieces of data are an algebra for the $F_{\sigma}$ functor.
%
We write $\str{X}$ for a $\sigma$-structure with carrier set $X$, following the model-theoretic notational convention.

\begin{definition}[Structure]
    A $\sigma$-structure $\str{X}$ is an $F_{\sigma}$ algebra, that is:
    \begin{itemize}
        \item a carrier set $X$, and
        \item an algebra map $\alpha_{X}\colon F_{\sigma}(X) \to X$.
    \end{itemize}
\end{definition}

\begin{example}
    Concretely, an $F_{\sigma_{\Mon}}$-algebra has the type
    \[
        \begin{array}{rcl}
            \alpha_{X} & :    & \dsum{o:\Fin[2]}{X^{\term{ar}(o)}} \to X       \\
                       & \eqv & (X^{\Fin[0] + \Fin[2]}) \to X                  \\
                       & \eqv & (X^{\Fin[0]} \to X) \times (X^{\Fin[2]} \to X) \\
                       & \eqv & (\unitt \to X) \times (X \times X \to X)
        \end{array}
    \]
    The natural numbers with the carrier set $\Nat$,
    and the constant $0$ and the addition operation $+$ give an example of such a structure.
    %
    Similarly, with the constant $1$ and the multiplication operation $\times$, we have another monoid structure.
\end{example}

\begin{definition}[Homomorphism]
    A homomorphism is
\end{definition}

For any object \( \mathfrak{Y} \) in $\sigma$-Alg, $(\blank) \circ \eta_X$ is an equivalence:

\begin{figure}[H]
    \centering
    % https://q.uiver.app/#q=WzAsMyxbMCwwLCJYIl0sWzIsMCwiVShBKSJdLFsyLDIsIlUoQikiXSxbMCwxLCJpIiwwLHsiY29sb3VyIjpbMSwxMDAsNjBdfSxbMSwxMDAsNjAsMV1dLFswLDIsImciLDJdLFsxLDIsIlUoZikiLDAseyJzdHlsZSI6eyJib2R5Ijp7Im5hbWUiOiJkb3R0ZWQifX19XV0=
    \[\begin{tikzcd}[ampersand replacement=\&]
            \mathfrak{F}(X) \\
            \\
            \mathfrak{Y}
            \arrow["f", dotted, from=1-1, to=3-1]
        \end{tikzcd}
        \mapsto
        \begin{tikzcd}[ampersand replacement=\&]
            X \&\& {F(X)} \\
            \\
            \&\& {Y}
            \arrow["\eta_X", color={rgb,255:red,255;green,54;blue,51}, from=1-1, to=1-3]
            \arrow["f \circ \eta_X"', from=1-1, to=3-3]
            \arrow["{f}", dotted, from=1-3, to=3-3]
        \end{tikzcd}\]
    \caption{Universal property of free algebras}
    \label{fig:universal-property}
\end{figure}



In this paper, we are interested in the constructions of free monoids and free commutative monoids. %% 3 pages
% !TEX root = cpp25-sort.tex

\section{Constructions of Free Monoids}
\label{sec:monoids}

In this section, we consider various constructions of free monoids in type theory, with proofs of their universal
property.
%
Since each construction satisfies the same categorical universal property,
by~\cref{lem:free-algebras-unique},
these are canonically equivalent (hence equal, by univalence) as types (and as monoids),
allowing us to transport proofs between them.
%
Using the unviersal property allows us to define and prove our programs correct in one go,
which is used in~\cref{sec:application}.

\subsection{Lists}
\label{mon:lists}

Cons-lists (or sequences) are simple inductive datatypes, well-known to functional programmers,
and are the most common representation of free monoids in programming languages.
%
In category theory, these correspond to Kelly's notion of algebraically-free
monoids~\cite{kellyUnifiedTreatmentTransfinite1980}.
%
\begin{definition}[Lists]
    \label{def:lists}
    \leavevmode
    \begin{code}
data List (A : UU) : UU where
  nil : List A
  _cons_ : A -> List A -> List A
\end{code}
\end{definition}
%
The (universal) generators map is the singleton: $\eta_A(a) \defeq [a] \jdgeq$~\inline{a cons nil},
the identity element is the empty list~\inline{nil},
and the monoid operation $\doubleplus$ is given by concatenation.
\begin{toappendix}

    \begin{definition}[Concatenation]
        We define the concatenation operation $\doubleplus : \List(A) \to \List(A) \to \List(A)$,
        by recursion on the first argument:
        \begin{align*}
            [] \doubleplus ys           & = ys                          \\
            (x \cons xs) \doubleplus ys & = x \cons (xs \doubleplus ys)
        \end{align*}
    \end{definition}
    The proof that $\doubleplus$ satisfies monoid laws is straightforward (see the formalization for details).

    \begin{definition}[Universal extension]
        For any monoid $\str{X}$, and given a map $f : A \to X$,
        we define the extension $\ext{f} : \List(A) \to \mathfrak{X}$ by recursion on the list:
        \begin{align*}
            \ext{f}([])         & = e                       \\
            \ext{f}(x \cons xs) & =  f(x) \mult \ext{f}(xs)
        \end{align*}
    \end{definition}
\end{toappendix}

\begin{propositionrep}
    $\ext{(\blank)}$ lifts a function $f : A \to X$ to a monoid homomorphism $\ext{f} : \List(A) \to \mathfrak{X}$.
\end{propositionrep}

\begin{proof}
    To show that $\ext{f}$ is a monoid homomorphism,
    we need to show $\ext{f}(xs \doubleplus ys) = \ext{f}(xs) \mult \ext{f}(ys)$.
    We can do so by induction on $xs$.

    Case []:
    $\ext{f}([] \doubleplus ys) = \ext{f}(ys)$,
    and $\ext{f}([]) \mult \ext{f}(ys) = e \mult \ext{f}(ys) = \ext{f}(ys)$
    by definition of $\ext{(\blank)}$. Therefore, we have
    $\ext{f}([] \doubleplus ys) = \ext{f}([]) \mult \ext{f}(ys)$.

    Case $x \cons xs$:
    \begin{align*}
         & \ext{f}((x \cons xs) \doubleplus ys)                                                        \\
         & = \ext{f}(([ x ] \doubleplus xs) \doubleplus ys) & \text{by definition of concatenation}    \\
         & = \ext{f}([ x ] \doubleplus (xs \doubleplus ys)) & \text{by associativity}                  \\
         & = \ext{f}(x \cons (xs \doubleplus ys))           & \text{by definition of concatenation}    \\
         & = f(x) \mult \ext{f}(xs \doubleplus ys)          & \text{by definition of $\ext{(\blank)}$} \\
         & = f(x) \mult (\ext{f}(xs) \mult \ext{f}(ys))     & \text{by induction}                      \\
         & = (f(x) \mult \ext{f}(xs)) \mult \ext{f}(ys)     & \text{by associativity}                  \\
         & = \ext{f}(x \cons xs) \mult \ext{f}(ys)          & \text{by definition of $\ext{(\blank)}$} \\
    \end{align*}

    Therefore, $\ext{(\blank)}$ does correctly lift a function to a monoid homomorphism.
\end{proof}

\begin{propositionrep}[Universal property for List]
    $(\List(A),\eta_A)$ is the free monoid on $A$.
\end{propositionrep}

\begin{proof}
    To show that $\ext{(\blank)}$ is an inverse to $\blank \comp \eta_A$,
    we first show $\ext{(\blank)}$ is the right inverse to $\blank \comp \eta_A$.
    For all $f$ and $x$, $(\ext{f} \circ \eta_A)(x) = \ext{f}(x \cons []) = f(x) \mult e = f(x)$,
    therefore by function extensionality, for any $f$, $\ext{f} \circ \eta_A = f$,
    and $(\blank \circ \eta_A) \comp \ext{(\blank)} = id$.

    To show $\ext{(\blank)}$ is the left inverse to $\blank \comp \eta_A$, we need to prove
    for any monoid homomorphism $f : \List(A) \to \mathfrak{X}$, $\ext{(f \comp \eta_A)}(xs) = f(xs)$.
    We can do so by induction on $xs$.

    Case []: $\ext{(f \comp \eta_A)}([]) = e$ by definition of the $\ext{(\blank)}$ operation,
    and $f([]) = e$ by homomorphism properties of $f$. Therefore, $\ext{(f \comp \eta_A)}([]) = f([])$.

    Case $x \cons xs$:
    \begin{align*}
         & \ext{(f \comp \eta_A)}(x \cons xs)                                                                \\
         & = (f \comp \eta_A)(x) \mult \ext{(f \comp \eta_A)}(xs) & \text{by definition of $\ext{(\blank)}$} \\
         & = (f \comp \eta_A)(x) \mult f(xs)                      & \text{by induction}                      \\
         & = f([x]) \mult f(xs)                                   & \text{by definition of $\eta_A$}         \\
         & = f([x] \doubleplus xs)                                & \text{by homomorphism properties of $f$} \\
         & = f(x \cons xs)                                        & \text{by definition of concatenation}
    \end{align*}

    By function extensionality, $\ext{(\blank)} \comp (\blank \circ \eta_A) = id$.
    Therefore, $\ext{(\blank)}$ and $(\blank) \circ [\_]$ are inverse of each other.

    We have now shown that $(\blank) \comp \eta_A$ is an equivalence from
    monoid homomorphisms $\List(A) \to \mathfrak{X}$
    to set functions $A \to X$, and its inverse is given by $\ext{(\blank)}$, which maps set
    functions $A \to X$ to monoid homomorphisms $\List(A) \to \mathfrak{X}$. Therefore, $\List$ is indeed
    the free monoid.
\end{proof}

\subsection{Array}\label{mon:array}

An alternate (non-inductive) representation of the free monoid on a carrier set,
or alphabet $A$, is $A^{\ast}$,
the set of all finite words or strings or sequences of characters \emph{drawn} from $A$,
which was known in category theory from~\cite{dubucFreeMonoids1974}.
%
In computer science, we think of this as an \emph{array},
which is a pair of a natural number $n$, denoting the length of the array,
and a lookup (or index) function $\Fin[n] \to A$, mapping each index to an element of $A$.
%
In type theory, this is also often understood as a container~\cite{abbottCategoriesContainers2003},
where $\Nat$ is the type of shapes, and $\Fin$ is the type (family) of positions.

\begin{definition}[Arrays]
    \label{def:arrays}
    \leavevmode
    \begin{code}
Array : UU -> UU
Array A = Sg(n : Nat) (Fin n -> A)
    \end{code}
\end{definition}
\vspace*{-2em}
\noindent
For example, $(3, \lambda\{ 0 \mapsto 3, 1 \mapsto 1, 2 \mapsto 2 \})$
represents the same list as $[3, 1, 2]$.
%
The (universal) generators map is the singleton: $\eta_A(a) = (1, \lambda\{ 0 \mapsto a \})$,
the identity element is $(0, \lambda\{\})$
and the monoid operation $\doubleplus$ is array concatenation.
%
\begin{lemmarep}
    \label{array:zero-is-id}
    Zero-length arrays $(0, f)$ are contractible.
\end{lemmarep}
\begin{proof}
    We need to show $f : \Fin[0] \to A$ is equal to $\lambda\{\}$.
    %
    By function extensionality this amounts to showing for all $x : \emptyt$, $f(x) = (\lambda\{\})(x)$,
    which holds by absurdity elimination on $x$.
    %
    Therefore, any array $(0, f)$ is equal to $(0, \lambda\{\})$.
\end{proof}

\begin{definition}[Concatenation]
    The concatenation operation $\doubleplus$, %% : \Array(A) \to \Array(A) \to \Array(A)$,
    is defined below, where $\oplus : (\Fin[n] \to A) \to (\Fin[m] \to A) \to (\Fin[n+m] \to\nolinebreak A)$
    is a combine operation:
    \begin{align*}
        (n , f) \doubleplus (m , g) & = (n + m , f \oplus g)        \\
        (f \oplus g)(k)             & = \begin{cases}
                                            f(k)     & \text{if}\ k < n \\
                                            g(k - n) & \text{otherwise}
                                        \end{cases}
    \end{align*}
\end{definition}

\begin{propositionrep}%[Monoid laws]
    $(\Array(A), \doubleplus)$ is a monoid.
\end{propositionrep}

\begin{proof}
    To show $\Array$ satisfies left unit,
    we want to show $(0, \lambda\{\}) \doubleplus (n, f) = (n, f)$.
    \begin{align*}
        (0 , \lambda\{\}) \doubleplus (n , f) & = (0 + n , \lambda\{\} \oplus f)      \\
        (\lambda\{\} \oplus f)(k)             & = \begin{cases}
                                                      (\lambda\{\})(k) & \text{if}\ k < 0 \\
                                                      f(k - 0)         & \text{otherwise}
                                                  \end{cases}
    \end{align*}

    It is trivial to see the length matches: $0 + n = n$. We also need to show $\lambda\{\} \oplus f = f$.
    Since $n < 0$ for any $n : \mathbb{N}$ is impossible, $(\lambda\{\} \oplus f)(k)$ would always reduce to
    $f(k - 0) = f(k)$, therefore $(0, \lambda\{\}) \doubleplus (n, f) = (n, f)$.

    To show $\Array$ satisfies right unit,
    we want to show $(n, f) \doubleplus (0, \lambda\{\}) = (n, f)$.
    \begin{align*}
        (n, f) \doubleplus (0 , \lambda\{\}) & = (n + 0, f \oplus \lambda\{\})           \\
        (f \oplus \lambda\{\})(k)            & = \begin{cases}
                                                     f(k)                 & \text{if}\ k < n \\
                                                     (\lambda\{\})(k - 0) & \text{otherwise}
                                                 \end{cases}
    \end{align*}

    It is trivial to see the length matches: $n + 0 = n$. We also need to show $f \oplus \lambda\{\} = f$.
    We note that the type of $f \oplus \lambda\{\}$ is $\Fin[n + 0] \to A$, therefore $k$ is of the type $\Fin[n + 0]$.
    Since $\Fin[n+0] \cong \Fin[n]$, it must always hold that $k < n$, therefore $(f \oplus \lambda\{\})(k)$ must
    always reduce to $f(k)$. Thus, $(n, f) \doubleplus (0, \lambda\{\}) = (n, f)$.

    For associativity, we want to show for any array $(n, f)$, $(m, g)$, $(o, h)$,
    $((n, f) \doubleplus (m, g)) \doubleplus (o, h) = (n, f) \doubleplus ((m, g) \doubleplus (o, h))$.

    \begin{align*}
        ((n, f) \doubleplus (m, g)) \doubleplus (o, h) & = ((n + m) + o, (f \oplus g) \oplus h)                \\
        ((f \oplus g) \oplus h)(k)                     & = \begin{cases}
                                                               \begin{cases}
                f(k)     & \text{if}\ k < n \\
                g(k - n) & \text{otherwise}
            \end{cases}
                                                                              & \text{if}\ k < n + m \\
                                                               h(k - (n + m)) & \text{otherwise}
                                                           \end{cases}                    \\
        (n, f) \doubleplus ((m, g) \doubleplus (o, h)) & = (n + (m + o), f \oplus (g \oplus h))                \\
        (f \oplus (g \oplus h))(k)                     & = \begin{cases}
                                                               f(k)                               & \ \text{k < n} \\
                                                               \begin{cases}
                g(k - n)     & \text{k - n < m} \\
                h(k - n - m) & \text{otherwise} \\
            \end{cases} & \text{otherwise}
                                                           \end{cases}
    \end{align*}

    We first case split on $k < n + m$ then $k < n$.

    Case $k < n + m$, $k < n$: Both $(f \oplus (g \oplus h))(k)$ and $((f \oplus g) \oplus h)(k)$ reduce to $f(k)$.

    Case $k < n + m$, $k \geq n$: $((f \oplus g) \oplus h)(k)$ reduce to $g(k - n)$ by definition.
    To show $(f \oplus (g \oplus h))(k)$ would also reduce to $g(k - n)$, we first need to show $\neg(k < n)$,
    which follows from $k \geq n$. We then need to show $k - n < m$.
    This can be done by simply subtracting $n$ from both side on $k < n + m$, which is well defined since $k \geq n$.

    Case $k \geq n + m$: $((f \oplus g) \oplus h)(k)$ reduce to $h(k - (n + m))$ by definition.
    To show $(f \oplus (g \oplus h))(k)$ would also reduce to $h(k - (n + m))$,
    we first need to show $\neg(k < n)$, which follows from $k \geq n + m$.
    We then need to show $\neg(k - n < m)$, which also follows from $k \geq n + m$.
    We now have $(f \oplus (g \oplus h))(k) = h(k - n - m)$. Since $k \geq n + m$, $h(k - n - m)$ is well defined
    and is equal to $h(k - (n + m))$, therefore $(f \oplus (g \oplus h))(k) = (f \oplus g) \oplus h)(k) = h(k - (n + m))$.

    In all cases $(f \oplus (g \oplus h))(k) = ((f \oplus g) \oplus h)(k)$, therefore associativity holds.
\end{proof}

\begin{lemmarep}[Array cons]\label{array:eta-suc}
    Any array $(S(n), f)$ is equal to $\eta_A (f(0)) \doubleplus (n, f \comp S)$.
\end{lemmarep}

\begin{proof}
    We want to show $\eta_A (f(0)) \doubleplus (n, f \comp S) = (S(n), f)$.
    \begin{align*}
        (1, \lambda\{ 0 \mapsto f(0) \}) \doubleplus (n , f \comp S) &
        = (1 + n, \lambda\{ 0 \mapsto f(0) \} \oplus (f \comp S))                                              \\
        (\lambda\{ 0 \mapsto f(0) \} \oplus (f \comp S))(k)          & = \begin{cases}
                                                                             f(0)               & \text{if}\ k < 1 \\
                                                                             (f \comp S)(k - 1) & \text{otherwise}
                                                                         \end{cases}
    \end{align*}

    It is trivial to see the length matches: $1 + n = S(n)$. We need to show
    $(\lambda\{ 0 \mapsto f(0) \} \oplus (f \comp S)) = f$.
    We prove by case splitting on $k < 1$.
    On $k < 1$, $(\lambda\{ 0 \mapsto f(0) \} \oplus (f \comp S))(k)$ reduces to $f(0)$.
    Since, the only possible for $k$ when $k < 1$ is 0, $(\lambda\{ 0 \mapsto f(0) \} \oplus (f \comp S))(k) = f(k)$
    when $k < 1$.
    On $k \geq 1$, $(\lambda\{ 0 \mapsto f(0) \} \oplus (f \comp S))(k)$ reduces to $(f \comp S)(k - 1) = f(S(k - 1))$.
    Since $k \geq 1$, $S(k - 1) = k$, therefore $(\lambda\{ 0 \mapsto f(0) \} \oplus (f \comp S))(k) = f(k)$
    when $k \geq 1$.
    Thus, in both cases, $(\lambda\{ 0 \mapsto f(0) \} \oplus (f \comp S)) = f$.
\end{proof}

\begin{lemmarep}[Array split]\label{array:split}
    For any array $(S(n), f)$ and $(m, g)$,
    \[
        (n + m, (f \oplus g) \comp S) = (n, f \comp S) \doubleplus (m, g)
        \enspace .
    \]
\end{lemmarep}

Informally, this means given an non-empty array $xs$ and any array $ys$,
concatenating $xs$ with $ys$ then dropping the first element is the same as
dropping the first element of $xs$ then concatenating with $ys$.

\begin{proof}
    It is trivial to see both array have length $n + m$. We want to show $(f \oplus g) \comp S = (f \comp S) \oplus g$.
    \begin{align*}
        ((f \oplus g) \comp S)(k) & = \begin{cases}
                                          f(S(k))        & \text{if}\ S(k) < S(n) \\
                                          g(S(k) - S(n)) & \text{otherwise}
                                      \end{cases} \\
        ((f \comp S) \oplus g)(k) & = \begin{cases}
                                          (f \comp S)(k) & \text{if}\ k < n \\
                                          g(k - n)       & \text{otherwise}
                                      \end{cases}
    \end{align*}

    We prove by case splitting on $k < n$.
    On $k < n$, $((f \oplus g) \comp S)(k)$ reduces to $f(S(k))$ since $k < n$ implies $S(k) < S(n)$,
    and $((f \comp S) \oplus g)(k)$ reduces to $(f \comp S)(k)$ by definition, therefore they are equal.
    On $k \geq n$, $((f \oplus g) \comp S)(k)$ reduces to $g(S(k) - S(n)) = g(k - n)$,
    and $((f \comp S) \oplus g)(k)$ reduces to $g(k - n)$ by definition, therefore they are equal.
\end{proof}

\begin{definition}[Universal extension]
    Given a monoid $\mathfrak{X}$, and a map $f : A \to X$,
    we define $\ext{f} : \Array(A) \to X$, by induction on the length of the array:
    \begin{align*}
        \ext{f}(0 , g)    & = e                                    \\
        \ext{f}(S(n) , g) & = f(g(0)) \mult \ext{f}(n , g \circ S)
    \end{align*}
\end{definition}

\begin{propositionrep}
    $\ext{(\blank)}$ lifts a function $f : A \to X$ to a monoid homomorphism $\ext{f} : \Array(A) \to \mathfrak{X}$.
\end{propositionrep}

\begin{proof}
    To show that $\ext{f}$ is a monoid homomorphism,
    we need to show $\ext{f}(xs \doubleplus ys) = \ext{f}(xs) \mult \ext{f}(ys)$.
    We can do so by induction on $xs$.

    Case $(0, g)$:
    We have $g = \lambda\{\}$ by~\cref{array:zero-is-id}.
    $\ext{f}((0, \lambda\{\}) \doubleplus ys) = \ext{f}(ys)$ by left unit,
    and $\ext{f}(0, \lambda\{\}) \mult \ext{f}(ys) = e \mult \ext{f}(ys) = \ext{f}(ys)$
    by definition of $\ext{(\blank)}$. Therefore,
    $\ext{f}((0, \lambda\{\}) \doubleplus ys) = \ext{f}(0, \lambda\{\}) \mult \ext{f}(ys)$.

    Case $(S(n), g)$: Let $ys$ be $(m, h)$.
    \begin{align*}
         & \ext{f}((S(n), g) \doubleplus (m, h))                                                                              \\
         & = \ext{f}(S(n + m), g \oplus h)                                 & \text{by definition of concatenation}            \\
         & = f((g \oplus h)(0)) \mult \ext{f}(n + m, (g \oplus h) \comp S) & \text{by definition of $\ext{(\blank)}$}         \\
         & = f(g(0)) \mult \ext{f}(n + m, (g \oplus h) \comp S)            & \text{by definition of $\oplus$, and $0 < S(n)$} \\
         & = f(g(0)) \mult \ext{f}((n, g \comp S) \doubleplus (m, h))      & \text{by~\cref{array:split}}                     \\
         & = f(g(0)) \mult (\ext{f}(n, g \comp S) \mult \ext{f}(m, h)))    & \text{by induction}                              \\
         & = (f(g(0)) \mult \ext{f}(n, g \comp S)) \mult \ext{f}(m, h))    & \text{by associativity}                          \\
         & = \ext{f}(S(n), g) \mult \ext{f}(m, h))                         & \text{by definition of $\ext{(\blank)}$}         \\
    \end{align*}

    Therefore, $\ext{(\blank)}$ does correctly lift a function to a monoid homomorphism.
\end{proof}

\begin{propositionrep}[Universal property for Array]
    \label{array:univ}
    \leavevmode
    $(\Array(A),\eta_A)$ is the free monoid on $A$.
\end{propositionrep}

\begin{proofsketch}
    We need to show that $\ext{(\blank)}$ is an inverse to ${(\blank) \comp \eta_A}$.
    $\ext{f} \comp \eta_A = f$ for all set functions $f : A \to X$ holds trivially.
    To show $\ext{(f \comp \eta_A)} = f$ for all homomorphisms $f : \Array(A) \to \mathfrak{X}$,
    we need $\forall xs.\, \ext{(f \comp \eta_A)}(xs) = f(xs)$.
    \cref{array:eta-suc,array:split} allow us to do induction on arrays,
    therefore we can prove $\forall xs.\, \ext{(f \comp \eta_A)}(xs) = f(xs)$ by induction on $xs$,
    very similarly to how this was proven for $\List$.
\end{proofsketch}

\begin{proof}
    To show that $\ext{(\blank)}$ is an inverse to $\blank \comp \eta_A$,
    we first show $\ext{(\blank)}$ is the right inverse to $\blank \comp \eta_A$.
    For all $f$ and $x$, $(\ext{f} \circ \eta_A)(x) = \ext{f}(1, \lambda\{0 \mapsto x\}) = f(x) \mult e = f(x)$,
    therefore by function extensionality, for any $f$, $\ext{f} \circ \eta_A = f$,
    and $(\blank \circ \eta_A) \comp \ext{(\blank)} = id$.

    To show $\ext{(\blank)}$ is the left inverse to $\blank \comp \eta_A$, we need to prove
    for any monoid homomorphism $f : \Array(A) \to \mathfrak{X}$, $\ext{(f \comp \eta_A)}(xs) = f(xs)$.
    We can do so by induction on $xs$.

    Case $(0, g)$:
    By~\cref{array:zero-is-id} we have $g = \lambda\{\}$.
    $\ext{(f \comp \eta_A)}(0, \lambda\{\}) = e$ by definition of the $\ext{(\blank)}$ operation,
    and $f(0, \lambda\{\}) = e$ by homomorphism properties of $f$.
    Therefore, $\ext{(f \comp \eta_A)}(0, g) = f(0, g)$.

    Case $(S(n), g)$, we prove it in reverse:
    \begin{align*}
         & f(S(n), g)                                                                                                     \\
         & = f(\eta_A(g(0)) \doubleplus (n, g \comp S))                        & \text{by~\cref{array:eta-suc}}           \\
         & = f(\eta_A(g(0))) \mult f(n, g \comp S)                             & \text{by homomorphism properties of $f$} \\
         & = (f \comp \eta_A)(g(0)) \mult \ext{(f \comp \eta_A)}(n, g \comp S) & \text{by induction}                      \\
         & = \ext{(f \comp \eta_A)}(S(n), g)                                   & \text{by definition of $\ext{(\blank)}$}
    \end{align*}

    By function extensionality, $\ext{(\blank)} \comp (\blank \circ \eta_A) = id$.
    Therefore, $\ext{(\blank)}$ and $(\blank) \circ [\_]$ are inverse of each other.

    We have now shown that $(\blank) \comp \eta_A$ is an equivalence from
    monoid homomorphisms $\Array(A) \to \mathfrak{X}$
    to set functions $A \to X$, and its inverse is given by $\ext{(\blank)}$, which maps set
    functions $A \to X$ to monoid homomorphisms $\Array(A) \to \mathfrak{X}$. Therefore, $\Array$ is indeed
    the free monoid.
\end{proof}

\subsubsection*{Remark}
An alternative proof of the universal property for $\Array$ can be given by directly constructing an equivalence (of
types, and monoid structures) between $\Array(A)$ and $\List(A)$ (using $\term{tabulate}$ and $\term{lookup}$), and then
using univalence and transport (see formalization).
 %% 2 pages
\section{Construction of Free Commutative Monoids}
\label{sec:commutative-monoids}

THe next step for us is to add commutativity -- extending our constructions of free monoids to free commutative monoids.
%
Informally, adding commutativity to free monoids turns ``ordered lists'' to ``unordered lists'',
where the ordering is the one imposed by the position or index of the elements in the list.
%
This is crucial to our goal of studying sorting,
as we will study sorting as a function from unordered lists to ordered lists.
which is later in~\cref{sec:sorting}.

It is known that finite multisets are commutative monoids (in fact, free commutative monoids),
under the operation of multiset union: $xs \cup ys = ys \cup xs$.
%
Its order is ``forgotten'' in the sense that it doesn't matter how two multisets are union-ed together,
or in more concrete terms, for example, $\{a, a, b, c\} = \{b, a, c, a\}$ are equal as finite multisets
(hence justifying the set notation).
%
This is unlike free monoids, using $\List$ for example,
where $[a, a, b, c] \neq [b, a, c, a]$ (hence justifying the list notation).

\subsection{Free monoids with a quotient}\label{cmon:qfreemon}

Instead of constructing free commutative monoids directly, the first construction we study is to take a free monoid and
impose commutativity on it.
%
We do this generally, by giving a construction of free commutative monoid as a quotient of \emph{any} free monoid.
%
Specific instances of this construction are given in \cref{cmon:plist} and \cref{cmon:bag}.

From the universal algebraic perspective developed in~\cref{sec:universal-algebra},
we can ask what it means to extend the equational theory of a given algebraic signature --
or how to construct a free commutative monoid as a quotient of a free monoid.
%
If $(\str{F}(A), \eta)$ is a free monoid construction satisfying its universal property,
then we'd like to quotient $F(A)$ by an \emph{appropriate relation} $\approx$,
that turns it into a free commutative monoid.
%
This is exactly the specification of a \emph{permutation relation}!

% $(\str{F}(A), i, \mult)$ quotiented by a permutation relation $({F(A) / \approx}, e, \doubleplus)$
% would be the free commutative monoid on $A$.

\begin{definition}[Permutation relation]
    \label{def:permutation-relation}
    \leavevmode
    A relation $\approx$ is a correct permutation relation iff it:
    \begin{itemize}
        \item is reflexive, symmetric, transitive (an equivalence),
        \item is a congruence wrt $\mult$: $a \mult b \to c \mult d \to a \mult c \approx b \mult d$,
        \item is commutative: $a \mult b \approx b \mult a$, and
        \item respects $\ext{(\blank)}$: $\forall f, \, a \approx b \to \ext{f}(a) = \ext{f}(b)$.
    \end{itemize}
\end{definition}

Let $q : F(A) \to {F(A) / \approx}$ be the quotient map (inclusion into the quotient).
%
The generators map is given by $q \comp \eta_A$, the identity element is $q(e)$,
and the $\doubleplus$ operation can also be lifted to the quotient by congruence.

\begin{proposition}
    $({\str{F}(A) / \approx}, \doubleplus, q(e))$ is a commutative monoid.
\end{proposition}

\begin{proof}
    Since $\approx$ is congruence wrt $\mult$,
    we can lift $\mult : F(A) \to F(A) \to F(A)$ to the quotient to obtain
    $\doubleplus : {F(A) / \approx} \to {F(A) / \approx} \to {F(A) / \approx}$.
    $\doubleplus$ would also satisfy the unit laws and associativity law which $\mult$ satisfy.
    Commutativity of $\doubleplus$ follows from the commutativity requirement of $\approx$,
    therefore $({F(A) / \approx}, \doubleplus, q(i))$ forms a commutative monoid.
\end{proof}

For clarity, we will use $\exthat{(\blank)}$ to denote the extension operation of $F(A)$,
and $\ext{(\blank)}$ for the extension operation of ${F(A) / \approx}$.

\begin{definition}
    Given a commutative monoid $\str{X}$ and a map $f : A \to X$,
    we define
    $\ext{f} : {\str{F}(A) / \approx} \; \to \mathfrak{X}$ as follows:
    we first obtain $\exthat{f} : \str{F}(A) \to \mathfrak{X}$ by universal property of $F$, and lift it
    to ${\str{F}(A)/\approx} \to \mathfrak{X}$, which is allowed since $\approx$ respects $\ext{(\blank)}$.
\end{definition}

Using the correct specification of a permutation relation, we can prove that $({\str{F}(A) / \approx})$ gives the free
commutative monoid on $A$.

\begin{lemma}
    For all $f : A \to X$, $x : F(A)$, $\ext{f}(q(x))$ reduces to $\exthat{f}(x)$.
\end{lemma}

\begin{proposition}[Universal property for ${\str{F}(A) / \approx}$]
    $({\str{F}(A) / \approx},\eta_A : A \xto{q} \str{F}(A) \to {\str{F}(A) / \approx})$
    is the free commutative monoid on $A$.
\end{proposition}

\begin{proof}
    To show that $\ext{(\blank)}$ is an inverse to $\blank \comp \eta_A$,
    we first show $\ext{(\blank)}$ is the right inverse to $\blank \comp \eta_A$.
    For all $f$ and $x$, $(\ext{f} \comp \eta_A)(x) = (\ext{f} \comp q)(\mu_A(x)) = \exthat{f}(\mu_A(x))$.
    By universal property of $F$, $\exthat{f}(\mu_A(x)) = f(x)$, therefore $(\ext{f} \comp \eta_A)(x) = f(x)$.
    By function extensionality, for any $f$, $\ext{f} \circ \eta_A = f$,
    and $(\blank \circ \eta_A) \comp \ext{(\blank)} = id$.

    To show $\ext{(\blank)}$ is the left inverse to $\blank \comp \eta_A$, we need to prove
    for any commutative monoid homomorphism $f : {F(A) / \approx} \to \mathfrak{X}$ and $x : {F(A) / \approx}$,
    $\ext{(f \comp \eta_A)}(x) = f(x)$. To prove this it is suffice to show for all $x : F(A)$,
    $\ext{(f \comp \eta_A)}(q(x)) = f(q(x))$.
    $\ext{(f \comp \eta_A)}(q(x))$ reduces to $\exthat{(f \comp q \comp \mu_A)}(x)$.
    Note that both $f$ and $q$ are homomorphism, therefore $f \comp q$ is a homomorphism. By
    universal property of $F$ we get $\exthat{(f \comp q \comp \mu_A)}(x) = (f \comp q)(x)$,
    therefore $\ext{(f \comp \eta_A)}(q(x)) = f(q(x))$.

    We have now shown that $(\blank) \comp \eta_A$ is an equivalence from
    commutative monoid homomorphisms ${F(A) / \approx} \to \mathfrak{X}$
    to set functions $A \to X$, and its inverse is given by $\ext{(\blank)}$, which maps set
    functions $A \to X$ to commutative monoid homomorphisms ${F(A) / \approx} \to \mathfrak{X}$.
    Therefore, ${F(A) / \approx}$ is indeed the free commutative monoid on $A$.
\end{proof}


\subsection{Lists quotiented by permutation}\label{cmon:plist}

A specific instance of this construction would be $\List$ quotiented with permutation to get commutativity. This
construction is also considered in~\cite{joramConstructiveFinalSemantics2023}, who gave a proof that $\PList$ is
equivalent to the free commutative monoid as a HIT. We give a direct proof of its universal property using our
generalistaion.

The permutation relation on lists is as follows, which swaps any two adjacent elements in the middle of the list.
This is only one example of such a relation, of course, there are many such in the literature.
\begin{definition}[PList]
    \leavevmode
    \begin{code}
data Perm (A : UU) : List A -> List A -> UU where
  perm-refl : forall {xs} -> Perm xs xs
  perm-swap : forall {x y xs ys zs}
           -> Perm (xs ++ x :: y :: ys) zs
           -> Perm (xs ++ y :: x :: ys) zs

PList : UU -> UU
PList A = List A / Perm
    \end{code}
\end{definition}

We have already given a proof of $\List$ being the free monoid in~\cref{mon:lists}.
By~\cref{cmon:qfreemon} it suffices to show $\Perm$ satisfies the axioms of permutation relation
to show $\PList$ is the free commutative monoid.

It is trivial to see how $\Perm$ satisfies reflexivity, symmetry, transitivity.
We can also prove $\Perm$ is congruent wrt to $\doubleplus$ inductively.
Commutativity is proven similarly, like the proof for $\SList$ in~\cref{slist:comm}.

\begin{proposition}\label{plist:sharp-sat}
    For all $f : A \to X$, $x, \, y : A$ and $xs, \, ys : \SList(A)$,
    $\ext{f}(xs\,\doubleplus\,x :: y :: ys) = \ext{f}(xs\,\doubleplus\,y :: x :: ys)$.
\end{proposition}

With this we can prove $\Perm$ respects $\ext{(\blank)}$.

\begin{proof}
    We can prove this by induction on $xs$. For $xs = []$, by homomorphism properties of $\ext{f}$,
    we can prove $\ext{f}(x :: y :: ys) = \ext{f}([ x ]) \mult \ext{f}([ y ]) \mult \ext{f}(ys)$.
    Since the image of $\ext{f}$ is a commutative monoid, we have
    $\ext{f}([ x ]) \mult \ext{f}([ y ]) = \ext{f}([ y ]) \mult \ext{f}([ x ])$, therefore proving
    $\ext{f}(x :: y :: ys) = \ext{f}(y :: x :: ys)$. For $xs = z :: zs$, we can prove
    $\ext{f}((z :: zs)\,\doubleplus\,x :: y :: ys) = \ext{f}([ z ]) \mult \ext{f}(zs\,\doubleplus\,x :: y :: ys)$.
    We can then complete the proof by induction to obtain
    $\ext{f}(zs\,\doubleplus\,x :: y :: ys) = \ext{f}(zs\,\doubleplus\,y :: x :: ys)$ and reassembling
    back to $\ext{f}((z :: zs)\,\doubleplus\,y :: x :: ys)$ by homomorphism properties of $\ext{f}$.
\end{proof}

\subsubsection*{Remark on representation}\label{plist:rep}
With this representation it is very easy to lift functions and properties defined on $\List$
to $\PList$ since $\PList$ is a quotient of $\List$. The inductive nature of $\PList$ makes it
very easy to define algorithms and proofs that are inductive in nature, e.g. defining insertion sort
on $\PList$ is very simple since insertion sort inductively sorts a list, which we can easily do by
pattern matching on $\PList$ since the construction of $\PList$ is definitionally inductive.
This property also makes it such that oftentimes inductively constructed $\PList$ would normalize to the
simplest form of the $\PList$, e.g. $q([ x ]) \doubleplus q([y, z])$ normalizes to $q([x,y,z])$ by
definition, saving the efforts of defining auxillary lemmas to prove their equality.

This inductive nature, however, makes it difficult to define efficient operations on $\PList$. Consider a
divide-and-conquer algorithm such as merge sort, which involves partitioning a $\PList$ of length $n+m$ into
two smaller $\PList$ of length $n$ and length $m$. The inductive nature of $\PList$ makes it such that
we must iterate all $n$ elements before we can make such a partition, thus making $\PList$ unintuitive
to work with when we want to reason with operations that involve arbitrary partitioning.

Also, whenever we define a function on $\PList$ by pattern matching we would also need to show
the function respects $\Perm$, i.e. $\Perm as\,bs \to f(as) = f(bs)$. This can be annoying because
of the many auxiliary variables in the constructor $\term{perm-swap}$, namely $xs$, $ys$, $zs$.
We need to show $f$ would respect a swap in the list anywhere between $xs$ and $ys$, which can
unnecessarily complicate the proof. For example in the inductive step of~\cref{plist:sharp-sat},
$\ext{f}((z :: zs)\,\doubleplus\,x :: y :: ys) = \ext{f}([ z ]) \mult \ext{f}(zs\,\doubleplus\,x :: y :: ys)$,
to actually prove this in Cubical Agda would involve first applying associativity to prove
$(z :: zs)\,\doubleplus\,x :: y :: ys = z :: (zs\,\doubleplus\,x :: y :: ys)$, before we can actually
apply homomorphism properties of $f$. In the final reassembling step, similarly,
we also need to re-apply associativity to go from $z :: (zs\,\doubleplus\,y :: x :: ys)$
to $(z :: zs)\,\doubleplus\,y :: x :: ys$. Also since we are working with an equivalence relation we
defined ($\Perm$) and not the equality type directly, we cannot exploit the many combinators defined
in the standard library for the equality type and often needing to re-define combinators ourselves.

\subsection{Swap-List}\label{cmon:slist}

Informally, quotients generate too many points, then quotient out into equivalence classes by the congruence relation.
%
Alternately, HITs use generators (points) and higher generators (paths) (and higher higher generators and so on\ldots).
%
We can define free commutative monoids using HITs were adjacent swaps generate all symmetries,
for example swap-list taken from \cite{choudhuryFreeCommutativeMonoids2023}, where they have also given a proof of its
universal property. We include this presentation for completeness, which is also a good presentation in some of our
proofs.

\begin{code}
data SList (A : UU) : UU where
  nil : SList A
  _cons_ : A -> SList A -> SList A
  swap : forall x y xs -> x :: y :: xs == y :: x :: xs
  trunc : forall x y -> (p q : x == y) -> p == q
\end{code}
\vspace{1em}

The $\term{trunc}$ constructor is necessary to truncate $\SList$ down to a set,
thereby ignoring any higher paths intrudoced by the $\term{swap}$ constructor.
This is opposed to $\List$, which does not need a $\term{trunc}$ constructor
because it does not have any path constructors, therefore it can be proven that $\List(A)$
is a set assuming $A$ is a set (see formalization).

\begin{definition}[Concatenation]
    We define the concatenation operation $\doubleplus : \SList(A) \to \SList(A) \to \SList(A)$
    recursively, where we also have to consider the (functorial) action on the $\term{swap}$ path using $\term{ap}$.
    \begin{align*}
        [] \doubleplus ys                                 & = ys                                   \\
        (x :: xs) \doubleplus ys                          & = x :: (xs \doubleplus ys)             \\
        \term{ap}_{\doubleplus ys}(\term{swap}(x, y, xs)) & = \term{swap}(x, y, ys \doubleplus xs)
    \end{align*}
\end{definition}

\cite{choudhuryFreeCommutativeMonoids2023} have already given a proof of $(\SList(A), \doubleplus, [])$ satisfying
commutative monoid laws. We explain the proof of $\doubleplus$ satisfying commutativity here.

\begin{lemma}[Head rearrange]\label{slist:cons}
    For all $x : A$, $xs : \SList(A)$, $x :: xs = xs \doubleplus [ x ]$.
\end{lemma}

\begin{proof}
    We can prove this by induction on $xs$.
    For $xs = []$ this is trivial. For $xs = y :: ys$, we have the induction hypothesis $x :: ys = ys \doubleplus [ x ]$.
    By applying $y :: \blank$ on both side and then apply $\term{swap}$, we can complete the proof.
\end{proof}

\begin{theorem}[Commutativity]\label{slist:comm}
    For all $xs,\,ys : \SList(A)$, $xs \doubleplus ys = ys \doubleplus xs$.
\end{theorem}

\begin{proof}
    By induction on $xs$ we can iteratively apply~\cref{slist:cons} to move all elements of $xs$
    to after $ys$. This would move $ys$ to the head and $xs$ to the end, thereby proving
    $xs \doubleplus ys = ys \doubleplus xs$.
\end{proof}

\subsubsection*{Remark on representation}\label{slist:rep}
Much like $\PList$ and $\List$, $\SList$ is inductively defined, therefore making it very intuitive to reason
with when defining inductive operations or inductive proofs on $\SList$, however difficult to reason with
when defining operations that involve arbitrary partitioning, for reasons similar to those given
in~\cref{plist:rep}.

Unlike $\PList$ which is defined as a set quotient, this is defined as a HIT, therefore handling equalities
between $\SList$ is much simpler than $\PList$. We would still need to prove a function $f$ respects
the path constructor of $\SList$ when pattern matching, i.e. $f(x :: y :: xs) = f(y :: x :: xs)$.
Unlike $\PList$ we do not need to worry about as many auxiliary variables since swap
only happens at the head of the list, whereas with $\PList$ we would need to prove
$f(xs\,\doubleplus\,x :: y :: ys) = f(xs\,\doubleplus\,y :: x :: ys)$. One may be tempted to just remove $xs$
from the $\term{perm-swap}$ constructor such that it becomes
$\term{perm-swap} : \forall x\,y\,ys\,zs \to \Perm\,(x :: y :: ys)\,zs \to \Perm\,(y :: x :: ys)\,zs$.
However this would break $\Perm$'s congruence wrt to $\doubleplus$, therefore violating the axioms of
permutation relations. Also, since we are working with the identity type directly, properties we would
expect from $\term{swap}$, such as reflexivity, transitivity, symmetry, congruence and such all comes directly by
construction, whereas with $\Perm$ we would have to prove these properties manually.
We can also use the many combinatorics defined in the standard library for equational reasoning,
making the handling of $\SList$ equalities a lot simpler.

\subsection{Bag}\label{cmon:bag}
Alternatively, we can also quotient $\Array$ with symmetries to get commutativity.
%
This construction is first considered in~\cite{altenkirchDefinableQuotientsType2011}
and~\cite{liQuotientTypesType2015}, then partially considered in~\cite{choudhuryFreeCommutativeMonoids2023},
and also in~\cite{joramConstructiveFinalSemantics2023},
who gave a similar construction, where only the index function is quotiented, instead of
the entire array.
%
\cite{danielssonBagEquivalenceProofRelevant2012} also considered $\Bag$ as a setoid relation
on $\List$ in an intensional MLTT setting.
%
\cite{joramConstructiveFinalSemantics2023} prove that their version of $\Bag$ is the free commutative monoid by
equivalence to the other HIT constructions.
%
We give a direct proof of its universal property instead, using the technology we have developed.

\begin{definition}[Bag]
    \label{def:bag}
    \leavevmode
    \begin{code}
_$\approx$_ : Array A $\to$ Array A $\to$ UU
(n , f) $\approx$ (m , g) = Sg(sig : Fin n eqv Fin m) v = w comp sig

Bag : UU -> UU
Bag A = Array A / _$\approx$_
    \end{code}
\end{definition}

Note that by the pigeonhole principle, $\sigma$ can only be constructed when $n = m$, though this requires a proof in
type theory (see the formalization).
%
Conceptually, we are quotienting array by a permutation or an automorphism on the indices.

We have already given a proof of $\Array$ being the free monoid in~\cref{mon:array}.
%
By~\cref{cmon:qfreemon} it suffices to show $\approx$ satisfies the axioms of permutation relations to show that $\Bag$
is the free commutative monoid.

\begin{proposition}
    $\approx$ is a equivalence relation.
\end{proposition}

\begin{proof}
    We can show any array $xs$ is related to itself by the identity isomorphism, therefore $\approx$ is reflexive.
    If $xs \approx ys$ by $\sigma$, we can show $ys \approx xs$ by $\sigma^{-1}$, therefore $\approx$ is symmetric.
    If $xs \approx ys$ by $\sigma$ and $ys \approx zs$ by $\phi$, we can show $xs \approx zs$ by $\sigma \comp \phi$,
    therefore $\approx$ is transitive.
\end{proof}

\begin{proposition}\label{bag:cong}
    $\approx$ is congruent wrt to $\doubleplus$.
\end{proposition}

\begin{proof}
    Given $(n, f) \approx (m, g)$ by $\sigma$ and $(u, p) \approx (v, q)$ by $\phi$,
    we want to show $(n, f) \doubleplus (u, p) \approx (m, g) \doubleplus (v, q)$ by some $\tau$.
    We can construct $\tau$ as follow:

    \begin{align*}
        \tau := \Fin[n+u] \xto{\sim} \Fin[n] + \Fin[u] \xto{\sigma,\,\phi} \Fin[m] + \Fin[v] \xto{\sim} \Fin[m+v] \\
    \end{align*}

    \begin{figure}[H]
        \centering
        % https://q.uiver.app/#q=WzAsMixbMCwwLCJcXHswLDEsXFxkb3RzLG4tMSwgbixuKzEsXFxkb3RzLG4rbS0xXFx9Il0sWzAsMSwiIFxce24sbisxXFxkb3RzLG4rbS0xLDAsMSxcXGRvdHMsbi0xXFx9Il0sWzAsMSwiIiwwLHsic3R5bGUiOnsidGFpbCI6eyJuYW1lIjoibWFwcyB0byJ9fX1dXQ==
        \begin{tikzcd}[ampersand replacement=\&,cramped]
            {\{\color{red}0,1,\dots,n-1, \color{blue} n,n+1,\dots,n+u-1 \color{black}\}} \\
            { \{\color{red}\sigma(0),\sigma(1)\dots,\sigma(n-1), \color{blue}\phi(0),\phi(1),\dots,\phi(u-1) \color{black}\}}
            \arrow["{\sigma,\phi}", maps to, from=1-1, to=2-1]
        \end{tikzcd}
        \caption{Operation of $\tau$}
        \label{fig:enter-label}
    \end{figure}

\end{proof}

\begin{proposition}\label{bag:comm}
    $\approx$ is commutative.
\end{proposition}

\begin{proof}
    We want to show for any arrays $(n, f)$ and $(m, g)$, $(n, f) \mult (m, g) \approx (m, g) \mult (n, f)$
    by some $\phi$.

    We can use combinators from formal operations in symmetric rig groupoids \cite{choudhurySymmetriesReversibleProgramming2022} to define $\phi$:
    \begin{align*}
        \phi := \Fin[n+m] \xto{\sim} \Fin[n] + \Fin[m] \xto{\term{swap}_{+}} \Fin[m] + \Fin[n] \xto{\sim} \Fin[m+n] \\
    \end{align*}

    \begin{figure}[H]
        \centering
        % https://q.uiver.app/#q=WzAsMixbMCwwLCJcXHswLDEsXFxkb3RzLG4tMSwgbixuKzEsXFxkb3RzLG4rbS0xXFx9Il0sWzAsMSwiIFxce24sbisxXFxkb3RzLG4rbS0xLDAsMSxcXGRvdHMsbi0xXFx9Il0sWzAsMSwiIiwwLHsic3R5bGUiOnsidGFpbCI6eyJuYW1lIjoibWFwcyB0byJ9fX1dXQ==
        \begin{tikzcd}[ampersand replacement=\&,cramped]
            {\{\color{red}0,1,\dots,n-1, \color{blue} n,n+1,\dots,n+m-1 \color{black}\}} \\
            { \{\color{blue}n,n+1\dots,n+m-1, \color{red}0,1,\dots,n-1 \color{black}\}}
            \arrow["\phi", maps to, from=1-1, to=2-1]
        \end{tikzcd}
        \caption{Operation of $\phi$}
        \label{fig:enter-label}
    \end{figure}

\end{proof}

\begin{proposition}
    $\approx$ respects $\ext{(\blank)}$ for arrays.
\end{proposition}

It suffices to show that $\ext{f}$ is invariant under permutation: for all $\phi\colon \Fin[n]\xto{\sim}\Fin[n]$,
$\ext{f}(n, i) \id \ext{f}(n, i \circ \phi)$. To prove this we first need to develop some formal combinatorics of
\emph{punching in} and \emph{punching out} indices. These operations are developed further
in~\cite{choudhurySymmetriesReversibleProgramming2022} for studying permutation codes.

\begin{lemma}\label{bag:tau}
    Given $\phi\colon \Fin[S(n)]\xto{\sim}\Fin[S(n)]$, there is a $\tau\colon \Fin[S(n)]\xto{\sim}\Fin[S(n)]$
    such that $\tau(0) = 0$, and $\ext{f}(S(n), i \circ \phi) = \ext{f}(S(n), i \circ \tau)$.
\end{lemma}

\begin{proof}
    Let $k$ be $\phi^{-1}(0)$, and $k + j = S(n)$, we can construct $\tau$ as follow:
    \begin{align*}
        \tau := \Fin[S(n)] \xto{\phi} \Fin[S(n)] \xto{\sim} \Fin[k+j] \xto{\sim} \Fin[k] + \Fin[j]
        \xto{\term{swap}_{+}} \Fin[j] + \Fin[k] \xto{\sim} \Fin[j+k] \xto{\sim} \Fin[S(n)]
    \end{align*}

    \begin{figure}[H]
        \centering
        \begin{tikzcd}[ampersand replacement=\&,cramped]
            {\{\color{blue}0, 1, 2, \dots, \color{red}k, k+1, k+2, \dots \color{black}\}} \\
            { \{\color{blue}x, y, z, \dots, \color{red}0, u, v, \dots \color{black}\}}
            \arrow["\phi", maps to, from=1-1, to=2-1]
        \end{tikzcd}
        \hspace{1em}
        \begin{tikzcd}[ampersand replacement=\&,cramped]
            {\{\color{blue}0, 1, 2, \dots, \color{red}k, k+1, k+2, \dots \color{black}\}} \\
            { \{\color{red}0, u, v, \dots, \color{blue}x, y, z, \dots \color{black}\}}
            \arrow["\tau", maps to, from=1-1, to=2-1]
        \end{tikzcd}
        \caption{Operation of $\tau$ constructed from $\phi$}
        \label{fig:enter-label}
    \end{figure}

    It is trivial to prove $\ext{f}(S(n), i \circ \phi) = \ext{f}(S(n), i \circ \tau)$ since the only
    operation on indices in $\tau$ is $\term{swap}_{+}$. It suffices to show $(S(n), i \circ \phi)$
    can be decomposed into two arrays such that $(S(n), i \circ \phi) = (k, g) \doubleplus (j, h)$
    for some $g$ and $h$. Since the image of $\ext{f}$ is a commutative monoid and $\ext{f}$ is a homomorphism,
    $\ext{f}((k, g) \doubleplus (j, h)) = \ext{f}(k, g) \mult \ext{f}(j, h) = \ext{f}(j, h) \mult \ext{f}(k, g) =
        \ext{f}((j, h) \doubleplus (k, g))$, thereby proving $\ext{f}(S(n), i \circ \phi) = \ext{f}(S(n), i \circ \tau)$.

\end{proof}

\begin{lemma}\label{bag:punch}
    Given $\tau\colon \Fin[S(n)]\xto{\sim}\Fin[S(n)]$ where $\tau(0) = 0$,
    there is a $\psi : \Fin[n] \xto{\sim} \Fin[n]$ such that $\tau \circ S = S \circ \psi$.
\end{lemma}

\begin{proof}
    We can construct $\psi$ by $\psi(x) = \tau(S(x)) - 1$.
    %
    Since $\tau$ maps only 0 to 0 by assumption, $\forall x. \, \tau(S(x)) > 0$, therefore
    the $(- 1)$ is well defined. This is the special case for $k = 0$ in the punch-in and punch-out
    equivalence for Lehmer codes in~\cite{choudhurySymmetriesReversibleProgramming2022}.

    \begin{figure}[H]
        \centering
        \begin{tikzcd}[ampersand replacement=\&,cramped]
            {\{\color{blue}0, \color{red}1, 2, 3, \dots \color{black}\}} \\
            { \{\color{blue}0, \color{red} x, y, z \dots \color{black}\}}
            \arrow["\tau", maps to, from=1-1, to=2-1]
        \end{tikzcd}
        \hspace{1em}
        \begin{tikzcd}[ampersand replacement=\&,cramped]
            {\{\color{red}0, 1, 2, \dots \color{black}\}} \\
            { \{\color{red} x-1, y-1, z-1 \dots \color{black}\}}
            \arrow["\psi", maps to, from=1-1, to=2-1]
        \end{tikzcd}
        \caption{Operation of $\psi$ constructed from $\tau$}
        \label{fig:enter-label}
    \end{figure}
\end{proof}

We now prove our main theorem.

\begin{theorem}[Permutation invariance]\label{bag:perm-sat}
    For all $\phi\colon \Fin[n]\xto{\sim}\Fin[n]$, $\ext{f}(n, i) \id \ext{f}(n, i \circ \phi)$.
\end{theorem}

\begin{proof}
    We can prove this by induction on $n$.
    \begin{itemize}
        \item On $n = 0$, $\ext{f}(0, i) \id \ext{f}(0, i \circ \phi) = e$.
        \item On $n = S(m)$,
              \begin{align*}
                   & \ext{f}(S(m), i \circ \phi)                                                                       \\
                   & = \ext{f}(S(m), i \circ \tau)                          & \text{by~\cref{bag:tau}}                 \\
                   & = f(i(\tau(0))) \mult \ext{f}(m, i \circ \tau \circ S) & \text{by definition of $\ext{(\blank)}$} \\
                   & = f(i(0)) \mult \ext{f}(m, i \circ \tau \circ S)       & \text{by construction of $\tau$}         \\
                   & = f(i(0)) \mult \ext{f}(m, i \circ S \circ \psi)       & \text{by~\cref{bag:punch}}               \\
                   & = f(i(0)) \mult \ext{f}(m, i \circ S)                  & \text{induction}                         \\
                   & = \ext{f}(S(m), i)                                     & \text{by definition of $\ext{(\blank)}$}
              \end{align*}
    \end{itemize}
\end{proof}

\subsubsection*{Remark on representation}\label{bag:rep}

Unlike $\PList$ and $\SList$, $\Bag$ and its underlying construction $\Array$ are not inductively defined,
making it difficult to work with when trying to do induction on them. For example,
in the proof~\cref{array:univ}, two lemmas~\cref{array:eta-suc} and~\cref{array:split} are needed to do
induction on $\Array$, as opposed to $\List$ and its quotients, where we can do induction simply by
pattern matching. Much like $\PList$, when defining functions on $\Bag$, we need to show they respect
$\approx$, i.e. $as \approx bs \to f(as) = f(bs)$. This is notably much more difficult than
$\PList$ and $\SList$, because whereas with $\PList$ and $\SList$ we only need to consider "small permutations"
(i.e. swapping adjacent elements), with $\Bag$ we need to consider all possible permutations. For example,
in the proof of~\cref{bag:perm-sat}, we need to first construct a $\tau$ which satisfies $\tau(0) = 0$ and prove
$\ext{f}(n, i \comp \sigma) = \ext{f}(n, i \comp \tau)$ before we can apply induction.

Also, on a more technical note, since $\Array$ and $\Bag$ are not simple data types, the definition of
the monoid operation on them $\doubleplus$ are necessarily more complicated, and unlike $\List$, $\PList$
and $\SList$, constructions of $\Array$ and $\Bag$ via $\doubleplus$ often would not normalize into a
very simple form, but would instead expand into giant trees of terms. This makes it such that when working
with $\Array$ and $\Bag$ we need to be very careful or otherwise Agda would be stuck trying to display
the normalized form of $\Array$ and $\Bag$ in the goal and context menu. Type-checking also becomes a lengthy
process that tests if the user possesses the virtue of patience.

However, performing arbitrary partitioning with $\Array$ and $\Bag$ is much easier than
$\List$, $\SList$, $\PList$. For example,
one can simply use the combinator $\Fin[n+m] \xto{\sim} \Fin[n] + \Fin[m]$ to partition the array,
then perform operations on the partitions such as swapping in~\cref{bag:comm},
or perform operations on the partitions individually such as two individual permutation in~\cref{bag:cong}.
This makes it such that when defining divide-and-conquer algorithms such as merge sort,
$\Bag$ and $\Array$ are more natural to work with than $\List$, $\SList$, and $\PList$.
 %% 4 pages
\section{Combinatorics}
\label{sec:combinatorics}

With universal properties we can structure our programs using ideas from algebra.
For example $\term{length}$ is a common operation defined inductively for $\List$,
but to state properties about $\term{length}$, e.g.
$\term{length}(xs \doubleplus ys) = \term{length}(xs) + \term{length}(ys)$,
we would need to prove them externally. With our framework we can define operations like $\term{length}$
using the $\ext{(\blank)}$ operation, which would also give us a proof that they are homomorphisms for free.
We can also use universal property to prove two different types are equal by showing they both
satisfy the same universal property, which is desirable especially if proving a direct equivalence between
the two types is very difficult.

To illustrate this, we give some examples of how some common operations can be defined
in terms of universal property.
We use $\FF(A)$ to denote the free monoid or free commutative monoid on $A$.

\subsection{Length}

Any presentation of free monoids or free commutative monoids has a $\term{length} : \FF(A) \to \Nat$ function.
$\Nat$ is a (commutative) monoid with $(0,+)$, so we can extend $\lambda x.\, 1\,:\,A \to \Nat$.
This allows us to define $\term{length}$ for any construction of free (commutative) monoid, and also
gives us a proof of $\term{length}$ being a homomorphism for free.

\begin{figure}[H]
    \centering
    \begin{tikzcd}[ampersand replacement=\&,cramped]
    	{\FF(A)} \&\& {(\mathbb{N}, 0, +)} \\
    	\\
    	A
    	\arrow["\eta_A", from=3-1, to=1-1]
    	\arrow["{\ext{(\lambda x. \, 1)}}", from=1-1, to=1-3]
    	\arrow["{\lambda x. \, 1}"', from=3-1, to=1-3]
    \end{tikzcd}
    \caption{Definition of $\term{length}$ by universal property}
    \label{fig:enter-label}
\end{figure}


\subsection{Membership}
Any presentation of free monoids or free commutative monoids has a membership predicate:
$\_\in\_ : A \to \FF(A) \to \hProp$. By fixing an element $x: A$, the element we want to define
the membership proof for, and 
assuming $A$ is a set, we can define $\yo_A(x) = \lambda y.\, x \id y : A \to \hProp$.
Since $\hProp$ forms a (commutative) monoid under $\vee$,
we can extend $\yo_A(x)$ to obtain $x \in \blank : \FF(A) \to \hProp$, giving us the membership predicate for $x$.

\begin{figure}[H]
    \centering
    % https://q.uiver.app/#q=WzAsNCxbMCwwLCJcXEZGKEEpIl0sWzIsMF0sWzAsMiwiQSJdLFszLDAsIihcXGhQcm9wLCBcXGJvdCwgXFx2ZWUpIl0sWzIsMCwiXFxldGFfQSJdLFswLDMsIih4IFxcaW4gXFxibGFuaykgLyBcXGV4dHtcXHlvX0EoeCl9Il0sWzIsMywiXFx5b19BKHgpIiwyXV0=
    \begin{tikzcd}[ampersand replacement=\&,cramped]
    	{\FF(A)} \&\& {} \& {(\hProp, \bot, \vee)} \\
    	\\
    	A
    	\arrow["{\eta_A}", from=3-1, to=1-1]
    	\arrow["{(x \in \blank) / \ext{\yo_A(x)}}", from=1-1, to=1-4]
    	\arrow["{\yo_A(x)}"', from=3-1, to=1-4]
    \end{tikzcd}
    \caption{Definition of membership proof for $x$ by universal property}
    \label{fig:enter-label}
\end{figure}


\subsection{Any and All predicate}

Any presentation of free monoids or free commutative monoids $\FF(A)$ has the
$\term{Any}$ and $\term{All}$ predicates, which allow us to lift a predicate $A \to \hProp$
to any or all elements of $xs : \FF(A)$. We note that
$\hProp$ forms a (commutative) monoid in two different ways: $(\bot,\vee)$ and $(\top,\wedge)$,
therefore given a predicate $P : A \to \hProp$, we get:
\begin{align*}
    Any(P) &= \ext{P} : \FF(A) \to (\hProp, \bot, \vee) \\
    All(P) &= \ext{P} : \FF(A) \to (\hProp, \top, \wedge)
\end{align*}

\begin{figure}[H]
    \centering
    \begin{minipage}[t]{0.49\textwidth}
        \centering
        \begin{tikzcd}[ampersand replacement=\&,cramped]
        	{\FF(A)} \&\& {(\hProp, \top, \wedge)} \\
        	\\
        	A
        	\arrow["\eta_A", from=3-1, to=1-1]
        	\arrow["{\ext{P}}", from=1-1, to=1-3]
        	\arrow["{P}"', from=3-1, to=1-3]
        \end{tikzcd}
        \caption{Definition of $\term{All}$ by universal property}
        \label{fig:enter-label}
    \end{minipage}
    \begin{minipage}[t]{0.49\textwidth}
        \centering
        \begin{tikzcd}[ampersand replacement=\&,cramped]
        	{\FF(A)} \&\& {(\hProp, \bot, \vee)} \\
        	\\
        	A
        	\arrow["\eta_A", from=3-1, to=1-1]
        	\arrow["{\ext{P}}", from=1-1, to=1-3]
        	\arrow["{P}"', from=3-1, to=1-3]
        \end{tikzcd}
        \caption{Definition of $\term{Any}$ by universal property}
        \label{fig:enter-label}
    \end{minipage}
\end{figure}

% We denote free monoid on $A$ with $\LL(A)$ and free commutative monoid on $A$ with $\MM(A)$.
% $\MM(A + B) \eqv \MM(A) \times \MM(B)$.


 %% 2 pages
% !TEX root = cpp25-sort.tex

\subsection{Total orders}
\label{sec:total-orders}

First, we recall the axioms of a total order $\leq$ on a set $A$.
\begin{definition}[Total order]
    \label{def:total-order}
    A total order on a set $A$ is a relation $\leq : A \to A \to \hProp$ that satisfies:
    \begin{itemize}
        \item reflexivity: $x \leq x$,
        \item transitivity: if $x \leq y$ and $y \leq z$, then $x \leq z$,
        \item antisymmetry: if $x \leq y$ and $y \leq x$, then $x = y$,
        \item strong-connectedness: $\forall x, y$, either $x \leq y$ or $y \leq x$.
    \end{itemize}
    Note that \emph{either-or} means that this is a (truncated) logical disjunction.
    In the context of this paper, we want to make a distinction between ``decidable total order''
    and ``total order''. A \emph{decidable} total order requires the $\leq$ relation to be decidable:
    \begin{itemize}
        \item decidability: $\forall x, y$, we have $x \leq y + \neg(x \leq y)$.
    \end{itemize}
\end{definition}
This strengthens the strong-connectedness axiom,
where we have either $x \leq y$ or $y \leq x$ merely as a proposition,
but decidability allows us to actually compute if $x \leq y$ is true.
\begin{proposition}
    \label{prop:decidable-total-order}
    In a decidable total order, it holds that ${\forall x, y}, \ps{x \leq y} + \ps{y \leq x}$.
    Further, this makes $A$ discrete, that is ${\forall x, y}, \ps{x \id y} + \ps{x \neq y}$.
\end{proposition}
%
An equivalent way to define a total order is using a binary meet operation.
\begin{definition}[Meet semi-lattice]
    \label{def:meet-semi-lattice}
    A meet semi-lattice is a set $A$ with a binary operation $\blank\meet\blank : A \to A \to A$ that is:
    \begin{itemize}
        \item idempotent: $x \meet x \id x$,
        \item associative: $(x \meet y) \meet z \id x \meet (y \meet z)$,
        \item commutative: $x \meet y \id y \meet x$.
    \end{itemize}
    A \emph{strongly-connected} meet semi-lattice further satisfies:
    \begin{itemize}
        \item strong-connectedness: $\forall x, y$, either $x \meet y \id x$ or $x \meet y \id y$.
    \end{itemize}
    A \emph{total} meet semi-lattice strengthens this to:
    \begin{itemize}
        \item totality: ${\forall x, y}, \ps{x \meet y \id x} + \ps{x \meet y \id y}$.
    \end{itemize}
\end{definition}

\begin{proposition}
    \label{prop:total-order-meet-semi-lattice}
    A total order $\leq$ on a set $A$ is equivalent to a strongly-connected meet semi-lattice structure on $A$.
    Further, a decidable total order on $A$ induces a total meet semi-lattice structure on $A$.
\end{proposition}
\begin{proofsketch}
    Given a (mere) total order $\leq$ on a set $A$,
    we define ${x \meet y} \defeq \term{if} x \leq y \term{then} x \term{else} y$.
    %
    Crucially, this map is \emph{locally-constant}, allowing us to eliminate from an $\hProp$ to an $\hSet$.
    %
    Meets satisfy the universal property of products, that is,
    ${c \leq a \meet b} \Leftrightarrow {c \leq a} \land {c \leq b}$,
    and the axioms follow by calculation using $\yo$-arguments.
    %
    Conversely, given a meet semi-lattice, we define $x \leq y \defeq x \meet y \id x$,
    which defines an $\hProp$-valued total ordering relation.
    %
    If the total order is decidable, we use the discreteness of $A$ from~\cref{prop:decidable-total-order}.
\end{proofsketch}

Finally, tying this back to~\cref{def:head-free-monoid}, we have the following for free commutative monoids.
\begin{definition}[$\term{head}$]
    \label{def:head-free-commutative-monoid}
    Assume a total order $\leq$ on a set $A$.
    We define a commutative monoid structure on $1 + A$,
    with unit \(e \defeq \inl(\ttt) : 1 + A\), and multiplication defined as:
    \[
        \begin{array}{rclcl}
            \inl(\ttt) & \oplus & b          & \defeq & b                         \\
            \inr(a)    & \oplus & \inl(\ttt) & \defeq & \inr(a)                   \\
            \inr(a)    & \oplus & \inr(b)    & \defeq & \inr(a \meet b) \enspace.
        \end{array}
    \]
    This gives a homomorphism \({\term{head} \defeq \ext{\inr}} : {\MM(A) \to 1 + A}\),
    which picks out the \emph{least} element of the free commutative monoid.
\end{definition}

\subsection{Sorting functions}
\label{sec:sorting}

The free commutative monoid is also a monoid, hence, there is a canonical monoid homomorphism
$q : \LL(A) \to \MM(A)$, which is given by $\ext{\eta_A}$.
%
Since $\MM(A)$ is (upto equivalence), a quotient of $\LL(A)$ by symmetries (or a permutation relation),
it is a surjection (in particular, a regular epimorphism in $\Set$ as constructed in type theory).
%
Concretely, $q$ simply includes the elements of $\LL(A)$ into equivalence classes of lists in $\MM(A)$,
which ``forgets'' the order that was imposed by the indexing of the list.

Classically, assuming the Axiom of Choice would allow us to construct a section (right-inverse) to the surjection $q$,
that is,
a function $s : \MM(A) \to \LL(A)$ such that $\forall x.\, q(s(x)) \id x$.
%
Or in informal terms, given the surjective inclusion into the quotient,
a section (uniformly) picks out a canonical representative for each equivalence class.
%
Constructively, does $q$ have a section? If symmetry kills the order, can it be resurrected?
\begin{figure}[H]
    \centering
    \scalebox{1.0}{
        % https://q.uiver.app/#q=WzAsMixbMCwwLCJcXExMKEEpIl0sWzMsMCwiXFxNTShBKSJdLFsxLDAsInMiLDAseyJjdXJ2ZSI6LTF9XSxbMCwxLCJxIiwwLHsiY3VydmUiOi0xfV1d
        \begin{tikzcd}[ampersand replacement=\&,cramped]
            {\LL(A)} \&\&\& {\MM(A)}
            \arrow["s", curve={height=-10pt}, from=1-4, to=1-1]
            \arrow["q", two heads, from=1-1, to=1-4]
        \end{tikzcd}
    }
    \caption{Relationship of $\LL(A)$ and $\MM(A)$}
    \label{fig:enter-label}
\end{figure}

Viewing the quotienting relation as a permutation relation (from~\cref{cmon:qfreemon}), a section $s$ to $q$ has to pick out
canonical representatives of equivalence classes generated by permutations.
%
Using $\SList$ as an example, $s(x \cons y \cons xs) \id s(y \cons x \cons xs)$ for any $x, y : A$ and $xs : \SList(A)$,
and since it must also respect $\forall xs.\,q(s(xs)) \id xs$, $s$ must preserve all the elements of $xs$.
It cannot be a trivial function such as $\lambda\,xs. []$ -- it must produce a permutation of the elements of $s$!
%
But to place these elements side-by-side in the list, $s$ must somehow impose an order on $A$
(invariant under permutation), turning unordered lists of $A$ into ordered lists of $A$.
%
Axiom of Choice (AC) giving us a section $s$ to $q$ ``for free'' is analagous to how
AC implies the well-ordering principle, which states every set can be well-ordered.
%
If we assumed AC our problem would be trivial!
%
Instead we study how to constructively define such a section, and in fact,
that is exactly the extensional view of a sorting algorithm,
and the implications of its existence is that $A$ can be ordered, or sorted.

\subsubsection{Section from Order}

\begin{proposition}
    Assume a decidable total order on $A$. There is a sort function $s: \MM(A) \to \LL(A)$
    which constructs a section to $q : \LL(A) \twoheadrightarrow \MM(A)$
\end{proposition}

\begin{proofsketch}
    We can construct such a sor functiont by implementing any sorting algorithm.
    In our formalization we chose insertion sort,
    because it can be defined easily using the inductive structure of $\SList(A)$ and $\List(A)$.
    To implement other sorting algorithms like mergesort,
    other representations such as $\Bag$ and $\Array$ would be preferable, as explained in~\cref{bag:rep}.
    To see how this proposition holds: $q(s(xs))$ orders an unordered list $xs$ by $s$,
    and discards the order again by $q$ --
    imposing and then forgetting an order on $xs$ simply \emph{permutes} its elements,
    which proves $q \comp s \htpy \idfunc$.
\end{proofsketch}

\redtext{This has been done before\ldots definable quotients\ldots we want to go the other way.}

\subsubsection{Order from Section}

The previous section allowed us to construct a section -- how do we know this is a \emph{correct} sort function?
%
At this point we ask: if we can construct a section from order, can we construct an order from section?
%
Indeed, just by the virtue of $s$ being a section,
we can (almost) construct a total-ordering relation on the carrier set!

\begin{definition}
    \label{def:least}
    Given a section $s$, we define:
    \[
        \begin{aligned}
            \term{least}(xs) & \defeq \term{head}(s(xs))                           \\
            x \leqs y        & \defeq \term{least}(\bag{x, y}) = \inr(x) \enspace.
        \end{aligned}
    \]
\end{definition}
%
That is, we take the two-element bag $\bag{x, y}$,
``sort'' it by $s$, and test if the $\term{head}$ element is $x$.
%
Note, this is equivalent to $x \leqs y \defeq s\bag{x, y} = [x,y]$,
because $s$ preserves length, and the second element is forced to be $y$.
%
% \begin{proposition}
%     $\leqs$ is decidable iff $A$ has decidable equality.
% \end{proposition}

\begin{proposition}
    \label{sort:almost-total}
    $\leqs$ is reflexive, antisymmetric, and total.
\end{proposition}
\begin{proof}
    For all $x$, $\term{least}(\bag{x, x})$ must be $\inr(x)$, therefore $x \leqs x$, giving reflexivity.
    For all $x$ and $y$, given $x \leqs y$ and $y \leqs x$,
    we have $\term{least}(\bag{x, y}) = \inr(x)$ and $\term{least}(\bag{y, x}) = \inr(y)$.
    Since $\bag{x, y} = \bag{y, x}$, $\term{least}(\bag{x, y}) = \term{least}(\bag{y, x})$,
    therefore we have $x = y$, giving antisymmetry.
    For all $x$ and $y$, $\term{least}(\bag{x, y})$ is merely either $\inr(x)$ or $\inr(y)$,
    therefore we have merely either $x \leqs y$ or $y \leqs x$, giving totality.
\end{proof}

Although $s$ correctly orders 2-element bags, it doesn't necessarily sort 3 or more elements --
$\leqs$ is not necessarily transitive (a counterexample is given in~\cref{prop:counterexample-transitivity}).
%
We will enforce this by imposing additional constraints on the \emph{image} of $s$.

\begin{toappendix}
    \begin{proposition}
        \label{prop:counterexample-transitivity}
        $\leqs$ is not necessarily transitive.
    \end{proposition}
    \begin{proof}
        We give a counter-example of $s$ that would violate transitivity.
        Consider this section $s : \SList(\Nat) \to \List(\Nat)$, we can define a sort function
        $\term{sort} : \SList(\Nat) \to \List(\Nat)$ which sorts $\SList(\Nat)$ ascendingly. We can use $\term{sort}$
        to construct $s$.
        \begin{align*}
            s(xs)        & = \begin{cases}
                                 \term{sort}(xs)                 & \text{if $\term{length}(xs)$ is odd} \\
                                 \term{reverse}(\term{sort}(xs)) & \text{otherwise}
                             \end{cases} \\
            s([2,3,1,4]) & = [4,3,2,1]                                                                     \\
            s([2,3,1])   & = [1,2,3]
        \end{align*}
    \end{proof}
\end{toappendix}

\begin{definition}[$\blank\in\im{s}$]
    \label{def:in-image}
    The fiber of $s$ over~$xs : \LL(A)$ is given by $\fib_{s}(xs) \defeq \dsum{ys : \MM(A)}{s(ys) = xs}$.
    %
    The image of $s$ is given by $\im{s} \defeq \dsum{xs : \LL(A)}{\Trunc[-1]{\fib_{s}(xs)}}$.
    %
    Simplifying, we say that $xs:\LL(A)$ is ``in the image of $s$'', or, $xs \in \im{s}$,
    if there merely exists a $ys:\MM(A)$ such that $s(ys) = xs$.
\end{definition}

If $s$ \emph{were} a sort function, the image of $s$ would be the set of $s$-``sorted'' lists,
therefore $\inimage{xs}$ would imply $xs$ is a correctly $s$-``sorted'' list.
%
First, we note that the 2-element case is correct.
%
\begin{proposition}
    \label{sort:sort-to-order}
    $x \leqs y$ \; iff \; $\inimage{[x, y]}$.
\end{proposition}
%
\noindent Then, we state the first axiom on $s$.
\begin{definition}[$\isheadleast$]
    \label{sort:head-least}
    A section $s$ satisfies $\isheadleast$ iff for all $x, y, xs$:
    \[
        y \in x \cons xs \; \land \; \inimage{x \cons xs} \; \to \; \inimage{[x, y]}
        \enspace.
    \]
\end{definition}
\noindent
We use the definition of list membership from~\cref{def:membership}.
The $\in$ symbol is intentionally overloaded
to make the axiom look like a logical ``cut'' rule.
Inforamlly, it says that the head of an $s$-``sorted'' list
is the least element of the list.

\begin{proposition}
    \label{prop:order-to-sort-head-least}
    If $A$ has a total order $\leq$,
    insertion sort defined using $\leq$ satisfies $\isheadleast$.
\end{proposition}

\begin{proposition}
    \label{sort:trans}
    If $s$ satisfies $\isheadleast$, $\leqs$ is transitive.
\end{proposition}
\begin{proof}
    Given $x \leqs y$ and $y \leqs z$, we want to show $x \leqs z$.
    Consider the 3-element bag $\bag{x,y,z} : \MM(A)$.
    %
    Let $u$ be $\term{least}(\bag{x,y,z})$,
    by~\cref{sort:head-least} and~\cref{sort:sort-to-order},
    we have $u \leqs x \land u \leqs y \land u \leqs z$.
    %
    Since $u \in \bag{x,y,z}$, $u$ must be one of the elements.
    %
    If $u = x$ we have $x \leqs z$.
    If $u = y$ we have $y \leqs x$,
    and since $x \leqs y$ and $y \leqs z$ by assumption,
    we have $x = y$ by antisymmetry, and then we have $x \leqs z$ by substitution.
    Finally, if $u = z$, we have $z \leqs y$, and since $y \leqs z$ and $x \leqs y$ by assumption,
    we have $z = y$ by antisymmetry, and then we have $x \leqs z$ by substitution.
\end{proof}

\subsubsection{Embedding orders into sections}

Following from \cref{sort:almost-total,sort:trans},
and \cref{prop:order-to-sort-head-least},
we have shown that a section $s$ that satisfies $\isheadleast$ produces a total order
$x \leqs y \defeq \term{least}(\bag{x, y}) \id \inr(x)$,
and a total order $\leq$ on the carrier set produces a section satisfying $\isheadleast$,
constructed by sorting with $\leq$.
%
This constitutes an embedding of decidable total orders into sections satisfying $\isheadleast$.

\begin{proposition}\label{sort:o2s2o}
    Assume $A$ has a decidable total order $\leq$, we can construct a section $s$ that
    satisfies $\isheadleast$, such that $\leqs$ constructed from $s$ is equivalent
    to $\leq$.
\end{proposition}
\begin{proof}
    By the insertion sort algorithm parameterized by $\leq$,
    it holds that $\inimage{[x, y]}$ iff $x \leq y$.
    By~\cref{sort:sort-to-order}, we have $x \leqs y$ iff $x \leq y$.
    We now have a total order $x \leqs y$ equivalent to $x \leq y$.
\end{proof}

\subsubsection{Equivalence of order and sections}

We want to upgrade the embedding to an isomorphism, and it
remains to show that we can turn a section satisfying $\isheadleast$ to a total order $\leqs$,
then construct the \emph{same} section back from $\leqs$.
%
Unfortunately, this fails (see~\cref{prop:counterexample-equivalence})!
%
We then introduce our second axiom of sorting.

\begin{toappendix}
    \begin{proposition}
        \label{prop:counterexample-equivalence}
        Assume $A$ is a set with different elements, i.e. $\exists x, y: A.\,x \neq y$,
        we cannot construct a full equivalence between sections that satisfy $\isheadleast$
        and total orders on $A$.
    \end{proposition}
    \begin{proof}
        We give a counter-example of $s$ that satisfy $\isheadleast$ but is not a sort function.
        Consider the insertion sort function $\term{sort} : \MM(\Nat) \to \LL(\Nat)$
        parameterized by $\leq$:
        \begin{align*}
            \term{reverseTail}([])         & = []                                  \\
            \term{reverseTail}(x \cons xs) & = x \cons \term{reverse}(xs)          \\
            s(xs)                          & = \term{reverseTail}(\term{sort}(xs)) \\
            s(\bag{2,3,1,4})               & = [1,4,3,2]                           \\
            s(\bag{2,3,1})                 & = [1, 3, 2]                           \\
            s(\bag{2,3})                   & = [2, 3]                              \\
        \end{align*}
        By~\cref{sort:o2s2o} we can use $\term{sort}$ to construct $\leqs$ which would be
        equivalent to $\leq$. However, the $\leqs$ constructed by $s$ would also be equivalent
        to $\leq$. This is because $s$ sorts 2-element list correctly, despite $s \neq \term{sort}$.
        Since two different sections satisfying $\isheadleast$ maps to the same total order,
        there cannot be a full equivalence.
    \end{proof}
\end{toappendix}

\begin{definition}[$\istailsort$]
    \label{def:tail-sort}
    A section $s$ satisfies $\istailsort$ iff
    for all $x, xs$,
    \[
        \inimage{x \cons xs} \to \inimage{xs}
    \]
\end{definition}

This says that $s$-``sorted'' lists are downwards-closed under cons-ing, that is,
the tail of an $s$-``sorted'' list is also $s$-``sorted''.
%
To prove the correctness of our axioms,
first we need to show that a section $s$ satisfying
$\isheadleast$ and $\istailsort$ is equal to insertion sort parameterized by
the $\leqs$ constructed from $s$.
%
In fact, the axioms we have introduced are equivalent to the standard inductive characterization of sorted lists,
found in textbooks, such as in~\cite{appelVerifiedFunctionalAlgorithms2023}.

\begin{code}
data Sorted ($\leq$ : A -> A -> UU) : List A -> UU where
  sorted-nil : Sorted []
  sorted-$\eta$ : forall x -> Sorted [ x ]
  sorted-cons : forall x y zs -> x $\leq$ y
     -> Sorted (y cons zs) -> Sorted (x cons y cons zs)
\end{code}
Note that $\term{Sorted}_{\leq}(xs)$ is a proposition for every $xs$,
and forces the list $xs$ to be permuted in a unique way.
\begin{lemma}
    Given an order $\leq$, for any $xs, ys : \LL(A)$,
    $q(xs) = q(ys) \land \term{Sorted}_{\leq}(xs) \land \term{Sorted}_{\leq}(ys) \to xs = ys$.
\end{lemma}

Insertion sort by $\leq$ always produces lists that satisfy $\term{Sorted}_{\leq}$.
Functions that also produce lists satisfying $\term{Sorted}_{\leq}$ are equal to insertion sort
by function extensionality.

\begin{proposition}\label{sort:sort-uniq}
    Given an order $\leq$,
    if a section $s$ always produces sorted list, i.e. $\forall xs.\,\term{Sorted}_{\leq}(s(xs))$,
    $s$ is equal to insertion sort by $\leq$.
\end{proposition}
\noindent
Finally, this gives us correctness of our axioms.

\begin{proposition}\label{sort:well-behave-sorts}
    Given a section $s$ that satisfies $\isheadleast$ and $\istailsort$,
    and $\leqs$ the order derived from $s$, then for all $xs : \MM(A)$,
    it holds that $\term{Sorted}_{\leqs}(s(xs))$.
    %
    Equivalently, for all lists $xs : \LL(A)$,
    it holds that
    $xs \in \im{s}$ iff $\term{Sorted}_{\leqs}(xs)$.
\end{proposition}
\begin{proof}
    We inspect the length of $xs : \MM(A)$.
    For lengths 0 and 1, this holds trivially.
    Otherwise, we proceed by induction:
    given a $xs : \MM(A)$ of length $\geq 2$, let $s(xs) = x \cons y \cons ys$.
    We need to show
    $x \leqs y \land \term{Sorted}_{\leqs}(y \cons ys)$ to construct
    $\term{Sorted}_{\leqs}(x \cons y \cons ys)$.
    By $\isheadleast$, we have $x \leqs y$, and by $\istailsort$, we
    inductively prove $\term{Sorted}_{\leqs}(y \cons ys)$.
\end{proof}

\begin{lemma}\label{sort:s2o2s}
    Given a decidable total order $\leq$ on $A$, we can construct
    a section $t_\leq$ satisfying $\isheadleast$ and $\istailsort$,
    such that, for the order $\leqs$ derived from $s$,
    we have $t_{\leqs} = s$.
\end{lemma}
\begin{proof}
    From $s$ we can construct a decidable total order $\leqs$ since $s$ satisfies
    $\isheadleast$ and $A$ has decidable equality by assumption.
    We construct $t_{\leqs}$ as insertion sort
    parameterized by $\leqs$ constructed from $s$.
    By ~\cref{sort:sort-uniq} and ~\cref{sort:well-behave-sorts}, $s = t_{\leqs}$.
\end{proof}

\begin{proposition}\label{sort:decord-to-deceq}
    Assume $A$ has a decidable total order $\leq$,
    then $A$ has decidable equality.
\end{proposition}
\begin{proof}
    We decide if $x \leq y$ and $y \leq x$, and by cases:
    \begin{itemize}
        \item
              if $x \leq y$ and $y \leq x$: by antisymmetry, $x = y$.
        \item
              if $\neg(x \leq y)$ and $y \leq x$: assuming $x = y$, have $x \leq y$,
              leading to contradiction by $\neg(x \leq y)$, hence $x \neq y$.
        \item
              if $x \leq y$ and $\neg(y \leq x)$: similar to the previous case.
        \item
              if $\neg(x \leq y)$ and $\neg(y \leq x)$: by totality, either
              $x \leq y$ or $y \leq x$, which leads to a contradiction.
    \end{itemize}
\end{proof}
\noindent
We can now state and prove our main theorem.
\begin{definition}[Sorting function]
    \leavevmode
    A sorting function is a section $s : \MM(A) \to \LL(A)$ to
    the canonical surjection $q : \LL(A) \twoheadrightarrow \MM(A)$ satisfying two axioms:
    \begin{itemize}[leftmargin=*]
        \item $\isheadleast$:
              \(\,
              \inimage{x \cons xs} \land y \in x \cons xs \to \inimage{[x, y]}
              \),
        \item $\istailsort$:
              \(\,
              \inimage{x \cons xs} \to \inimage{xs}
              \).
    \end{itemize}
\end{definition}
\begin{theorem}\label{sort:main}
    Let $\term{DecTotOrd}(A)$ be the set of decidable total orders on $A$,
    $\term{Sort}(A)$ be the set of correct sorting functions with carrier set $A$,
    and $\term{Discrete}(A)$ be a predicate which states $A$ has decidable equality.
    There is a map $o2s \colon \term{DecTotOrd}(A) \to \term{Sort}(A) \times \term{Discrete}(A)$,
    which is an equivalence.
\end{theorem}
\begin{proof}
    $o2s$ is constructed by parameterizing insertion sort with $\leq$.
    By~\cref{sort:decord-to-deceq}, $A$ is $\term{Discrete}$.
    %
    The inverse $s2o(s)$ is constructed by~\cref{def:least}, which produces
    a total order by~\cref{sort:almost-total,sort:trans},
    and a decidable total order by $\term{Discrete}(A)$.
    %
    By~\cref{sort:o2s2o} we have $s2o \comp o2s \id \idfunc$,
    and by~\cref{sort:s2o2s} we have $o2s \comp s2o \id \idfunc$,
    giving an isomorphism, hence an equivalence.
\end{proof}

\subsubsection*{Remarks}

The sorting axioms we have come up with are abstract properties of functions.
%
As a sanity check, we can verify that the colloquial correctness specification of a sorting function (starting from a
total order) matches our axioms.
%
\begin{proposition}
    \label{prop:sort-correctness}
    Assume a decidable total order $\leq$ on $A$.
    %
    A sorting algorithm is a map $\term{sort} : {\LL(A) \to \OLL(A)}$,
    that turns lists into ordered lists,
    where $\OLL(A)$ is defined as $\dsum{xs : \LL(A)}{\term{Sorted}_{\leq}(xs)}$,
    such that:
    % https://q.uiver.app/#q=WzAsMyxbMCwwLCJcXExMKEEpIl0sWzIsMCwiXFxPTEwoQSkiXSxbMSwxLCJcXE1NKEEpIl0sWzAsMSwiXFx0ZXJte3NvcnR9Il0sWzAsMiwicSIsMl0sWzEsMiwicSBcXGNvbXAgXFxwaV8xIl1d
    \[\begin{tikzcd}
            {\LL(A)} && {\OLL(A)} \\
            & {\MM(A)}
            \arrow["{\term{sort}}", from=1-1, to=1-3]
            \arrow["q"', from=1-1, to=2-2]
            \arrow["{q \comp \pi_1}", from=1-3, to=2-2]
        \end{tikzcd}\]
    Sorting functions give sorting algorithms.
\end{proposition}
\begin{proof}
    We construct a section $s:\MM(A) \to \LL(A)$,
    and set $\term{sort} \defeq s \comp q$,
    which produces ordered lists by~\cref{sort:well-behave-sorts}.
\end{proof}
 %% 3 pages
\section{Formalization}
\label{sec:formalization}

In this section, we discuss some aspects of the formalization.
The rest of the paper uses informal type theoretic language,
and is accessible without understanding any details of the formalization.
Our formalization is done in cubical Agda, which has a few differences and a few
shortcomings due to proof engineering issues.

\vc{The formalization has a few differences from the paper and a few shortcomings due to proof engineering issues.}
\vc{If this is too short, move to a subsection of~\cref{sec:discussion}}.
\vc{Powerful abstractions lend themselves to elegant formalizations.}

We note the axioms of sorting in the formalization is named differently,
$\issorted$ is $\term{is-sorted}$
$\isheadleast$ is $\term{is-head-least}$, and
$\istailsort$ is $\term{is-tail-sort}$.

\begin{table}[h]
\centering
\begin{tabular}{lll}
\hline
\textbf{Code} & \textbf{Description} & \textbf{Reference}               \\ \hline
\texttt{Cubical.Structures.Free}         & Free algebras            & (Section~\cref{sec:universal-algebra}) \\
\texttt{Cubical.Structures.Set.Mon.List} & $\term{List}$  & (Section~\cref{mon:lists}) \\
\texttt{Cubical.Structures.Set.Mon.Array} & $\term{Array}$  & (Section~\cref{mon:array})\\
\texttt{Cubical.Structures.Set.CMon.QFreeMon} & Quotiented-free monoid & (Section~\cref{cmon:qfreemon}) \\
\texttt{Cubical.Structures.Set.CMon.PList} & Quotiented-list & (Section~\cref{cmon:plist}) \\
\texttt{Cubical.Structures.Set.CMon.SList} & Swapped-list & (Section~\cref{cmon:slist}) \\
\texttt{Cubical.Structures.Set.CMon.Bag} & Bag & (Section~\cref{cmon:bag}) \\
\texttt{Cubical.Structures.Set.CMon.SList.Sort} & Sort and order relationship & (Section~\cref{sec:sorting}) \\
\texttt{Cubical.Structures.Set.CMon.SList.Sort.Equiv} & Sort and order equivalence & (Section~\cref{sort:main}) \\
\hline
\end{tabular}
\caption{Status of formalised results}
\label{tab:formalised_results}
\end{table}


 %% 1 page
\section{Discussion}
\label{sec:discussion}

We conclude by discussing some high-level observations, related work, and future directions.

\subsection*{Free commutative monoids}

The construction of finite multisets and free commutative monoids has a long history, and various authors have different
approaches to it. We refer the reader to the discussions
in~\cite{choudhuryFreeCommutativeMonoids2023,joramConstructiveFinalSemantics2023} for a detailed survey of these
constructions.
%
Our work, in particular, was motivated by the colloquial observation that:
``there is no way to represent free commutative monoids using inductive types''.
%
From the categorical point of view, this is simply the fact that the free commutative monoid endofunctor on $\Set$ is
not polynomial (doesn't preserve pullbacks).
%
This has led various authors to think about clever encodings of free commutative monoids using inductive types by adding
assumptions on the carrier set -- in particular, the assumption of total ordering on the carrier set leads to the
construction of ``fresh-lists'', by~\cite{kupkeFreshLookCommutativity2023}, which was the inspiration for our work!

Perhaps it is worth noting what this observation means for functional programmers.
%
In programming practice, it is usually the case that all user-defined types have some sort of total order on them,
either because they're finite, or they can be enumerated in some way.
%
Therefore, under these assumptions, the construction of fresh lists is a very reasonable way to represent free
commutative monoids, or finite multisets.

\subsection*{Sorting}

Sorting is a classic problem in computer science, and the functional programming view of sorting and its correctness has
been studied by various authors.
%
The simplest view of sorting is a function $\term{sort}: \LL(\Nat) \to \LL(\Nat)$,
which permutes the list and outputs an ordered list, which is studied in~\cite{appelVerifiedFunctionalAlgorithms2023}.
%
Fundamentally, this is a very extrinsic view of program verification, which is common in the \emph{Coq} community,
and further, a very special case of a more general sorting algorithm.
%
The more refined intrinsic view of correct sorting has been studied in~\cite{hinzeSortingBialgebrasDistributive2012},
and further expanded in~\cite{alexandruIntrinsicallyCorrectSorting2023}, which matches our point of view, as explained
in~\cref{prop:sort-correctness}.
%
Their work is not just about extensional correctness of sorting, but also deriving various sorting algorithms
starting from bialgebraic semantics and distributive laws.
%
Our work is complementary to theirs, in that we are not concerned with the computational content of sorting, but rather
the abstract properties of sorting functions, which are independent of a given ordering.
%
It remains to be seen how these ideas could be combined -- the abstract property of sorting, with the intrinsic essence
of sorting algortihms -- and that is a direction for future work.
%
We do want to point out a fascinating connection between our work and theirs: our observation of recovering $\leq$ from
the section, by ``sorting'' a 2-element list actually appears in a correctness proof
in~\cite[Section~4.6,pg.324]{hengleinSortingSearchingDistribution2013}!

This paper only talks about sorting lists and bags, but the abstract property of correct sorting functions could be
applied to more general inductive types! We speculate that this could lead to some interesting connections with sorting
(binary) trees, and constructions of (binary) search trees, from classical computer science.

\subsection*{Algebra}

One of the contributions of our work is also a rudimentary framework for universal algebra, but done in a more
categorical style, which lends itself to an elegant formalization in type theory.
%
We believe this framework could be improved and generalised to higher dimensions, moving from sets to groupoids,
and using a system of coherences on top of a system of equations, which we are already pursuing.
%
Groupoidyfing free (commutative) monoids to free (symmetric) monoidal groupoids is a natural next step, and its
connections to assumptions about total orders on the type of objects would be an important direction to explore.


% To conclude, our framework of universal algebra can be generalised from sets to groupoids, using a system of
% coherences on top of the system of equations.

% Hinze's work: Sorting and Searching by Distribution: From Generic Discrimination to Generic Trie, see 4.6 on page 324


% \vc{Can you sort anything that doesn't have a total order?}
% \vc{Can you sort binary trees?}

% In a sense, this problem has already been solved, first by Hinze et al. \cite{hinzeSortingBialgebrasDistributive2012}
% and later extended by Alexandru in their thesis \cite{alexandruIntrinsicallyCorrectSorting2023}.
% Their formalization is defined in terms of bialgebras, which not
% only captures the correctness of sorting algorithms purely in a categorical settings, but
% also isolate the computational essence of sorting algorithms in terms of distributive laws,
% allowing us to construct more sorting algorithms "for free". Their work are thus necessarily
% below the level of extensional equality, i.e. input-output behavior, and allow us to reason
% with the structures of the sorting algorithms themselves. Our work only concerns the correctness
% of sorting algorithms, with the goal to axiomatize sorting functions as functions satisfying
% some abstract properties, independent of a given ordering, which allows us to gain
% insight into how sorting relates to order and vice versa.
% % and its implications on axiom of choice.
% % maybe write more on its relationship to AC, in a sepreate paragraph?
 %% 2 pages

%% total: 20.5 pages

\renewcommand{\appendixsectionformat}[2]{
  {Supplementary material for Section~#1 (#2)}
}

\printbibliography
\appendix

\end{document}

%%% Local Variables:
%%% mode: context
%%% fill-column: 120
%%% TeX-master: t
%%% End:
