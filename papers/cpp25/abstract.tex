Sorting algorithms are among the most fundamental in computing science, and are frequently studied in relation to functional programming. There have been many investigations of the correctness of sorting algorithms in terms of total orders and category theory. We provide a new perspective on this topic by approaching it via universal algebra, viewing a sorting algorithm as a section (right inverse) to a surjective function from a free monoid to a free commutative monoid. This emphasizes the fact that introducing symmetry (the passage from free monoids to free commutative monoids) eliminates ordering, while sorting (the right inverse) recovers ordering.

Our first main contribution is a new axiomatization of correct sorting algorithms, using two axioms that are not expressed in terms of a pre-existing total order. Instead, the total order can be recovered from the definition of an algorithm that satisfies the axioms. Our second main contribution is a formlization of the informal intuition that commutative monoids are unordered lists. Additionally, we give new proofs of some standard results about constructions of free monoids and free commutative monoids. We formalize all of the theory in Cubical Agda.  





%Sorting algorithms are one of the most common algorithms in functional programming.
%Previous works have investigated the correctness of sorting algorithms in terms of
%total orders and category theory. 
%
%We provide another perspective on the correctness of sorting algorithms with
%universal algebra by understanding it in terms of surjective functions from
%free monoids to free commutative monoids, leveraging the power of cubical
%type theory to implement theses ideas.