% !TEX root = cpp25-sort.tex

Sorting algorithms are fundamental to computer science, and their correctness
criteria are well understood as rearranging elements of a list
according to a specified total order on the underlying set of elements.
We give a conceptual reconstruction of this point of view, by considering
sorting algorithms as abstract functions instead -- sorting functions determine a
well-behaved section (right inverse) to the canonical surjection sending a free
monoid to a free commutative monoid of its elements.
Introducing symmetry by passing from free monoids to free commutative monoids
eliminates ordering, while sorting (the right inverse) recovers ordering.
From this, we give an axiomatization of sorting which does not require a
pre-existing total order on the underlying set, and then show that there is an
equivalence between (decidable) total orders on the underlying set and correct
sorting functions.

The first part of the paper develops concepts from universal algebra and various
constructions of free monoids and free commutative monoids, which are used to
develop the second part of the paper about the axiomatization of sorting
functions.
The paper uses informal mathematical language, and comes with an accompanying
formalization in Cubical Agda.
