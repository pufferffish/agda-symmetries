\section{Application: Sorting Functions}
\label{sec:application}

We will now put to work the universal properties of our types of (ordered) lists and unordered lists,
to define operations on them systematically, which are mathematically sound, and reason about them.
%
First, we explore definitions of various operations on both free monoids and free commutative monoids.
%
By univalence (and the structure identity principle), everything henceforth holds for any presentation of free monoids
and free commutative monoids, therefore we avoid picking a specific construction.
%
We use $\FF(A)$ to denote the free monoid or free commutative monoid on $A$, $\LL(A)$ to exclusively denote the free
monoid construction, and $\MM(A)$ to exclusively denote the free commutative monoid construction.

%
For example $\term{length}$ is a common operation defined inductively for $\List$,
but usually in proof engineering, properties about $\term{length}$, e.g.
$\term{length}(xs \doubleplus ys) = \term{length}(xs) + \term{length}(ys)$,
are proven externally.
%
In our framework of free algebras, where the $\ext{(\blank)}$ operation is a correct-by-construction homomorphism,
we can define operations like $\term{length}$ directly by universal extension,
which also gives us a proof that they are homomorphisms for free.
Note, the fold operation in functional programming is the homomorphism mapping out into the monoid of endofunctions.
%
A further application of the universal property is to prove two different types are equal, by showing they both satisfy
the same universal property (see~\cref{lem:free-algebras-unique}), which is desirable especially when proving a direct
equivalence between the two types turns out to be a difficult exercise in combinatorics.

\section{Combinatorics}
\label{sec:combinatorics}

With universal properties we can structure our programs using ideas from algebra.
For example $\term{length}$ is a common operation defined inductively for $\List$,
but to state properties about $\term{length}$, e.g.
$\term{length}(xs \doubleplus ys) = \term{length}(xs) + \term{length}(ys)$,
we would need to prove them externally. With our framework we can define operations like $\term{length}$
using the $\ext{(\blank)}$ operation, which would also give us a proof that they are homomorphisms for free.
We can also use universal property to prove two different types are equal by showing they both
satisfy the same universal property, which is desirable especially if proving a direct equivalence between
the two types is very difficult.

To illustrate this, we give some examples of how some common operations can be defined
in terms of universal property.
We use $\FF(A)$ to denote the free monoid or free commutative monoid on $A$.

\subsection{Length}

Any presentation of free monoids or free commutative monoids has a $\term{length} : \FF(A) \to \Nat$ function.
$\Nat$ is a (commutative) monoid with $(0,+)$, so we can extend $\lambda x.\, 1\,:\,A \to \Nat$.
This allows us to define $\term{length}$ for any construction of free (commutative) monoid, and also
gives us a proof of $\term{length}$ being a homomorphism for free.

\begin{figure}[H]
    \centering
    \begin{tikzcd}[ampersand replacement=\&,cramped]
    	{\FF(A)} \&\& {(\mathbb{N}, 0, +)} \\
    	\\
    	A
    	\arrow["\eta_A", from=3-1, to=1-1]
    	\arrow["{\ext{(\lambda x. \, 1)}}", from=1-1, to=1-3]
    	\arrow["{\lambda x. \, 1}"', from=3-1, to=1-3]
    \end{tikzcd}
    \caption{Definition of $\term{length}$ by universal property}
    \label{fig:enter-label}
\end{figure}


\subsection{Membership}
Any presentation of free monoids or free commutative monoids has a membership predicate:
$\_\in\_ : A \to \FF(A) \to \hProp$. By fixing an element $x: A$, the element we want to define
the membership proof for, and 
assuming $A$ is a set, we can define $\yo_A(x) = \lambda y.\, x \id y : A \to \hProp$.
Since $\hProp$ forms a (commutative) monoid under $\vee$,
we can extend $\yo_A(x)$ to obtain $x \in \blank : \FF(A) \to \hProp$, giving us the membership predicate for $x$.

\begin{figure}[H]
    \centering
    % https://q.uiver.app/#q=WzAsNCxbMCwwLCJcXEZGKEEpIl0sWzIsMF0sWzAsMiwiQSJdLFszLDAsIihcXGhQcm9wLCBcXGJvdCwgXFx2ZWUpIl0sWzIsMCwiXFxldGFfQSJdLFswLDMsIih4IFxcaW4gXFxibGFuaykgLyBcXGV4dHtcXHlvX0EoeCl9Il0sWzIsMywiXFx5b19BKHgpIiwyXV0=
    \begin{tikzcd}[ampersand replacement=\&,cramped]
    	{\FF(A)} \&\& {} \& {(\hProp, \bot, \vee)} \\
    	\\
    	A
    	\arrow["{\eta_A}", from=3-1, to=1-1]
    	\arrow["{(x \in \blank) / \ext{\yo_A(x)}}", from=1-1, to=1-4]
    	\arrow["{\yo_A(x)}"', from=3-1, to=1-4]
    \end{tikzcd}
    \caption{Definition of membership proof for $x$ by universal property}
    \label{fig:enter-label}
\end{figure}


\subsection{Any and All predicate}

Any presentation of free monoids or free commutative monoids $\FF(A)$ has the
$\term{Any}$ and $\term{All}$ predicates, which allow us to lift a predicate $A \to \hProp$
to any or all elements of $xs : \FF(A)$. We note that
$\hProp$ forms a (commutative) monoid in two different ways: $(\bot,\vee)$ and $(\top,\wedge)$,
therefore given a predicate $P : A \to \hProp$, we get:
\begin{align*}
    Any(P) &= \ext{P} : \FF(A) \to (\hProp, \bot, \vee) \\
    All(P) &= \ext{P} : \FF(A) \to (\hProp, \top, \wedge)
\end{align*}

\begin{figure}[H]
    \centering
    \begin{minipage}[t]{0.49\textwidth}
        \centering
        \begin{tikzcd}[ampersand replacement=\&,cramped]
        	{\FF(A)} \&\& {(\hProp, \top, \wedge)} \\
        	\\
        	A
        	\arrow["\eta_A", from=3-1, to=1-1]
        	\arrow["{\ext{P}}", from=1-1, to=1-3]
        	\arrow["{P}"', from=3-1, to=1-3]
        \end{tikzcd}
        \caption{Definition of $\term{All}$ by universal property}
        \label{fig:enter-label}
    \end{minipage}
    \begin{minipage}[t]{0.49\textwidth}
        \centering
        \begin{tikzcd}[ampersand replacement=\&,cramped]
        	{\FF(A)} \&\& {(\hProp, \bot, \vee)} \\
        	\\
        	A
        	\arrow["\eta_A", from=3-1, to=1-1]
        	\arrow["{\ext{P}}", from=1-1, to=1-3]
        	\arrow["{P}"', from=3-1, to=1-3]
        \end{tikzcd}
        \caption{Definition of $\term{Any}$ by universal property}
        \label{fig:enter-label}
    \end{minipage}
\end{figure}

% We denote free monoid on $A$ with $\LL(A)$ and free commutative monoid on $A$ with $\MM(A)$.
% $\MM(A + B) \eqv \MM(A) \times \MM(B)$.


 %% 2 pages
% !TEX root = cpp25-sort.tex

\subsection{Total orders}
\label{sec:total-orders}

First, we recall the axioms of a total order $\leq$ on a set $A$.
\begin{definition}[Total order]
    \label{def:total-order}
    A total order on a set $A$ is a relation $\leq : A \to A \to \hProp$ that satisfies:
    \begin{itemize}
        \item reflexivity: $x \leq x$,
        \item transitivity: if $x \leq y$ and $y \leq z$, then $x \leq z$,
        \item antisymmetry: if $x \leq y$ and $y \leq x$, then $x = y$,
        \item strong-connectedness: $\forall x, y$, either $x \leq y$ or $y \leq x$.
    \end{itemize}
    Note that \emph{either-or} means that this is a (truncated) logical disjunction.
    In the context of this paper, we want to make a distinction between ``decidable total order''
    and ``total order''. A \emph{decidable} total order requires the $\leq$ relation to be decidable:
    \begin{itemize}
        \item decidability: $\forall x, y$, we have $x \leq y + \neg(x \leq y)$.
    \end{itemize}
\end{definition}
This strengthens the strong-connectedness axiom,
where we have either $x \leq y$ or $y \leq x$ merely as a proposition,
but decidability allows us to actually compute if $x \leq y$ is true.
\begin{proposition}
    \label{prop:decidable-total-order}
    In a decidable total order, it holds that ${\forall x, y}, \ps{x \leq y} + \ps{y \leq x}$.
    Further, this makes $A$ discrete, that is ${\forall x, y}, \ps{x \id y} + \ps{x \neq y}$.
\end{proposition}
%
An equivalent way to define a total order is using a binary meet operation.
\begin{definition}[Meet semi-lattice]
    \label{def:meet-semi-lattice}
    A meet semi-lattice is a set $A$ with a binary operation $\blank\meet\blank : A \to A \to A$ that is:
    \begin{itemize}
        \item idempotent: $x \meet x \id x$,
        \item associative: $(x \meet y) \meet z \id x \meet (y \meet z)$,
        \item commutative: $x \meet y \id y \meet x$.
    \end{itemize}
    A \emph{strongly-connected} meet semi-lattice further satisfies:
    \begin{itemize}
        \item strong-connectedness: $\forall x, y$, either $x \meet y \id x$ or $x \meet y \id y$.
    \end{itemize}
    A \emph{total} meet semi-lattice strengthens this to:
    \begin{itemize}
        \item totality: ${\forall x, y}, \ps{x \meet y \id x} + \ps{x \meet y \id y}$.
    \end{itemize}
\end{definition}

\begin{proposition}
    \label{prop:total-order-meet-semi-lattice}
    A total order $\leq$ on a set $A$ is equivalent to a strongly-connected meet semi-lattice structure on $A$.
    Further, a decidable total order on $A$ induces a total meet semi-lattice structure on $A$.
\end{proposition}
\begin{proofsketch}
    Given a (mere) total order $\leq$ on a set $A$,
    we define ${x \meet y} \defeq \term{if} x \leq y \term{then} x \term{else} y$.
    %
    Crucially, this map is \emph{locally-constant}, allowing us to eliminate from an $\hProp$ to an $\hSet$.
    %
    Meets satisfy the universal property of products, that is,
    ${c \leq a \meet b} \Leftrightarrow {c \leq a} \land {c \leq b}$,
    and the axioms follow by calculation using $\yo$-arguments.
    %
    Conversely, given a meet semi-lattice, we define $x \leq y \defeq x \meet y \id x$,
    which defines an $\hProp$-valued total ordering relation.
    %
    If the total order is decidable, we use the discreteness of $A$ from~\cref{prop:decidable-total-order}.
\end{proofsketch}

Finally, tying this back to~\cref{def:head-free-monoid}, we have the following for free commutative monoids.
\begin{definition}[$\term{head}$]
    \label{def:head-free-commutative-monoid}
    Assume a total order $\leq$ on a set $A$.
    We define a commutative monoid structure on $1 + A$,
    with unit \(e \defeq \inl(\ttt) : 1 + A\), and multiplication defined as:
    \[
        \begin{array}{rclcl}
            \inl(\ttt) & \oplus & b          & \defeq & b                         \\
            \inr(a)    & \oplus & \inl(\ttt) & \defeq & \inr(a)                   \\
            \inr(a)    & \oplus & \inr(b)    & \defeq & \inr(a \meet b) \enspace.
        \end{array}
    \]
    This gives a homomorphism \({\term{head} \defeq \ext{\inr}} : {\MM(A) \to 1 + A}\),
    which picks out the \emph{least} element of the free commutative monoid.
\end{definition}

\subsection{Sorting functions}
\label{sec:sorting}

The free commutative monoid is also a monoid, hence, there is a canonical monoid homomorphism
$q : \LL(A) \to \MM(A)$, which is given by $\ext{\eta_A}$.
%
Since $\MM(A)$ is (upto equivalence), a quotient of $\LL(A)$ by symmetries (or a permutation relation),
it is a surjection (in particular, a regular epimorphism in $\Set$ as constructed in type theory).
%
Concretely, $q$ simply includes the elements of $\LL(A)$ into equivalence classes of lists in $\MM(A)$,
which ``forgets'' the order that was imposed by the indexing of the list.

Classically, assuming the Axiom of Choice would allow us to construct a section (right-inverse) to the surjection $q$,
that is,
a function $s : \MM(A) \to \LL(A)$ such that $\forall x.\, q(s(x)) \id x$.
%
Or in informal terms, given the surjective inclusion into the quotient,
a section (uniformly) picks out a canonical representative for each equivalence class.
%
Constructively, does $q$ have a section? If symmetry kills the order, can it be resurrected?
\begin{figure}[H]
    \centering
    \scalebox{1.0}{
        % https://q.uiver.app/#q=WzAsMixbMCwwLCJcXExMKEEpIl0sWzMsMCwiXFxNTShBKSJdLFsxLDAsInMiLDAseyJjdXJ2ZSI6LTF9XSxbMCwxLCJxIiwwLHsiY3VydmUiOi0xfV1d
        \begin{tikzcd}[ampersand replacement=\&,cramped]
            {\LL(A)} \&\&\& {\MM(A)}
            \arrow["s", curve={height=-10pt}, from=1-4, to=1-1]
            \arrow["q", two heads, from=1-1, to=1-4]
        \end{tikzcd}
    }
    \caption{Relationship of $\LL(A)$ and $\MM(A)$}
    \label{fig:enter-label}
\end{figure}

Viewing the quotienting relation as a permutation relation (from~\cref{cmon:qfreemon}), a section $s$ to $q$ has to pick out
canonical representatives of equivalence classes generated by permutations.
%
Using $\SList$ as an example, $s(x \cons y \cons xs) \id s(y \cons x \cons xs)$ for any $x, y : A$ and $xs : \SList(A)$,
and since it must also respect $\forall xs.\,q(s(xs)) \id xs$, $s$ must preserve all the elements of $xs$.
It cannot be a trivial function such as $\lambda\,xs. []$ -- it must produce a permutation of the elements of $s$!
%
But to place these elements side-by-side in the list, $s$ must somehow impose an order on $A$
(invariant under permutation), turning unordered lists of $A$ into ordered lists of $A$.
%
Axiom of Choice (AC) giving us a section $s$ to $q$ ``for free'' is analagous to how
AC implies the well-ordering principle, which states every set can be well-ordered.
%
If we assumed AC our problem would be trivial!
%
Instead we study how to constructively define such a section, and in fact,
that is exactly the extensional view of a sorting algorithm,
and the implications of its existence is that $A$ can be ordered, or sorted.

\subsubsection{Section from Order}

\begin{proposition}
    Assume a decidable total order on $A$. There is a sort function $s: \MM(A) \to \LL(A)$
    which constructs a section to $q : \LL(A) \twoheadrightarrow \MM(A)$
\end{proposition}

\begin{proofsketch}
    We can construct such a sor functiont by implementing any sorting algorithm.
    In our formalization we chose insertion sort,
    because it can be defined easily using the inductive structure of $\SList(A)$ and $\List(A)$.
    To implement other sorting algorithms like mergesort,
    other representations such as $\Bag$ and $\Array$ would be preferable, as explained in~\cref{bag:rep}.
    To see how this proposition holds: $q(s(xs))$ orders an unordered list $xs$ by $s$,
    and discards the order again by $q$ --
    imposing and then forgetting an order on $xs$ simply \emph{permutes} its elements,
    which proves $q \comp s \htpy \idfunc$.
\end{proofsketch}

\redtext{This has been done before\ldots definable quotients\ldots we want to go the other way.}

\subsubsection{Order from Section}

The previous section allowed us to construct a section -- how do we know this is a \emph{correct} sort function?
%
At this point we ask: if we can construct a section from order, can we construct an order from section?
%
Indeed, just by the virtue of $s$ being a section,
we can (almost) construct a total-ordering relation on the carrier set!

\begin{definition}
    \label{def:least}
    Given a section $s$, we define:
    \[
        \begin{aligned}
            \term{least}(xs) & \defeq \term{head}(s(xs))                           \\
            x \leqs y        & \defeq \term{least}(\bag{x, y}) = \inr(x) \enspace.
        \end{aligned}
    \]
\end{definition}
%
That is, we take the two-element bag $\bag{x, y}$,
``sort'' it by $s$, and test if the $\term{head}$ element is $x$.
%
Note, this is equivalent to $x \leqs y \defeq s\bag{x, y} = [x,y]$,
because $s$ preserves length, and the second element is forced to be $y$.
%
% \begin{proposition}
%     $\leqs$ is decidable iff $A$ has decidable equality.
% \end{proposition}

\begin{proposition}
    \label{sort:almost-total}
    $\leqs$ is reflexive, antisymmetric, and total.
\end{proposition}
\begin{proof}
    For all $x$, $\term{least}(\bag{x, x})$ must be $\inr(x)$, therefore $x \leqs x$, giving reflexivity.
    For all $x$ and $y$, given $x \leqs y$ and $y \leqs x$,
    we have $\term{least}(\bag{x, y}) = \inr(x)$ and $\term{least}(\bag{y, x}) = \inr(y)$.
    Since $\bag{x, y} = \bag{y, x}$, $\term{least}(\bag{x, y}) = \term{least}(\bag{y, x})$,
    therefore we have $x = y$, giving antisymmetry.
    For all $x$ and $y$, $\term{least}(\bag{x, y})$ is merely either $\inr(x)$ or $\inr(y)$,
    therefore we have merely either $x \leqs y$ or $y \leqs x$, giving totality.
\end{proof}

Although $s$ correctly orders 2-element bags, it doesn't necessarily sort 3 or more elements --
$\leqs$ is not necessarily transitive (a counterexample is given in~\cref{prop:counterexample-transitivity}).
%
We will enforce this by imposing additional constraints on the \emph{image} of $s$.

\begin{toappendix}
    \begin{proposition}
        \label{prop:counterexample-transitivity}
        $\leqs$ is not necessarily transitive.
    \end{proposition}
    \begin{proof}
        We give a counter-example of $s$ that would violate transitivity.
        Consider this section $s : \SList(\Nat) \to \List(\Nat)$, we can define a sort function
        $\term{sort} : \SList(\Nat) \to \List(\Nat)$ which sorts $\SList(\Nat)$ ascendingly. We can use $\term{sort}$
        to construct $s$.
        \begin{align*}
            s(xs)        & = \begin{cases}
                                 \term{sort}(xs)                 & \text{if $\term{length}(xs)$ is odd} \\
                                 \term{reverse}(\term{sort}(xs)) & \text{otherwise}
                             \end{cases} \\
            s([2,3,1,4]) & = [4,3,2,1]                                                                     \\
            s([2,3,1])   & = [1,2,3]
        \end{align*}
    \end{proof}
\end{toappendix}

\begin{definition}[$\blank\in\im{s}$]
    \label{def:in-image}
    The fiber of $s$ over~$xs : \LL(A)$ is given by $\fib_{s}(xs) \defeq \dsum{ys : \MM(A)}{s(ys) = xs}$.
    %
    The image of $s$ is given by $\im{s} \defeq \dsum{xs : \LL(A)}{\Trunc[-1]{\fib_{s}(xs)}}$.
    %
    Simplifying, we say that $xs:\LL(A)$ is ``in the image of $s$'', or, $xs \in \im{s}$,
    if there merely exists a $ys:\MM(A)$ such that $s(ys) = xs$.
\end{definition}

If $s$ \emph{were} a sort function, the image of $s$ would be the set of $s$-``sorted'' lists,
therefore $\inimage{xs}$ would imply $xs$ is a correctly $s$-``sorted'' list.
%
First, we note that the 2-element case is correct.
%
\begin{proposition}
    \label{sort:sort-to-order}
    $x \leqs y$ \; iff \; $\inimage{[x, y]}$.
\end{proposition}
%
\noindent Then, we state the first axiom on $s$.
\begin{definition}[$\isheadleast$]
    \label{sort:head-least}
    A section $s$ satisfies $\isheadleast$ iff for all $x, y, xs$:
    \[
        y \in x \cons xs \; \land \; \inimage{x \cons xs} \; \to \; \inimage{[x, y]}
        \enspace.
    \]
\end{definition}
\noindent
We use the definition of list membership from~\cref{def:membership}.
The $\in$ symbol is intentionally overloaded
to make the axiom look like a logical ``cut'' rule.
Inforamlly, it says that the head of an $s$-``sorted'' list
is the least element of the list.

\begin{proposition}
    \label{prop:order-to-sort-head-least}
    If $A$ has a total order $\leq$,
    insertion sort defined using $\leq$ satisfies $\isheadleast$.
\end{proposition}

\begin{proposition}
    \label{sort:trans}
    If $s$ satisfies $\isheadleast$, $\leqs$ is transitive.
\end{proposition}
\begin{proof}
    Given $x \leqs y$ and $y \leqs z$, we want to show $x \leqs z$.
    Consider the 3-element bag $\bag{x,y,z} : \MM(A)$.
    %
    Let $u$ be $\term{least}(\bag{x,y,z})$,
    by~\cref{sort:head-least} and~\cref{sort:sort-to-order},
    we have $u \leqs x \land u \leqs y \land u \leqs z$.
    %
    Since $u \in \bag{x,y,z}$, $u$ must be one of the elements.
    %
    If $u = x$ we have $x \leqs z$.
    If $u = y$ we have $y \leqs x$,
    and since $x \leqs y$ and $y \leqs z$ by assumption,
    we have $x = y$ by antisymmetry, and then we have $x \leqs z$ by substitution.
    Finally, if $u = z$, we have $z \leqs y$, and since $y \leqs z$ and $x \leqs y$ by assumption,
    we have $z = y$ by antisymmetry, and then we have $x \leqs z$ by substitution.
\end{proof}

\subsubsection{Embedding orders into sections}

Following from \cref{sort:almost-total,sort:trans},
and \cref{prop:order-to-sort-head-least},
we have shown that a section $s$ that satisfies $\isheadleast$ produces a total order
$x \leqs y \defeq \term{least}(\bag{x, y}) \id \inr(x)$,
and a total order $\leq$ on the carrier set produces a section satisfying $\isheadleast$,
constructed by sorting with $\leq$.
%
This constitutes an embedding of decidable total orders into sections satisfying $\isheadleast$.

\begin{proposition}\label{sort:o2s2o}
    Assume $A$ has a decidable total order $\leq$, we can construct a section $s$ that
    satisfies $\isheadleast$, such that $\leqs$ constructed from $s$ is equivalent
    to $\leq$.
\end{proposition}
\begin{proof}
    By the insertion sort algorithm parameterized by $\leq$,
    it holds that $\inimage{[x, y]}$ iff $x \leq y$.
    By~\cref{sort:sort-to-order}, we have $x \leqs y$ iff $x \leq y$.
    We now have a total order $x \leqs y$ equivalent to $x \leq y$.
\end{proof}

\subsubsection{Equivalence of order and sections}

We want to upgrade the embedding to an isomorphism, and it
remains to show that we can turn a section satisfying $\isheadleast$ to a total order $\leqs$,
then construct the \emph{same} section back from $\leqs$.
%
Unfortunately, this fails (see~\cref{prop:counterexample-equivalence})!
%
We then introduce our second axiom of sorting.

\begin{toappendix}
    \begin{proposition}
        \label{prop:counterexample-equivalence}
        Assume $A$ is a set with different elements, i.e. $\exists x, y: A.\,x \neq y$,
        we cannot construct a full equivalence between sections that satisfy $\isheadleast$
        and total orders on $A$.
    \end{proposition}
    \begin{proof}
        We give a counter-example of $s$ that satisfy $\isheadleast$ but is not a sort function.
        Consider the insertion sort function $\term{sort} : \MM(\Nat) \to \LL(\Nat)$
        parameterized by $\leq$:
        \begin{align*}
            \term{reverseTail}([])         & = []                                  \\
            \term{reverseTail}(x \cons xs) & = x \cons \term{reverse}(xs)          \\
            s(xs)                          & = \term{reverseTail}(\term{sort}(xs)) \\
            s(\bag{2,3,1,4})               & = [1,4,3,2]                           \\
            s(\bag{2,3,1})                 & = [1, 3, 2]                           \\
            s(\bag{2,3})                   & = [2, 3]                              \\
        \end{align*}
        By~\cref{sort:o2s2o} we can use $\term{sort}$ to construct $\leqs$ which would be
        equivalent to $\leq$. However, the $\leqs$ constructed by $s$ would also be equivalent
        to $\leq$. This is because $s$ sorts 2-element list correctly, despite $s \neq \term{sort}$.
        Since two different sections satisfying $\isheadleast$ maps to the same total order,
        there cannot be a full equivalence.
    \end{proof}
\end{toappendix}

\begin{definition}[$\istailsort$]
    \label{def:tail-sort}
    A section $s$ satisfies $\istailsort$ iff
    for all $x, xs$,
    \[
        \inimage{x \cons xs} \to \inimage{xs}
    \]
\end{definition}

This says that $s$-``sorted'' lists are downwards-closed under cons-ing, that is,
the tail of an $s$-``sorted'' list is also $s$-``sorted''.
%
To prove the correctness of our axioms,
first we need to show that a section $s$ satisfying
$\isheadleast$ and $\istailsort$ is equal to insertion sort parameterized by
the $\leqs$ constructed from $s$.
%
In fact, the axioms we have introduced are equivalent to the standard inductive characterization of sorted lists,
found in textbooks, such as in~\cite{appelVerifiedFunctionalAlgorithms2023}.

\begin{code}
data Sorted ($\leq$ : A -> A -> UU) : List A -> UU where
  sorted-nil : Sorted []
  sorted-$\eta$ : forall x -> Sorted [ x ]
  sorted-cons : forall x y zs -> x $\leq$ y
     -> Sorted (y cons zs) -> Sorted (x cons y cons zs)
\end{code}
Note that $\term{Sorted}_{\leq}(xs)$ is a proposition for every $xs$,
and forces the list $xs$ to be permuted in a unique way.
\begin{lemma}
    Given an order $\leq$, for any $xs, ys : \LL(A)$,
    $q(xs) = q(ys) \land \term{Sorted}_{\leq}(xs) \land \term{Sorted}_{\leq}(ys) \to xs = ys$.
\end{lemma}

Insertion sort by $\leq$ always produces lists that satisfy $\term{Sorted}_{\leq}$.
Functions that also produce lists satisfying $\term{Sorted}_{\leq}$ are equal to insertion sort
by function extensionality.

\begin{proposition}\label{sort:sort-uniq}
    Given an order $\leq$,
    if a section $s$ always produces sorted list, i.e. $\forall xs.\,\term{Sorted}_{\leq}(s(xs))$,
    $s$ is equal to insertion sort by $\leq$.
\end{proposition}
\noindent
Finally, this gives us correctness of our axioms.

\begin{proposition}\label{sort:well-behave-sorts}
    Given a section $s$ that satisfies $\isheadleast$ and $\istailsort$,
    and $\leqs$ the order derived from $s$, then for all $xs : \MM(A)$,
    it holds that $\term{Sorted}_{\leqs}(s(xs))$.
    %
    Equivalently, for all lists $xs : \LL(A)$,
    it holds that
    $xs \in \im{s}$ iff $\term{Sorted}_{\leqs}(xs)$.
\end{proposition}
\begin{proof}
    We inspect the length of $xs : \MM(A)$.
    For lengths 0 and 1, this holds trivially.
    Otherwise, we proceed by induction:
    given a $xs : \MM(A)$ of length $\geq 2$, let $s(xs) = x \cons y \cons ys$.
    We need to show
    $x \leqs y \land \term{Sorted}_{\leqs}(y \cons ys)$ to construct
    $\term{Sorted}_{\leqs}(x \cons y \cons ys)$.
    By $\isheadleast$, we have $x \leqs y$, and by $\istailsort$, we
    inductively prove $\term{Sorted}_{\leqs}(y \cons ys)$.
\end{proof}

\begin{lemma}\label{sort:s2o2s}
    Given a decidable total order $\leq$ on $A$, we can construct
    a section $t_\leq$ satisfying $\isheadleast$ and $\istailsort$,
    such that, for the order $\leqs$ derived from $s$,
    we have $t_{\leqs} = s$.
\end{lemma}
\begin{proof}
    From $s$ we can construct a decidable total order $\leqs$ since $s$ satisfies
    $\isheadleast$ and $A$ has decidable equality by assumption.
    We construct $t_{\leqs}$ as insertion sort
    parameterized by $\leqs$ constructed from $s$.
    By ~\cref{sort:sort-uniq} and ~\cref{sort:well-behave-sorts}, $s = t_{\leqs}$.
\end{proof}

\begin{proposition}\label{sort:decord-to-deceq}
    Assume $A$ has a decidable total order $\leq$,
    then $A$ has decidable equality.
\end{proposition}
\begin{proof}
    We decide if $x \leq y$ and $y \leq x$, and by cases:
    \begin{itemize}
        \item
              if $x \leq y$ and $y \leq x$: by antisymmetry, $x = y$.
        \item
              if $\neg(x \leq y)$ and $y \leq x$: assuming $x = y$, have $x \leq y$,
              leading to contradiction by $\neg(x \leq y)$, hence $x \neq y$.
        \item
              if $x \leq y$ and $\neg(y \leq x)$: similar to the previous case.
        \item
              if $\neg(x \leq y)$ and $\neg(y \leq x)$: by totality, either
              $x \leq y$ or $y \leq x$, which leads to a contradiction.
    \end{itemize}
\end{proof}
\noindent
We can now state and prove our main theorem.
\begin{definition}[Sorting function]
    \leavevmode
    A sorting function is a section $s : \MM(A) \to \LL(A)$ to
    the canonical surjection $q : \LL(A) \twoheadrightarrow \MM(A)$ satisfying two axioms:
    \begin{itemize}[leftmargin=*]
        \item $\isheadleast$:
              \(\,
              \inimage{x \cons xs} \land y \in x \cons xs \to \inimage{[x, y]}
              \),
        \item $\istailsort$:
              \(\,
              \inimage{x \cons xs} \to \inimage{xs}
              \).
    \end{itemize}
\end{definition}
\begin{theorem}\label{sort:main}
    Let $\term{DecTotOrd}(A)$ be the set of decidable total orders on $A$,
    $\term{Sort}(A)$ be the set of correct sorting functions with carrier set $A$,
    and $\term{Discrete}(A)$ be a predicate which states $A$ has decidable equality.
    There is a map $o2s \colon \term{DecTotOrd}(A) \to \term{Sort}(A) \times \term{Discrete}(A)$,
    which is an equivalence.
\end{theorem}
\begin{proof}
    $o2s$ is constructed by parameterizing insertion sort with $\leq$.
    By~\cref{sort:decord-to-deceq}, $A$ is $\term{Discrete}$.
    %
    The inverse $s2o(s)$ is constructed by~\cref{def:least}, which produces
    a total order by~\cref{sort:almost-total,sort:trans},
    and a decidable total order by $\term{Discrete}(A)$.
    %
    By~\cref{sort:o2s2o} we have $s2o \comp o2s \id \idfunc$,
    and by~\cref{sort:s2o2s} we have $o2s \comp s2o \id \idfunc$,
    giving an isomorphism, hence an equivalence.
\end{proof}

\subsubsection*{Remarks}

The sorting axioms we have come up with are abstract properties of functions.
%
As a sanity check, we can verify that the colloquial correctness specification of a sorting function (starting from a
total order) matches our axioms.
%
\begin{proposition}
    \label{prop:sort-correctness}
    Assume a decidable total order $\leq$ on $A$.
    %
    A sorting algorithm is a map $\term{sort} : {\LL(A) \to \OLL(A)}$,
    that turns lists into ordered lists,
    where $\OLL(A)$ is defined as $\dsum{xs : \LL(A)}{\term{Sorted}_{\leq}(xs)}$,
    such that:
    % https://q.uiver.app/#q=WzAsMyxbMCwwLCJcXExMKEEpIl0sWzIsMCwiXFxPTEwoQSkiXSxbMSwxLCJcXE1NKEEpIl0sWzAsMSwiXFx0ZXJte3NvcnR9Il0sWzAsMiwicSIsMl0sWzEsMiwicSBcXGNvbXAgXFxwaV8xIl1d
    \[\begin{tikzcd}
            {\LL(A)} && {\OLL(A)} \\
            & {\MM(A)}
            \arrow["{\term{sort}}", from=1-1, to=1-3]
            \arrow["q"', from=1-1, to=2-2]
            \arrow["{q \comp \pi_1}", from=1-3, to=2-2]
        \end{tikzcd}\]
    Sorting functions give sorting algorithms.
\end{proposition}
\begin{proof}
    We construct a section $s:\MM(A) \to \LL(A)$,
    and set $\term{sort} \defeq s \comp q$,
    which produces ordered lists by~\cref{sort:well-behave-sorts}.
\end{proof}
 %% 3 pages
