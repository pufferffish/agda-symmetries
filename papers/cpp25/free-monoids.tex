% !TEX root = cpp25-sort.tex

\section{Constructions of Free Monoids}
\label{sec:monoids}

In this section, we consider various constructions of free monoids in type theory, with proofs of their universal
property.
%
Since each construction satisfies the same categorical universal property,
by~\cref{lem:free-algebras-unique},
these are canonically equivalent (hence equal, by univalence) as types (and as monoids),
allowing us to transport proofs between them.
%
Using the unviersal property allows us to define and prove our programs correct in one go,
which is used in~\cref{sec:application}.

\subsection{Lists}
\label{mon:lists}

Cons-lists (or sequences) are simple inductive datatypes, well-known to functional programmers,
and are the most common representation of free monoids in programming languages.
%
In category theory, these correspond to Kelly's notion of algebraically-free
monoids~\cite{kellyUnifiedTreatmentTransfinite1980}.
%
\begin{definition}[Lists]
    \label{def:lists}
    \leavevmode
    \begin{code}
data List (A : UU) : UU where
  nil : List A
  _cons_ : A -> List A -> List A
\end{code}
\end{definition}
%
The (universal) generators map is the singleton: $\eta_A(a) \defeq [a] \jdgeq$~\inline{a cons nil},
the identity element is the empty list~\inline{nil},
and the monoid operation $\doubleplus$ is given by concatenation.
\begin{toappendix}

    \begin{definition}[Concatenation]
        We define the concatenation operation $\doubleplus : \List(A) \to \List(A) \to \List(A)$,
        by recursion on the first argument:
        \begin{align*}
            [] \doubleplus ys           & = ys                          \\
            (x \cons xs) \doubleplus ys & = x \cons (xs \doubleplus ys)
        \end{align*}
    \end{definition}
    The proof that $\doubleplus$ satisfies monoid laws is straightforward (see the formalization for details).

    \begin{definition}[Universal extension]
        For any monoid $\str{X}$, and given a map $f : A \to X$,
        we define the extension $\ext{f} : \List(A) \to \mathfrak{X}$ by recursion on the list:
        \begin{align*}
            \ext{f}([])         & = e                       \\
            \ext{f}(x \cons xs) & =  f(x) \mult \ext{f}(xs)
        \end{align*}
    \end{definition}
\end{toappendix}

\begin{propositionrep}
    $\ext{(\blank)}$ lifts a function $f : A \to X$ to a monoid homomorphism $\ext{f} : \List(A) \to \mathfrak{X}$.
\end{propositionrep}

\begin{proof}
    To show that $\ext{f}$ is a monoid homomorphism,
    we need to show $\ext{f}(xs \doubleplus ys) = \ext{f}(xs) \mult \ext{f}(ys)$.
    We can do so by induction on $xs$.

    Case []:
    $\ext{f}([] \doubleplus ys) = \ext{f}(ys)$,
    and $\ext{f}([]) \mult \ext{f}(ys) = e \mult \ext{f}(ys) = \ext{f}(ys)$
    by definition of $\ext{(\blank)}$. Therefore, we have
    $\ext{f}([] \doubleplus ys) = \ext{f}([]) \mult \ext{f}(ys)$.

    Case $x \cons xs$:
    \begin{align*}
         & \ext{f}((x \cons xs) \doubleplus ys)                                                        \\
         & = \ext{f}(([ x ] \doubleplus xs) \doubleplus ys) & \text{by definition of concatenation}    \\
         & = \ext{f}([ x ] \doubleplus (xs \doubleplus ys)) & \text{by associativity}                  \\
         & = \ext{f}(x \cons (xs \doubleplus ys))           & \text{by definition of concatenation}    \\
         & = f(x) \mult \ext{f}(xs \doubleplus ys)          & \text{by definition of $\ext{(\blank)}$} \\
         & = f(x) \mult (\ext{f}(xs) \mult \ext{f}(ys))     & \text{by induction}                      \\
         & = (f(x) \mult \ext{f}(xs)) \mult \ext{f}(ys)     & \text{by associativity}                  \\
         & = \ext{f}(x \cons xs) \mult \ext{f}(ys)          & \text{by definition of $\ext{(\blank)}$} \\
    \end{align*}

    Therefore, $\ext{(\blank)}$ does correctly lift a function to a monoid homomorphism.
\end{proof}

\begin{propositionrep}[Universal property for List]
    $(\List(A),\eta_A)$ is the free monoid on $A$.
\end{propositionrep}

\begin{proof}
    To show that $\ext{(\blank)}$ is an inverse to $\blank \comp \eta_A$,
    we first show $\ext{(\blank)}$ is the right inverse to $\blank \comp \eta_A$.
    For all $f$ and $x$, $(\ext{f} \circ \eta_A)(x) = \ext{f}(x \cons []) = f(x) \mult e = f(x)$,
    therefore by function extensionality, for any $f$, $\ext{f} \circ \eta_A = f$,
    and $(\blank \circ \eta_A) \comp \ext{(\blank)} = id$.

    To show $\ext{(\blank)}$ is the left inverse to $\blank \comp \eta_A$, we need to prove
    for any monoid homomorphism $f : \List(A) \to \mathfrak{X}$, $\ext{(f \comp \eta_A)}(xs) = f(xs)$.
    We can do so by induction on $xs$.

    Case []: $\ext{(f \comp \eta_A)}([]) = e$ by definition of the $\ext{(\blank)}$ operation,
    and $f([]) = e$ by homomorphism properties of $f$. Therefore, $\ext{(f \comp \eta_A)}([]) = f([])$.

    Case $x \cons xs$:
    \begin{align*}
         & \ext{(f \comp \eta_A)}(x \cons xs)                                                                \\
         & = (f \comp \eta_A)(x) \mult \ext{(f \comp \eta_A)}(xs) & \text{by definition of $\ext{(\blank)}$} \\
         & = (f \comp \eta_A)(x) \mult f(xs)                      & \text{by induction}                      \\
         & = f([x]) \mult f(xs)                                   & \text{by definition of $\eta_A$}         \\
         & = f([x] \doubleplus xs)                                & \text{by homomorphism properties of $f$} \\
         & = f(x \cons xs)                                        & \text{by definition of concatenation}
    \end{align*}

    By function extensionality, $\ext{(\blank)} \comp (\blank \circ \eta_A) = id$.
    Therefore, $\ext{(\blank)}$ and $(\blank) \circ [\_]$ are inverse of each other.

    We have now shown that $(\blank) \comp \eta_A$ is an equivalence from
    monoid homomorphisms $\List(A) \to \mathfrak{X}$
    to set functions $A \to X$, and its inverse is given by $\ext{(\blank)}$, which maps set
    functions $A \to X$ to monoid homomorphisms $\List(A) \to \mathfrak{X}$. Therefore, $\List$ is indeed
    the free monoid.
\end{proof}

\subsection{Array}\label{mon:array}

An alternate (non-inductive) representation of the free monoid on a carrier set,
or alphabet $A$, is $A^{\ast}$,
the set of all finite words or strings or sequences of characters \emph{drawn} from $A$,
which was known in category theory from~\cite{dubucFreeMonoids1974}.
%
In computer science, we think of this as an \emph{array},
which is a pair of a natural number $n$, denoting the length of the array,
and a lookup (or index) function $\Fin[n] \to A$, mapping each index to an element of $A$.
%
In type theory, this is also often understood as a container~\cite{abbottCategoriesContainers2003},
where $\Nat$ is the type of shapes, and $\Fin$ is the type (family) of positions.

\begin{definition}[Arrays]
    \label{def:arrays}
    \leavevmode
    \begin{code}
Array : UU -> UU
Array A = Sg(n : Nat) (Fin n -> A)
    \end{code}
\end{definition}
\vspace*{-2em}
\noindent
For example, $(3, \lambda\{ 0 \mapsto 3, 1 \mapsto 1, 2 \mapsto 2 \})$
represents the same list as $[3, 1, 2]$.
%
The (universal) generators map is the singleton: $\eta_A(a) = (1, \lambda\{ 0 \mapsto a \})$,
the identity element is $(0, \lambda\{\})$
and the monoid operation $\doubleplus$ is array concatenation.
%
\begin{lemmarep}
    \label{array:zero-is-id}
    Zero-length arrays $(0, f)$ are contractible.
\end{lemmarep}
\begin{proof}
    We need to show $f : \Fin[0] \to A$ is equal to $\lambda\{\}$.
    %
    By function extensionality this amounts to showing for all $x : \emptyt$, $f(x) = (\lambda\{\})(x)$,
    which holds by absurdity elimination on $x$.
    %
    Therefore, any array $(0, f)$ is equal to $(0, \lambda\{\})$.
\end{proof}

\begin{definition}[Concatenation]
    The concatenation operation $\doubleplus$, %% : \Array(A) \to \Array(A) \to \Array(A)$,
    is defined below, where $\oplus : (\Fin[n] \to A) \to (\Fin[m] \to A) \to (\Fin[n+m] \to\nolinebreak A)$
    is a combine operation:
    \begin{align*}
        (n , f) \doubleplus (m , g) & = (n + m , f \oplus g)        \\
        (f \oplus g)(k)             & = \begin{cases}
                                            f(k)     & \text{if}\ k < n \\
                                            g(k - n) & \text{otherwise}
                                        \end{cases}
    \end{align*}
\end{definition}

\begin{propositionrep}%[Monoid laws]
    $(\Array(A), \doubleplus)$ is a monoid.
\end{propositionrep}

\begin{proof}
    To show $\Array$ satisfies left unit,
    we want to show $(0, \lambda\{\}) \doubleplus (n, f) = (n, f)$.
    \begin{align*}
        (0 , \lambda\{\}) \doubleplus (n , f) & = (0 + n , \lambda\{\} \oplus f)      \\
        (\lambda\{\} \oplus f)(k)             & = \begin{cases}
                                                      (\lambda\{\})(k) & \text{if}\ k < 0 \\
                                                      f(k - 0)         & \text{otherwise}
                                                  \end{cases}
    \end{align*}

    It is trivial to see the length matches: $0 + n = n$. We also need to show $\lambda\{\} \oplus f = f$.
    Since $n < 0$ for any $n : \mathbb{N}$ is impossible, $(\lambda\{\} \oplus f)(k)$ would always reduce to
    $f(k - 0) = f(k)$, therefore $(0, \lambda\{\}) \doubleplus (n, f) = (n, f)$.

    To show $\Array$ satisfies right unit,
    we want to show $(n, f) \doubleplus (0, \lambda\{\}) = (n, f)$.
    \begin{align*}
        (n, f) \doubleplus (0 , \lambda\{\}) & = (n + 0, f \oplus \lambda\{\})           \\
        (f \oplus \lambda\{\})(k)            & = \begin{cases}
                                                     f(k)                 & \text{if}\ k < n \\
                                                     (\lambda\{\})(k - 0) & \text{otherwise}
                                                 \end{cases}
    \end{align*}

    It is trivial to see the length matches: $n + 0 = n$. We also need to show $f \oplus \lambda\{\} = f$.
    We note that the type of $f \oplus \lambda\{\}$ is $\Fin[n + 0] \to A$, therefore $k$ is of the type $\Fin[n + 0]$.
    Since $\Fin[n+0] \cong \Fin[n]$, it must always hold that $k < n$, therefore $(f \oplus \lambda\{\})(k)$ must
    always reduce to $f(k)$. Thus, $(n, f) \doubleplus (0, \lambda\{\}) = (n, f)$.

    For associativity, we want to show for any array $(n, f)$, $(m, g)$, $(o, h)$,
    $((n, f) \doubleplus (m, g)) \doubleplus (o, h) = (n, f) \doubleplus ((m, g) \doubleplus (o, h))$.

    \begin{align*}
        ((n, f) \doubleplus (m, g)) \doubleplus (o, h) & = ((n + m) + o, (f \oplus g) \oplus h)                \\
        ((f \oplus g) \oplus h)(k)                     & = \begin{cases}
                                                               \begin{cases}
                f(k)     & \text{if}\ k < n \\
                g(k - n) & \text{otherwise}
            \end{cases}
                                                                              & \text{if}\ k < n + m \\
                                                               h(k - (n + m)) & \text{otherwise}
                                                           \end{cases}                    \\
        (n, f) \doubleplus ((m, g) \doubleplus (o, h)) & = (n + (m + o), f \oplus (g \oplus h))                \\
        (f \oplus (g \oplus h))(k)                     & = \begin{cases}
                                                               f(k)                               & \ \text{k < n} \\
                                                               \begin{cases}
                g(k - n)     & \text{k - n < m} \\
                h(k - n - m) & \text{otherwise} \\
            \end{cases} & \text{otherwise}
                                                           \end{cases}
    \end{align*}

    We first case split on $k < n + m$ then $k < n$.

    Case $k < n + m$, $k < n$: Both $(f \oplus (g \oplus h))(k)$ and $((f \oplus g) \oplus h)(k)$ reduce to $f(k)$.

    Case $k < n + m$, $k \geq n$: $((f \oplus g) \oplus h)(k)$ reduce to $g(k - n)$ by definition.
    To show $(f \oplus (g \oplus h))(k)$ would also reduce to $g(k - n)$, we first need to show $\neg(k < n)$,
    which follows from $k \geq n$. We then need to show $k - n < m$.
    This can be done by simply subtracting $n$ from both side on $k < n + m$, which is well defined since $k \geq n$.

    Case $k \geq n + m$: $((f \oplus g) \oplus h)(k)$ reduce to $h(k - (n + m))$ by definition.
    To show $(f \oplus (g \oplus h))(k)$ would also reduce to $h(k - (n + m))$,
    we first need to show $\neg(k < n)$, which follows from $k \geq n + m$.
    We then need to show $\neg(k - n < m)$, which also follows from $k \geq n + m$.
    We now have $(f \oplus (g \oplus h))(k) = h(k - n - m)$. Since $k \geq n + m$, $h(k - n - m)$ is well defined
    and is equal to $h(k - (n + m))$, therefore $(f \oplus (g \oplus h))(k) = (f \oplus g) \oplus h)(k) = h(k - (n + m))$.

    In all cases $(f \oplus (g \oplus h))(k) = ((f \oplus g) \oplus h)(k)$, therefore associativity holds.
\end{proof}

\begin{lemmarep}[Array cons]\label{array:eta-suc}
    Any array $(S(n), f)$ is equal to $\eta_A (f(0)) \doubleplus (n, f \comp S)$.
\end{lemmarep}

\begin{proof}
    We want to show $\eta_A (f(0)) \doubleplus (n, f \comp S) = (S(n), f)$.
    \begin{align*}
        (1, \lambda\{ 0 \mapsto f(0) \}) \doubleplus (n , f \comp S) &
        = (1 + n, \lambda\{ 0 \mapsto f(0) \} \oplus (f \comp S))                                              \\
        (\lambda\{ 0 \mapsto f(0) \} \oplus (f \comp S))(k)          & = \begin{cases}
                                                                             f(0)               & \text{if}\ k < 1 \\
                                                                             (f \comp S)(k - 1) & \text{otherwise}
                                                                         \end{cases}
    \end{align*}

    It is trivial to see the length matches: $1 + n = S(n)$. We need to show
    $(\lambda\{ 0 \mapsto f(0) \} \oplus (f \comp S)) = f$.
    We prove by case splitting on $k < 1$.
    On $k < 1$, $(\lambda\{ 0 \mapsto f(0) \} \oplus (f \comp S))(k)$ reduces to $f(0)$.
    Since, the only possible for $k$ when $k < 1$ is 0, $(\lambda\{ 0 \mapsto f(0) \} \oplus (f \comp S))(k) = f(k)$
    when $k < 1$.
    On $k \geq 1$, $(\lambda\{ 0 \mapsto f(0) \} \oplus (f \comp S))(k)$ reduces to $(f \comp S)(k - 1) = f(S(k - 1))$.
    Since $k \geq 1$, $S(k - 1) = k$, therefore $(\lambda\{ 0 \mapsto f(0) \} \oplus (f \comp S))(k) = f(k)$
    when $k \geq 1$.
    Thus, in both cases, $(\lambda\{ 0 \mapsto f(0) \} \oplus (f \comp S)) = f$.
\end{proof}

\begin{lemmarep}[Array split]\label{array:split}
    For any array $(S(n), f)$ and $(m, g)$,
    \[
        (n + m, (f \oplus g) \comp S) = (n, f \comp S) \doubleplus (m, g)
        \enspace .
    \]
\end{lemmarep}

Informally, this means given an non-empty array $xs$ and any array $ys$,
concatenating $xs$ with $ys$ then dropping the first element is the same as
dropping the first element of $xs$ then concatenating with $ys$.

\begin{proof}
    It is trivial to see both array have length $n + m$. We want to show $(f \oplus g) \comp S = (f \comp S) \oplus g$.
    \begin{align*}
        ((f \oplus g) \comp S)(k) & = \begin{cases}
                                          f(S(k))        & \text{if}\ S(k) < S(n) \\
                                          g(S(k) - S(n)) & \text{otherwise}
                                      \end{cases} \\
        ((f \comp S) \oplus g)(k) & = \begin{cases}
                                          (f \comp S)(k) & \text{if}\ k < n \\
                                          g(k - n)       & \text{otherwise}
                                      \end{cases}
    \end{align*}

    We prove by case splitting on $k < n$.
    On $k < n$, $((f \oplus g) \comp S)(k)$ reduces to $f(S(k))$ since $k < n$ implies $S(k) < S(n)$,
    and $((f \comp S) \oplus g)(k)$ reduces to $(f \comp S)(k)$ by definition, therefore they are equal.
    On $k \geq n$, $((f \oplus g) \comp S)(k)$ reduces to $g(S(k) - S(n)) = g(k - n)$,
    and $((f \comp S) \oplus g)(k)$ reduces to $g(k - n)$ by definition, therefore they are equal.
\end{proof}

\begin{definition}[Universal extension]
    Given a monoid $\mathfrak{X}$, and a map $f : A \to X$,
    we define $\ext{f} : \Array(A) \to X$, by induction on the length of the array:
    \begin{align*}
        \ext{f}(0 , g)    & = e                                    \\
        \ext{f}(S(n) , g) & = f(g(0)) \mult \ext{f}(n , g \circ S)
    \end{align*}
\end{definition}

\begin{propositionrep}
    $\ext{(\blank)}$ lifts a function $f : A \to X$ to a monoid homomorphism $\ext{f} : \Array(A) \to \mathfrak{X}$.
\end{propositionrep}

\begin{proof}
    To show that $\ext{f}$ is a monoid homomorphism,
    we need to show $\ext{f}(xs \doubleplus ys) = \ext{f}(xs) \mult \ext{f}(ys)$.
    We can do so by induction on $xs$.

    Case $(0, g)$:
    We have $g = \lambda\{\}$ by~\cref{array:zero-is-id}.
    $\ext{f}((0, \lambda\{\}) \doubleplus ys) = \ext{f}(ys)$ by left unit,
    and $\ext{f}(0, \lambda\{\}) \mult \ext{f}(ys) = e \mult \ext{f}(ys) = \ext{f}(ys)$
    by definition of $\ext{(\blank)}$. Therefore,
    $\ext{f}((0, \lambda\{\}) \doubleplus ys) = \ext{f}(0, \lambda\{\}) \mult \ext{f}(ys)$.

    Case $(S(n), g)$: Let $ys$ be $(m, h)$.
    \begin{align*}
         & \ext{f}((S(n), g) \doubleplus (m, h))                                                                              \\
         & = \ext{f}(S(n + m), g \oplus h)                                 & \text{by definition of concatenation}            \\
         & = f((g \oplus h)(0)) \mult \ext{f}(n + m, (g \oplus h) \comp S) & \text{by definition of $\ext{(\blank)}$}         \\
         & = f(g(0)) \mult \ext{f}(n + m, (g \oplus h) \comp S)            & \text{by definition of $\oplus$, and $0 < S(n)$} \\
         & = f(g(0)) \mult \ext{f}((n, g \comp S) \doubleplus (m, h))      & \text{by~\cref{array:split}}                     \\
         & = f(g(0)) \mult (\ext{f}(n, g \comp S) \mult \ext{f}(m, h)))    & \text{by induction}                              \\
         & = (f(g(0)) \mult \ext{f}(n, g \comp S)) \mult \ext{f}(m, h))    & \text{by associativity}                          \\
         & = \ext{f}(S(n), g) \mult \ext{f}(m, h))                         & \text{by definition of $\ext{(\blank)}$}         \\
    \end{align*}

    Therefore, $\ext{(\blank)}$ does correctly lift a function to a monoid homomorphism.
\end{proof}

\begin{propositionrep}[Universal property for Array]
    \label{array:univ}
    \leavevmode
    $(\Array(A),\eta_A)$ is the free monoid on $A$.
\end{propositionrep}

\begin{proofsketch}
    We need to show that $\ext{(\blank)}$ is an inverse to ${(\blank) \comp \eta_A}$.
    $\ext{f} \comp \eta_A = f$ for all set functions $f : A \to X$ holds trivially.
    To show $\ext{(f \comp \eta_A)} = f$ for all homomorphisms $f : \Array(A) \to \mathfrak{X}$,
    we need $\forall xs.\, \ext{(f \comp \eta_A)}(xs) = f(xs)$.
    \cref{array:eta-suc,array:split} allow us to do induction on arrays,
    therefore we can prove $\forall xs.\, \ext{(f \comp \eta_A)}(xs) = f(xs)$ by induction on $xs$,
    very similarly to how this was proven for $\List$.
\end{proofsketch}

\begin{proof}
    To show that $\ext{(\blank)}$ is an inverse to $\blank \comp \eta_A$,
    we first show $\ext{(\blank)}$ is the right inverse to $\blank \comp \eta_A$.
    For all $f$ and $x$, $(\ext{f} \circ \eta_A)(x) = \ext{f}(1, \lambda\{0 \mapsto x\}) = f(x) \mult e = f(x)$,
    therefore by function extensionality, for any $f$, $\ext{f} \circ \eta_A = f$,
    and $(\blank \circ \eta_A) \comp \ext{(\blank)} = id$.

    To show $\ext{(\blank)}$ is the left inverse to $\blank \comp \eta_A$, we need to prove
    for any monoid homomorphism $f : \Array(A) \to \mathfrak{X}$, $\ext{(f \comp \eta_A)}(xs) = f(xs)$.
    We can do so by induction on $xs$.

    Case $(0, g)$:
    By~\cref{array:zero-is-id} we have $g = \lambda\{\}$.
    $\ext{(f \comp \eta_A)}(0, \lambda\{\}) = e$ by definition of the $\ext{(\blank)}$ operation,
    and $f(0, \lambda\{\}) = e$ by homomorphism properties of $f$.
    Therefore, $\ext{(f \comp \eta_A)}(0, g) = f(0, g)$.

    Case $(S(n), g)$, we prove it in reverse:
    \begin{align*}
         & f(S(n), g)                                                                                                     \\
         & = f(\eta_A(g(0)) \doubleplus (n, g \comp S))                        & \text{by~\cref{array:eta-suc}}           \\
         & = f(\eta_A(g(0))) \mult f(n, g \comp S)                             & \text{by homomorphism properties of $f$} \\
         & = (f \comp \eta_A)(g(0)) \mult \ext{(f \comp \eta_A)}(n, g \comp S) & \text{by induction}                      \\
         & = \ext{(f \comp \eta_A)}(S(n), g)                                   & \text{by definition of $\ext{(\blank)}$}
    \end{align*}

    By function extensionality, $\ext{(\blank)} \comp (\blank \circ \eta_A) = id$.
    Therefore, $\ext{(\blank)}$ and $(\blank) \circ [\_]$ are inverse of each other.

    We have now shown that $(\blank) \comp \eta_A$ is an equivalence from
    monoid homomorphisms $\Array(A) \to \mathfrak{X}$
    to set functions $A \to X$, and its inverse is given by $\ext{(\blank)}$, which maps set
    functions $A \to X$ to monoid homomorphisms $\Array(A) \to \mathfrak{X}$. Therefore, $\Array$ is indeed
    the free monoid.
\end{proof}

\subsubsection*{Remark}
An alternative proof of the universal property for $\Array$ can be given by directly constructing an equivalence (of
types, and monoid structures) between $\Array(A)$ and $\List(A)$ (using $\term{tabulate}$ and $\term{lookup}$), and then
using univalence and transport (see formalization).
