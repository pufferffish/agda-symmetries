% !TEX root = cpp25-sort.tex

\subsection{Prelude}
\label{sec:prelude}

Any presentation of free monoids or free commutative monoids has a $\term{length} : \FF(A) \to \Nat$ function.
%
$\Nat$ is a monoid with $(0,+)$, and further, the $+$ operation is commutative.
%
\begin{definition}[length]
      \label{def:length}
      The length homomorphism is defined as
      \(
      \term{length} \defeq \ext{(\lambda x.\, 1)} : \FF(A) \to \Nat
      \).
\end{definition}

Going further, any presentation of free monoids or free commutative monoids has a membership predicate
${\blank\in\blank} : A \to \FF(A) \to \hProp$, for any set $A$.
%
For extension, we use the fact that $\hProp$ forms a (commutative) monoid under disjunction: $\vee$ and false: $\bot$.
%
\begin{definition}[$\in$]
      \label{def:membership}
      The membership predicate on a set $A$ for each element $x:A$ is
      \(
      {x\in\blank} \defeq \ext{\yo_{A}(x)} : {\FF(A) \to \hProp}
      \),
      where we define
      \(
      \yo_A(x) \defeq {\lambda y.\, {x \id y}} : {A \to \hProp}
      \).
\end{definition}
%
$\yo$ is formally the Yoneda map under the ``types are groupoids'' correspondence,
where $x:A$ is being sent to its representable in the Hom-groupoid (formed by the identity type), of type $\hProp$.
%
Note that the proofs of (commutative) monoid laws for $\hProp$ use equality,
which requires the use of univalence (or at least, propositional extensionality).
%
By construction, this membership predicate satisfies its homomorphic properties
(the colluquial~$\term{here}/\term{there}$ constructors for de Bruijn indices).

We note that $\hProp$ is actually one type level higher than $A$.
To make the type level explicit, $A$ is of type level $\ell$, and since $\hProp_\ell$
is the type of all types $X : \UU_\ell$ that are mere propositions, $\hProp_\ell$ has
type level $\ell + 1$. While we can reduce to the type level of $\hProp_\ell$ to $\ell$ if
we assume Voevodsky's propositional resizing axiom~\cite{voevodskyResizingRulesTheir2011},
we chose not to do so and work within a relative monad framework similar
to~\cite[Section~3]{choudhuryFreeCommutativeMonoids2023}.
In the formalization, $\ext{(\blank)}$ is type level polymorphic to accommodate for this.
We explain this further in~\cref{sec:formalization}.

More generally, any presentation of free (commutative) monoids $\FF(A)$ also supports the
$\term{Any}$ and $\term{All}$ predicates, which allow us to lift a predicate $A \to \hProp$ (on $A$),
to \emph{any} or \emph{all} elements of $xs : \FF(A)$, respectively.
%
In fact, $\hProp$ forms a (commutative) monoid in two different ways: $(\bot,\vee)$ and $(\top,\wedge)$
(disjunction and conjunction), which are the two different ways to get $\term{Any}$ and $\term{All}$, respectively.
\begin{definition}[$\term{Any}$ and $\term{All}$]
      \label{def:any-all}
      \begin{align*}
            \type{Any}(P) & \defeq \ext{P} : \FF(A) \to (\hProp, \bot, \vee)   \\
            \type{All}(P) & \defeq \ext{P} : \FF(A) \to (\hProp, \top, \wedge)
      \end{align*}
\end{definition}

There is a $\term{head}$ function on lists, which is a function that returns the first element of a non-empty list.
%
Formally, this is a monoid homomorphism from $\LL(A)$ to $1 + A$.

\begin{definition}[$\term{head}$]
      \label{def:head-free-monoid}
      The head homomorphism is defined as
      \(
      \term{head} \defeq \ext{\inr} : \LL(A) \to 1 + A
      \),
      where the monoid structure on $1 + A$ has unit
      \(
      e \defeq \inl(\ttt) : 1 + A
      \),
      and multiplication picks the leftmost element that is defined.
      \[
            \begin{array}{rclcl}
                  \inl(\ttt) & \oplus & b & \defeq & b       \\
                  \inr(a)    & \oplus & b & \defeq & \inr(a) \\
            \end{array}
      \]
\end{definition}

Note that the monoid operation $\oplus$ is not commutative --
swapping the input arguments to $\oplus$ would return the leftmost or rightmost element.
%
To make it commutative would require a canonical way to pick between two elements --
this leads us to the next section.
